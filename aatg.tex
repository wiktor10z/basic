\documentclass{article}
\usepackage[utf8]{inputenc}
\usepackage{polski}
\usepackage{amsmath}
\usepackage{anysize}
\usepackage{amssymb}
\DeclareMathOperator{\xor}{xor}
\begin{document}

Wiktor Zuba 320501\newline

Zielony hackenbush\newline
Strategią gry w hackenbush jest doprowadzenie swoim ruchem do stanu w którym xor drzewowy obliczany jako $\xor\limits_{i\in I}(d_i+1)$,
(gdzie $I$ to zbiór synów wierzchołka, a $d_i$ to xor wierzchołka $i$, dodawanie w systemie binarnym) jest równy $0$ w korzeniu.\newline
Dowód:\newline
Jeśli za każdym razem uda nam się, żeby po naszym ruchu xor był równy $0$ (a po ruchu przeciwnika nie), to ponieważ gra kończy się w skończonym czasie,
a puste drzewo (w przypadku lasu w całym dowodzie łączymy korzenie wszystkich drzew w jeden wierzchołek, co nie zmienia zadania) ma xor zerowy,
to jest to strategia wygrywająca.\newline
Przeciwnik nie może popsuć naszej strategii: jeśli zmienimy xor jednego z poddrzew, to zmieniamy i xor w jego korzeniu,
i tak rekurencyjnie do korzenia całego drzewa. Jedynym możliwym ruchem jest wycięcie jednej krawędzi (wraz z całym poddrzewem),
co zmienia xor w wierchołku z którego wychodziła ta krawędź. Ponieważ wraz z ruchem musimy zmienić xor w korzeniu,
to przeciwnik nie może uzyskać xor=0 po swoim ruchu.\newline
Lemat: Zawsze możemy zmniejszyć xor wierchołka (do dowolnej mniejszej wartości) wycinając coś w poddrzewie w nim się zaczynającym (a więc w szczególności wyzerować xor w korzeniu).\newline
Dowód lematu (i jednocześnie algorytm grania):\newline
W wierzchołku mamy xor postaci $a1x$, a chcemy mieć $a0y$ - gdzie $x$ i $y$ reprezentują zapis $n$ najmniej istotnych bitów.\newline
Wybieramy takie poddrzewo $k\in I$, że $d_k+1$ ma $1$ na $n+1$ miejscu od końca (musi takie istnieć inaczej xor tego miejsca równy $1$ byłby xorem po samych $0$).\newline
$d_k+1=b1c$, a chcemy, żeby było równe $b0d$, gdzie $d=\xor(c,x,y)$ (wtedy xor naddrzewa będzie równy $\xor(a1x,b1c,b0d)=\xor(a1x,b1c,b0c,x,y)=a0y$).\newline
Jeśli $b=0,d=0$, to należy usunąć całe poddrzewo odcinając krawędź prowadzącą do wierzchołka $k$.
W przeciwnym przypadku, stosujemy algortym dla poddrzewa $k$ i wartości $b0d-1$.\newline

Uwagi: Zchodząc po drzewie do liści musimy otrzymać sytuację w której $b=0,d=0$, gdyż w krawędzi liścia o xorze równym $1$
jedyną wartoścą do jakiej możemy zmniejszyć xor jest $0$.\newline
Strategią wygrywającą jest odcinanie krawędzi podanej w dowodzie.
\newpage

Wiktor Zuba 320501\newline

Czerwono-Niebieski Hackenbush\newline
Gra czerwono-niebieska nie jest symetryczna, dlatego też funkcja oceniająca gry musi być niesymetryczna
(dodatnia wartość wskazuje na stan dobry dla jednego gracza, zaś ujemna dla gracza drugiego).\newline
Własności poszukiwanej funkcji:\newline
Dla gry pustej oraz gier w których odwrócenie koloru dało by ten sam las (z dokładnością do przenumerowania drzew i poddrzew...) wartość musi być równa $0$
ponieważ gra jest zdeterminowana (skończona ilość możliwych przebiegów dla konkretnego lasu, brak remisów) i wygrywający (posiadający strategię wygrywającą)
jest gracz który rusza się jako drugi (strategy stealing - powtarzanie ruchów symetrycznie).
Jako, że naszym celem jest usunięcie wszystkich krawędzi przeciwnika (i zostawienie własnych) dowolny nasz ruch pogarsza naszą sytuację (udowodnione później).
Biorąc za $1$ wartość gry w której jest pojedyńcza krawędź gracza pierwszego i $-1$ dla pojedyńczej krawędzi gracza drugiego mając las drzew wysokości $1$
stan możemy wyliczyć dodając wartości drzew (możemy to uogólnić na wszystkie lasy). Jeżeli wartość w wierzchołku jest dodatnia ($x>0$) i krawędź z której
wychodzi ten wierzchołek należy do gracza pierwszego, to nie opłaca się mu jej usuwać przed dokonaniem zmiany w tym wierzchołku,
więc stan w tej krawędzi będzie równy $1+x$. Jeżeli wartość w wierzchołku jest ujemna, to wartość w krawędzi powinna być dodatnia mimo wszystko
(gracz może usunąć tę krawędź i wartość gry się dla niego pogorszy), ale też powinna być mniejsza niż $1$.\newline
Wśród funkcji spełniających wyżej wymienione wartości dobrze sprawdza się (po testach na kilku różnych sensownych funkcjach) funkcja dająca rezultat pomiedzy
największą wartością po ruchu gracza pierwszego (po wszystkich możliwych ruchach), a najmniejszą po ruchach gracza drugiego (podobnie),
jako że celem obu graczy jest optymalizacja funkcji w swoją stronę (w szczególności znak funkcji na końcu gry ma znak taki jak gracz wygrywający lub $0$).
Funkcja biorąc średnią arytmetyczną tych wartości przyjmuje wartości o postaci $\frac{x}{2^n}$.
Funkcja dla pojedyńczej krawędzi gracza pierwszego jest równa $1$. W grze z dwoma krawędziami - gracza pierwszego na dole i drugiego z niej wychodzącej
ma wartość $\frac{1}{2}$. Dla $k$ krawędzi gracza drugiego wychodzących z krawędzi gracza pierwszego wartość wychodzi $\frac{1}{2^k}$.
Ogólnie jako, że ruch można wykonać w którymś poddrzewie drzewa wartość dla naddrzewa może być zsyntezowana z wartości poddrzew.\newline
Wartością funkcji dla lasu drzew (lub wierzchołka wewnętrznego) jest suma wartości dla drzew (krawędzi wychodzących).
Jak wyżej wartością dla krawędzi gdy stan w wierzchołku kończącym krawędź jest tego samego znaku (lub $0$) $=x$,
to wartość w krawędzi jest równa sgn(gracza)$*(1+|x|)$. W przypadku gdy znak jest przeciwny znajdujemy takie $m$, że $-|x|+m>1$
i jako wartość zwracamy sgn(gracza)$*\frac{-|x|+m}{2^{m-1}}$ (sens takiej definicji wartości widoczny w dowodzie poniżej).\newline
Pomocniczo wprowadzam również wartość dla krawędzi (wartość w wierzchołku gdyby wychodziła z niego tylko ta krawędź).\newline
Twierdzenie: Jeżeli stan gry przed ruchem gracza jest na jego korzyść (wartość o odpowiednim znaku), to ma on strategię wygrywającą, zaś jeśli wartość stanu jest równa $0$,
to po ruchu dowolnego gracza będzie ona już od niego różna (na niekorzyść gracza), dlatego wartość $0$ jest również przegrywająca.\newline
Lemat: Dla gracza pierwszego i stanu poddrzewa postaci $\frac{x}{2^n}$ nieskracalnej może on wykonać ruch,
po którym wartość będzie niemniejsza niż $\frac{x-1}{2^n}$ ale musi być mniejsza niż $\frac{x}{2^n}$.\newline
Lemat$\Rightarrow$Twierdzenie - wartość jest większa od $0\Rightarrow$ po ruchu wartość jest nieujemna.
Wartość $\le0\Rightarrow$ po ruchu wartość jest ujemna.\newline
Dowód lematu (+ strategia ruchu dla gracza pierwszego) rekurencyjny: stan gry w wierzchołku jest równy $\frac{x}{2^n}$ w postaci nieskracalnej.
$\frac{x}{2^n}=\sum\limits_{i\in\text{krawędzie wychodzące}}\frac{y_i}{2^{n_i}}$ - wybieramy $k$ t.że. $n_k=\max_{i}{n_i}$ (oczywiście $n_k\ge n$).\newline
Jeśli $n_k=0$, to wszystkie drzewa wychodzące z lasu są monochromatyczne - jeśli nie ma żadnych naszych krawędzi,
to nie możemy wykonać ruchu - już przegraliśmy gdyż nie możemy wykonać ruchu
(ten przypadek występuje tylko dla korzenia = lasu, inaczej do niego nie wejdziemy),
jeśli zaś występują nasze krawędzie, to odcinamy jedną taką, że nic z niej nie wychodzi - wartość gry będzie równa $x-1=\frac{x-1}{2^n}$.\newline
Jeśli $n_k>0$, korzystamy z lematu dla krawędzi $k$ (Wartośc zmniejszy się o coś $>0$ i $\le\frac{1}{2^{n_k}}\le\frac{1}{2^n}$).\newline
Dowód dla krawędzi (tego samego co dla wierzchołka):$\frac{x}{2^n}=\frac{y_k}{2^{n_k}}$ - rozpatrujemy przypadki:\newline
$x\ge2^n$ $\Rightarrow$ krawędź jest nasza $\rightarrow$ korzystamy rekurencyjnie z dowodu dla wierzchołka wychodzącego i wartości $\frac{x}{2^n}-1$.\newline
$2^n>x>0$, $\Rightarrow$ krawędź jest nasza.\newline
Jeśli wszystkie krawędzie w poddrzewie wychodzącym z tej krawędzi należą do przeciwnika,
to wartość jest równa $\frac{1}{2^n}$, gdzie $n=$ilość tych krawędzi - usuwamy tę krawędź - wartość wychodzi $0$, co spełnia tezę.
W przeciwnym przypadku $\exists_k$ : $n=k+m-1,k+\frac{y}{2^m}=\frac{x}{2^m}$ $\rightarrow$ $\frac{y}{2^m}<0$ jest wartością w wierzchołku wychodzącym z tej krawędzi
- korzystamy z dowodu dla tego wierzchołka (przy generowaniu wartości występuje to samo $k$, $\frac{y-1}{2^m}\rightarrow\frac{x-1}{2^n}$ ).\newline
$0>x>-2^n$- krawędź należy do przeciwnika, $\exists_k$ : $n=k+m-1,-k+\frac{y}{2^m}=\frac{x}{2^m}$ $\rightarrow$ $\frac{y}{2^m}>0$ jest wartością w wierzchołku wychodzącym z tej krawędzi
- korzystamy z dowodu dla tego wierzchołka jak wyżej.\newline
$-2^n\ge x$ $\Rightarrow$ krawędź jest przeciwnika - korzystamy rekurencyjnie z dowodu dla wierzchołka wychodzącego i wartości $\frac{x}{2^n}+1$.\newline
Wartość musi się pogorszyć, jako, że wycięcie naszej krawędzi zmienia jej wartość z dodatniej na $0$, a wartość w naddrzewie jest monotoniczna
(ściśle rosnąca) ze względu na wartości w poddrzewach.\newline\newline


Uwagi:
W przypadku krawędzi przypadek $x=0$ nie występuje - znak jest taki jak znak właściciela krawędzi.\newline
Chodząc po drzewie w głąb dowód kończy się w momencie albo gdy z wierzchołka wychodzą jedynie poddrzewa monochromatyczne,
lub gdy z naszej krawędzi wystają jedynie krawędzie przeciwnika (czyli to samo, tylko przypadek gdy są jedynie drzewa jednokolorowe poza korzeniem).\newline
Strategią jest wycięcie tej krawędzi, która została zadeklarowana do wycięcia w jednym z tych dwóch przypadków do którego dojdziemy wędrując od korzenia lasu
(drzewa) jak w dowodzie.
"To samo $k$" wnioskujemy stąd, że $k$ dla $\frac{y}{2^m}<0$ jest wartością $1+\lfloor\frac{y}{2^m}\rfloor$ (i symetrycznie dla $>0$),
zaś by zmienić wartość funkcji podłoga musielibyśmy zmienić wartość $y$ o więcej niż $1$ (aby $\frac{y}{2^m}$ nieskracalne przekroczyło liczbę całkowitą). 


\end{document}