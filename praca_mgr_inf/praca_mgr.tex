\documentclass{pracamgr}
\usepackage{polski}
\usepackage[utf8]{inputenc}
\usepackage{amssymb}
\usepackage{amsmath}
\usepackage{amsthm}
\usepackage[pdftex]{graphicx}
\usepackage{multicol}
\usepackage{enumitem}
\usepackage{hyperref}
\usepackage{wrapfig}
\usepackage{setspace}

\author{Wiktor Zuba}

\nralbumu{320501}

\title{Efektywne algorytmy kombinatoryczne na hiperkostce z wadliwymi wierzchołkami}

\tytulang{Effective combinatorical algorithms on a hypercube with faulty vertices}

\kierunek{Informatyka}

\opiekun{prof. Wojciech Rytter\\
Instytut Informatyki}

\date{September 2017}

\dziedzina{
 11.0 Matematyka, Informatyka:\\
 11.3 Informatyka\\
}


\klasyfikacja{
 Mathematics of computing\\
 Discrete mathematics\\
 Graph theory\\
 Graph algorithms\\
}
 %05--XX Combinatorics\\
 %05Cxx Graph theory\\
 %05C85 Graph algorithms\\
 
 %68--XX Computer science\\
 %68Rxx Discrete mathematics in relation to computer science\\
 %68R10 Graph theory (including graph drawing)\\
 
 %Mathematics of computing\\
 %Discrete mathematics\\
 %Graph theory\\
 %Graph algorithms\\
 

\keywords{hiperkostka, wadliwe wierzchołki, graf z wadami, spójność, długie ścieżki, cykl Hamiltona, ścieżka indukowana, wąż w pudełku}

\newtheorem{theorem}{Twierdzenie}[chapter]
\newtheorem{note}[theorem]{Notacja}
\newtheorem{defi}[theorem]{Definicja}
\newtheorem{conjecture}[theorem]{Przypuszczenie}
\newtheorem{lemma}[theorem]{Lemat}
\newtheorem{remark}[theorem]{Uwaga}
\newtheorem{observation}[theorem]{Obserwacja}
\newtheorem{fact}[theorem]{Fakt}
\newtheorem{corollary}[theorem]{Wniosek}
\newtheorem{claim}[theorem]{Stwierdzenie}


\begin{document}
\maketitle

\begin{abstract}
W pracy przedstawiono bardzo popularną i użyteczną klasę grafów, jaką są hiperkostki n- wymiarowe,
jak również mniej regularną jednak ogólniejszą klasę nazwaną hiperkostkami z wadliwymi wierzchołkami,
powstałą po usunięciu z oryginalnego grafu pewnej niewielkiej liczby wierzchołków. Dzięki niezwykle
regularnej strukturze takich obiektów możliwe jest rozwiązywanie wielu problemów grafowych lepiej i istotnie
szybciej niż w przypadku ogólnym. W pracy zawarte jest streszczenie ważniejszych wyników z literatury oraz własne wnioski i rezultaty.
W  pracy opisałem głównie cztery grupy problemów: Badanie spójności,
w tym przedstawienie dwóch różnych algorytmów sprawdzających ją w czasie wielomianowym od wymiaru i ilości wad.
Istnienie w grafie długich cykli i ścieżek (zajmujących prawie cały graf) przy odpowiednio małych zbiorach wad.
Hamiltonowskość w szczególnych przypadkach i sformułowanie wyników ogólnych.
Własne konstrukcje możliwie najdłuższych ścieżek indukowanych, wyniki eksperymentalne oraz  ograniczenia znalezione w literaturze.
\end{abstract}

\tableofcontents

 \chapter*{Wprowadzenie}
 \addcontentsline{toc}{chapter}{Wprowadzenie}
  Jedną z najbardziej popularnych i regularnych klas grafów, są hiperkostki -- grafy wierzchołków kostek $n$--wymiarowych,
  które połączone są krawędziami odpowiadającymi krawędziom jednowymiarowym kostki.
  Jak opisał je w swojej książce \cite{Ruskey} Frank Ruskey "The queen of graphs is the hypercube." hiperkostki posiadają
  wiele pożądanych własności grafowych (cykl Hamiltona, 2-kolorowalność, ...),
  zaś dla większości pozostałych łatwo jest podać dodatkowe kryteria (cykl Eulera, planarność, ...).
  
  Graf taki jest często wykorzystywany, jako struktura sieci komputerowym, dzięki połączeniu cech względnie niskiej gęstości i dużej odporności na usterki
  w dowolnych węzłach i połączeniach.
  Badanie własności tych sieci z możliwymi usterkami jest jedną z motywacji do badania nowej klasy grafów -- hiperkostek z wadami.
  Klasa ta obejmuje hiperkostki, w których część wierzchołków i krawędzi jest oznaczona, jako wadliwe lub usunięte
  (zazwyczaj jest to ilość mała względem wielkości grafu, w szczególności wielomianowa od wymiaru kostki).  
  Pomimo, że grafy te są wciąż wykładniczo wielkie w stosunku do wymiaru, to ze względu na wyjątkowe własności hiperkostki wiele standardowych
  problemów grafowych można rozwiązać w czasie wielomianowym względem wymiaru i ilości wad.
  
  Grafy te reprezentować można również, jako zbiór wszystkich ciągów binarnych długości $n$,
  z relacją krawędzi pomiędzy dwoma ciągami różniącymi się na jednym bicie.
  Dzięki tej interpretacji chodzenie po grafie przydaje się do wielu algorytmów tekstowych. Dowolny cykl Hamiltona hiperkostki
  daje kod Greya -- przeiterowanie po wszystkich możliwych słowach binarnych określonej długości jednocześnie zmieniając za każdym razem zaledwie jeden
  bit.\newline
  
  Niniejsza praca ma na celu przedstawienie istotniejszych własności wadliwych hiperkostek, części z najważniejszych problemów grafowych
  oraz znalezionych w literaturze algorytmów rozwiązujących je w złożoności znacznie lepszej niż ta dla grafów bez specjalnej struktury.
  
  Praca składa się z pięciu rozdziałów. W pierwszym przedstawiam definicje obiektów używanych w pracy, oraz ważniejsze własności
  hiperkostki. Ze sporej ich części korzystam w późniejszych rozważaniach nad hiperkostkami z wadami.
  Stanowią też one dobry punkt odniesienia do tego, czego można oczekiwać po wprowadzeniu do grafu niewielkich zmian.
  
  W rozdziale 2 rozpatrywany jest problem spójności w hiperkostce z usuniętymi wierzchołkami, który jest niezwykle istotny dla sieci komputerowych
  z tą strukturą, w sytuacji, gdy pewne komputery mogą ulec awarii. Przedstawione są tam dwa algorytmy rozwiązujące problem w czasie wielomianowym od wymiaru kostki $n$
  i liczności zbioru wierzchołków wadliwych $F$ -- algorytm korzystający z własności $\varepsilon$--ekspansji wierzchołkowej w czasie $O(|F|^2\cdot n^{3.5})$,
  oraz drugi korzystający z lokalnej spójności w czasie $O(|F|\cdot n^3)$. Wymieniam tam również kilka problemów pobocznych rozwiązywanych przez te algorytmy.
  
  W kolejnym rozdziale wprowadzam problem bardzo długich (wielkości powyżej $2^n-2|F|-4$) ścieżek, cykli oraz par ścieżek występujących w wadliwych hiperkostkach.
  Przedstawiam tam kilka twierdzeń dających proste warunki wystarczające na ich istnienie w grafie.
  Spełnianie tych warunków pozwala na odwiedzenie prawie całego grafu poruszając się wyłącznie po ścieżce bez powtórzeń,
  a więc pozwala na przeniesienie wielu algorytmów ze zwykłej hiperkostki bez potrzeby dokonywania drastycznych modyfikacji.

  W czwartym rozdziale przedstawiam kilka wyników teoretycznych na temat istnienia cyklów Hamiltona i hamiltonowskiej wiązalności w hiperkostkach z usuniętymi
  wierzchołkami. Następnie naszkicowuje dowody hamiltonowskości dla wadliwych hiperkostek ze zbiorami wad indukującymi w hiperkostce
  podgrafy o specjalnych strukturach (mniejsza hiperkostka, ścieżka, cykl, drzewo).
  
  W ostatnim rozdziale rozpatruje problem możliwie najdłuższej ścieżki indukowanej w hiperkostce (hiperkostka z usuniętymi wierzchołkami, która jest ścieżką)
  nazywany w literaturze wężem w pudełku. Przedstawiam tam własną konstrukcję ścieżek, które dla hiperkostki wymiaru $n$ mają rozmiar równy $n+1$
  wyrazowi ciągu Fibonacciego oraz wyniki eksperymentalne dla $n\le34$ (czyli dla większej ilości niż w większości prac).
  Pod koniec rozdziału wymieniam też znalezioną w literaturze, trudniejszą, lecz dającą lepsze rezultaty konstrukcję długiego węża oraz
  ograniczenia górne.

  
  
 \chapter{Własności hiperkostki}
  \section{Podstawowe definicje}
   \begin{note}\label{[n]}
    Dla $n\in\mathbb{N}$\quad $[n]=\{0,...,n-1\}$ (zbiór pierwszych $n$ liczb naturalnych).
   \end{note}
   \subsection{Hiperkostka}
    \begin{defi}\label{hiperkostka}
     \emph{Hiperkostką wymiaru $n$ ($Q_n$)} nazwiemy graf, w którym każdy wierzchołek odpowiada ciągowi binarnemu długości $n$,
     zaś krawędzią połączone są te wierzchołki, których ciągi binarne różnią się na dokładnie jednej pozycji (rysunek).\newline
     $V(Q_n)=\{(v_{n-1},...,v_{0}):v_i\in\{0,1\}\}, E(Q_n)=\{(u,v):\sum_{i}|u_i-v_i|=1\}$
    \end{defi}
    \begin{center}
     \includegraphics[scale=0.6]{img/Q_4.jpg}
    \end{center}
    W przypadku pełnej hiperkostki
    bardzo łatwo jest określić długość najkrótszej ścieżki pomiędzy wierzchołkami --
    jest ona równa ilości pozycji, na których różnią się ciągi tych wierzchołków.\newline
    Hiperkostka jest grafem dwudzielnym, w którym jedną częścią jest zbiór wierzchołków o ciągach z parzystą liczbą jedynek,
    zaś drugą tych o ich nieparzystej liczbie.\newline
    Co więcej przy badaniu hiperkostek często dzieli się je na $n+1$ warstw, gdzie dla $i\in[n+1]$ $i$-tą warstwę stanowią te wierzchołki,
    których ciągi binarne mają dokładnie $i$ jedynek (warstwa zawiera zatem wierzchołki oddalone o $i$ od wierzchołka zerowego ($\overline{0}$)).
    W tym ujęciu części dwudzielne składają się odpowiednio z parzystych i z nieparzystych warstw.
    
    \begin{defi}\label{kierunek krawędzi}
     Ze względu na oczywistą interpretację hiperkostki, jako punkty i odcinki w przestrzeni $n$--wymiarowej możemy mówić o \emph{kierunku}
     krawędzi wychodzącej z wierzchołka, definiowanym, jako numer pozycji (lub równoważnie współrzędnej), która ulega zmianie po przejściu przez tę krawędź.
    \end{defi}
    \begin{note}\label{delta wierzcholkow}
     Dla dwóch wierzchołków hiperkostki definiujemy:
     $u\Delta v=\{i:u_i\neq v_i\}$, gdzie $(u_{n-1},...,u_{0})$ i $(v_{n-1},...,v_{0})$ to ciągi binarne wierzchołków $u$ i $v$ odpowiednio
     ($|u\Delta v|$ wyznacza odległość wierzchołków w hiperkostce).
    \end{note}
    \begin{defi}\label{numerowanie klasyczne}
     \emph{Numerowaniem klasycznym (naturalnym)} hiperkostki nazwiemy takie numerowanie $\varphi:V(Q_n)\rightarrow\{1,...,|V(Q_n)|\}$ jej wierzchołków, że
     $\varphi(v)=1+\sum_{i}v_i\cdot2^i$ (rysunek).
    \end{defi}
    \begin{center}
     \includegraphics[scale=0.6]{img/Q_4_klasyczne.jpg}
    \end{center}
    \begin{defi}\label{numerowanie warstwowe}
     \emph{Numerowaniem warstwowym} hiperkostki nazwiemy jej numerowanie w kolejności przeszukiwania grafu wszerz zaczynając od wierzchołka $\overline{0}$
     z wybieraniem sąsiadów w kolejności leksykograficznej (rysunek).
    \end{defi}
    \begin{center}
    \includegraphics[scale=0.6]{img/Q_4_warstwowe.jpg}
   \end{center}   
    \begin{observation}\label{numerowanie warstwowe 2}
     Numerowanie warstwowe to takie numerowanie $\varphi:V(Q_n)\rightarrow\{1,...,|V(Q_n)|\}$ wierzchołków hiperkostki,
     że wierzchołki z $i$-tej warstwy otrzymują numery od $\sum_{j=0}^{i-1}{n\choose j}+1$ do $\sum_{j=0}^{i}{n\choose j}$.
     W obrębie jednej warstwy numery przyznawane są przeciwnie do kolejności leksykograficznej na odwróconych słowach.
     $\varphi(v)>\varphi(u)\Leftrightarrow (\sum_{i=0}^n v_i>\sum_{i=0}^n u_i)
     \vee((\sum_{i=0}^n v_i=\sum_{i=0}^n u_i)\wedge(\sum_{i=0}^n2^{n-i}v_i<\sum_{i=0}^n2^{n-i}u_i))$
    \end{observation}
    \begin{proof}
     Indukcyjnie po warstwach.\newline
     Dla warstw $0$ i $n$ oczywiste.\newline
     Zakładając, że $i$-ta warstwa jest ponumerowana w tym porządku weźmy dwa różne wierzchołki $u,v$ z warstwy $i+1$:
     $u=(\overline{y_1},1,\overline{x}),v=(\overline{y_2},0,\overline{x})$.\newline
     Jeśli $\overline{y_1}$ zawiera same $0$, to $\overline{y_2}$ zawiera dokładnie jedną $1$, sąsiedzi tych wierzchołków z poprzedniej warstwy
     o najmniejszych numerach to odpowiednio $(\overline{0},0,x),(\overline{0},0,x)$,
     tak, więc zostaną ponumerowane, jako sąsiedzi tego samego wierzchołka, jednak $u$ otrzyma mniejszy numer, jako sąsiad mniejszy leksykograficznie.\newline
     Jeśli $\overline{y_1}$ zawiera $1$, to $\overline{y_2}$ też, więc sąsiedzi tych wierzchołków z poprzedniej warstwy
     o najmniejszych numerach to odpowiednio $(\overline{y'_1},1,\overline{x}),(\overline{y'_2},0,\overline{x})$, gdzie $\overline{y'_1}$ i $\overline{y'_2}$,
     to odpowiednio $\overline{y_1}$ i $\overline{y_2}$ z pierwszymi $1$ zamienionymi na $0$. Z założenia indukcyjnego sąsiad $u$ ma mniejszy numer niż sąsiad $v$,
     więc $u$ ma mniejszy numer niż $v$.
    \end{proof}
   \subsection{Sąsiedztwo}
    \begin{defi}\label{sasiedztwo wierzcholka}
     Dla grafu $G$ oraz wierzchołka $v\in V(G)$ definiujemy \emph{sąsiedztwo wierzchołka}, jako zbiór wierzchołków połączonych z nim krawędzią:
     $N(v)=\{u\in V(G):(u,v)\in E(G)\}$.
    \end{defi}
    \vspace*{5pt}
    \begin{defi}\label{otoczenie zbioru wierzcholkow}
     Dla grafu $G$ oraz zbioru wierzchołków $S\subseteq V(G)$ definiujemy \emph{otoczenie zbioru wierzchołków}, jako zbiór tych sąsiadów wierzchołków ze zbioru,
     które same do tego zbioru nie należą: $N(S)=(\bigcup_{v\in S}N(v))\backslash S$
    \end{defi}
    \vspace*{5pt}
    \begin{defi}\label{wnetrze zbioru wierzcholkow}
     Dla grafu $G$ oraz zbioru wierzchołków $S\subseteq V(G)$ definiujemy \emph{wnętrze zbioru wierzchołków}, jako zbiór tych wierzchołków z $S$,
     których wszyscy sąsiedzi również należą do tego zbioru: $In(S)=\{v\in S:N(v)\subseteq S\}$
    \end{defi}
   \subsection{Podgrafy}
    \begin{defi}\label{podgraf indukowany}
     Dla danego $V\subseteq V(G)$\quad $G[V]=(V,\{uv\in E(G):u,v\in V\})$ oznacza \emph{podgraf indukowany} przez podzbiór wierzchołków $V$.
    \end{defi}
    \vspace*{2pt}
    \begin{defi}\label{roznica grafow}
     Dla danego $V\subseteq V(G)$\quad $G-V=G[V(G)\backslash V]$ oznacza \emph{graf $G$ z usuniętymi wierzchołkami $V$}.
    \end{defi}
    \vspace*{2pt}
    \begin{defi}\label{kwadrat grafu}
     Dla danego grafu $G$\newline $G^2=(V(G),E(G)\cup\{uv:\exists_{w\in V(G)}uw\in E(G)\cap wv\in E(G)\})$
     oznacza \emph{kwadrat grafu}, czyli graf z dodanymi krawędziami między wierzchołkami oddalonymi o co najwyżej 2.
    \end{defi}
   \subsection{Graf z wadami}
    W grafie $G$ możemy wyróżnić niektóre wierzchołki (czasem również krawędzie) i oznaczyć jako wadliwe.
    Graf z niepustym takim wyróżnionym zbiorem wierzchołków wadliwych $F\subseteq V(G)$ nazywamy \emph{grafem z wadami} (lub \emph{grafem wadliwym}).
    
    Wadliwe wierzchołki (i/lub krawędzie) najczęściej traktowane są, jako usunięte z grafu -- mówimy w tym przypadku o grafie $G-F$.
    Wyróżnianie wadliwych wierzchołków w grafie zamiast definiowania nowego grafu jest umotywowane głównie w przypadkach,
    gdy pełny graf łatwo zdefiniować i zapisać w pamięci małej względem jego rozmiaru (np. klika, hiperkostka, graf de Bruijna),
    a zbiór wadliwych wierzchołków jest również mały.
  \section{Podstawy kombinatoryczne}
   %Wzor Stirlinga: $n!=\sqrt{2\pi n}(\frac{n}{e})^n e^{\lambda_n}$\quad $\frac{1}{12n+1}<\lambda_n<\frac{1}{12n}$\newline
   %${2m \choose m}=\frac{(2m)!}{m!\cdot m!}=
   %\frac{2\sqrt{\pi m}}{2\pi m}\cdot\frac{2^{2m}m^{2m}}{m^{2m}}\cdot\frac{e^{2m}}{e^{2m}}\cdot\frac{e^{\lambda_{2m}}}{e^{2\lambda_m}}=
   %\frac{2^{2m}}{\sqrt{\pi m}}\cdot\frac{e^{\lambda_{2m}}}{e^{2\lambda_m}}$\newline
   %${2m+1 \choose m}={2m+1 \choose m+1}=\frac{(2m+1)(2m)!}{(m+1)m!\cdot m!}=\frac{2m+1}{m+1}\cdot\frac{2^{2m}}{\sqrt{\pi m}}\cdot\frac{e^{\lambda_{2m}}}{e^{2\lambda_m}}$\newline
   %$-\frac{1}{6m}<\frac{-4}{24m+1}<\frac{-3m-1}{m(24m+1)}<\frac{-18m-1}{(24m+1)6m}=\frac{1}{24m+1}-\frac{2}{12m}<\lambda_{2m}-2\lambda_{m}<\frac{1}{24m}-\frac{2}{12m+1}<0$\newline
   %$\frac{1}{\sqrt{m+1}}=\sqrt{m(m+1)}\cdot\frac{1}{(m+1)\sqrt{m}}\le\frac{m+(m+1)}{2}\cdot\frac{1}{(m+1)\sqrt{m}}=\frac{2m+1}{2(m+1)\sqrt{m}}<\frac{1}{\sqrt{m}}$\newline
   %$ 1+\frac{1}{m+1}>e^{\frac{1}{3m}}\Rightarrow e^{-\frac{1}{6m}}>\frac{\sqrt{m+1}}{\sqrt{m+2}}>\frac{\sqrt{m}}{\sqrt{m+1}}$\newline
   %Daje to ograniczenia: $\frac{2^{2m}}{\sqrt{\pi (m+1)}}<{2m \choose m}<\frac{2^{2m}}{\sqrt{\pi m}}$
   %i $\frac{2^{2m+1}}{\sqrt{\pi (m+2)}}<{2m+1 \choose m}={2m+1 \choose m+1}<\frac{2^{2m+1}}{\sqrt{\pi m}}$\newline
   %Lub w krotszym zapisie:
   %$\frac{2^n}{\sqrt{\pi(\lfloor\frac{n}{2}\rfloor}+2)}<{n\choose \lfloor\frac{n}{2}\rfloor}<\frac{2^n}{\sqrt{\pi\cdot\lfloor\frac{n}{2}\rfloor}}$
   Poniżej dowodzę kilka równań i nierówności kombinatorycznych używanych w dalszej części pracy.\newline
   \vspace*{-20pt}    
   \begin{spacing}{2}
   \noindent
    ${2n \choose n}=\frac{2^{2n}\Gamma(n+\frac{1}{2})}{\sqrt{\pi}\Gamma(n+1)}$,\quad\quad
    ${2n+1 \choose n}={2n+1 \choose n+1}=\frac{2^{2n+1}\Gamma(n+\frac{3}{2})}{\sqrt{\pi}\Gamma(n+2)}$\newline
    $\Gamma(z)=\int\limits_{0}^{\infty}x^{z-1}e^{-x}dx$\quad
    dla $n\in\mathbb{N}$ $\Gamma(n)=(n-1)!$,\newline
    ogólniej dla $x\in\mathbb{R},x>1$ $\frac{\Gamma(x+1)}{\Gamma(x)}=x$,\quad\quad
    $\frac{\Gamma(x+\frac{1}{2})}{\Gamma(x)}<\frac{\Gamma(x+1)}{\Gamma(x+\frac{1}{2})}\Rightarrow
    \sqrt{x-\frac{1}{2}}<\frac{\Gamma(x+\frac{1}{2})}{\Gamma(x)}<\sqrt{x}$\newline
    Daje to ograniczenia:
    $\frac{2^{2n}}{\sqrt{\pi(n+\frac{1}{2})}}<{2n\choose n}<\frac{2^{2n}}{\sqrt{\pi n}}$,\quad\quad
    $\frac{2^{2n+1}}{\sqrt{\pi(n+\frac{3}{2})}}<{2n+1\choose n}={2n+1\choose n+1}<\frac{2^{2n+1}}{\sqrt{\pi(n+1)}}$\newline
    Lub równoważnie:
    $\frac{2^n}{\sqrt{\pi(\lceil\frac{n}{2}\rceil+\frac{1}{2})}}<{n\choose\lfloor\frac{n}{2}\rfloor}
    ={n\choose\lceil\frac{n}{2}\rceil}<\frac{2^n}{\sqrt{\pi\lceil\frac{n}{2}\rceil}}$
   \end{spacing}
   \begin{spacing}{1.7}
    \begin{lemma}\label{binomial sum upper bound}
     Dla $k\le\lfloor\frac{n+1}{2}\rfloor$ zachodzi ograniczenie $\sum_{i=0}^{k-1}{n\choose i}\le2^{n-1}\frac{{n\choose k}}{{n\choose \lfloor\frac{n}{2}\rfloor}}$
    \end{lemma}
    \begin{proof}
     (Uogólnienie dowodu z podobnego lematu dla $n=2m,k<m$ z \cite{LPV})\vspace*{-10pt}\newline% Lemma3.8.2
     Załóżmy najpierw, że $k<\lfloor\frac{n}{2}\rfloor$\newline
     Zdefiniujmy $c=\frac{{n\choose k}}{{n\choose \lfloor\frac{n}{2}\rfloor}}<1,t=\lfloor\frac{n}{2}\rfloor-k,$
     $A=\sum_{i=0}^{k-1}{n\choose i}, B=\sum_{i=k}^{\lfloor\frac{n}{2}\rfloor-1}{n\choose i}$\newline
     \noindent
     $\forall_{1\le i\le k}$ $\frac{{n\choose k-i}}{{n\choose \lfloor\frac{n}{2}\rfloor-i}}<\frac{{n\choose k-i+1}}{{n\choose \lfloor\frac{n}{2}\rfloor-i+1}}$
     $\Leftrightarrow$ $\frac{{n\choose k-i}}{{n\choose k-i+1}}<\frac{{n\choose \lfloor\frac{n}{2}\rfloor-i}}{{n\choose \lfloor\frac{n}{2}\rfloor-i+1}}$
     $\Leftrightarrow \frac{k-i+1}{n-k+i}<\frac{\lfloor\frac{n}{2}\rfloor-i+1}{\lceil\frac{n}{2}\rceil+i}$\newline
     (a ostatnie wynika z szeregu prostych nierówności: $\frac{k-i+1}{n-k+i}\le\frac{k-i+1}{k+i+1}
     <\frac{\lfloor\frac{n}{2}\rfloor-i+1}{\lfloor\frac{n}{2}\rfloor+i+1}\le\frac{\lfloor\frac{n}{2}\rfloor-i+1}{\lceil\frac{n}{2}\rceil+i}$)\newline
     Daje nam to ograniczenia $\forall_{1\le i\le k}$ $\frac{{n\choose k-i}}{{n\choose \lfloor\frac{n}{2}\rfloor-i}}<c$.\newline
     Suma ostatnich $t$ wyrazów szeregu $A$ jest majoryzowana przez $c\cdot B$, wcześniejszych $t$ przez $c$ razy suma ostatnich $t$ (a więc przez $c^2\cdot B$).
     Daje nam to oszacowanie $A<(c+c^2+c^3+...+c^{\lfloor\frac{k}{t}\rfloor})\cdot B<(c+c^2+c^3+...)\cdot B=\frac{c}{1-c}\cdot B$.
     Jednocześnie $A+B=\sum_{i=0}^{\lfloor\frac{n}{2}\rfloor-1}{n\choose i}<2^{n-1}$.
     $A=c\cdot A+(1-c)\cdot A=c(A+\frac{1-c}{c}A)<c\cdot(A+B)<c\cdot 2^{n-1}$.\newline
     Pozostaje udowodnić przypadki większych $k$:\newline
     Dla $n=2m,k=m$ $\sum_{i=0}^{m-1}{2m\choose i}=2^{2m-1}-\frac{1}{2}{2m\choose m}<2^{2m-1}=
     2^{n-1}\cdot\frac{{n\choose k}}{{n\choose \lfloor\frac{n}{2}\rfloor}}$\newline
     Dla $n=2m+1,k=m$ $\sum_{i=0}^{m-1}{2m+1\choose i}=2^{2m}-{2m+1\choose m}<2^{2m}=
     2^{n-1}\cdot\frac{{n\choose k}}{{n\choose \lfloor\frac{n}{2}\rfloor}}$\newline
     Dla $n=2m+1,k=m+1$ $\sum_{i=0}^{m}{2m+1\choose i}=2^{2m}=
     2^{n-1}\cdot\frac{{n\choose k}}{{n\choose \lfloor\frac{n}{2}\rfloor}}$ (jedyna nieostra nierówność)\newline
    \end{proof}
    \vspace*{-25pt}
    \begin{lemma}\label{sum of log}
     $2^n(\log_2(n)-2)\le\sum_{i=1}^{n}{n\choose i}\log_2(i)\le 2^n(\log_2(n)-\frac{1}{2})$ (dla $n>16$).
    \end{lemma}
    \vspace*{-20pt}
    \begin{proof}
     Przypadek $n=2k+1$:\newline
     Dzielę sumę na $\sum_{i=1}^{k}{n\choose i}\log_2(i)+\sum_{i=k+1}^{n}{n\choose i}\log_2(i)$.\newline
     $\sum_{i=k+1}^{n}{n\choose i}=2^{n-1}\Rightarrow 2^{n-1}(\log_2{n}-1)\le\sum_{i=k+1}^{n}{n\choose i}\log_2(i)\le2^{n-1}\log_2{n}$.\newline
     $\sum_{i=1}^{k}\log_2(i)=\log_2(k!)$(ze wzoru Stirlinga) $\ge k\log_2(k)-k\log_2(e)-\frac{1}{2}(\log_2(2\pi)+\log_2(k))
     \ge k(\log_2(k+\frac{1}{2})-2)$ (dla $k>7$) =$k(\log_2(n)-3)$\newline
     Dla $i\le k$ ${n\choose i}\le{n\choose i+1}$, $\log_2(i)<\log_2(i+1)$, więc
     $2^{n-1}(\log_2(n)-3)\le(\sum_{i=1}^{k}{n\choose i})\cdot(\frac{1}{k}\sum_{i=1}^{k}log_2(i))\le\sum_{i=1}^{k}{n\choose i}\log_2(i)
     \le(\sum_{i=1}^{k}{n\choose i})\cdot\log_2(k+\frac{1}{2})=2^{n-1}(\log_2(n)-1)$. Po połączeniu daje to:\newline
     $2^n(\log_2(n)-2)\le\sum_{i=1}^{n}{n\choose i}\log_2(i)\le2^n(\log_2(n)-\frac{1}{2})$.\newline
     Przypadek parzystego $n$:\newline
     Po wyłączeniu wyrazu środkowego można analogicznie szacować części przed i po tym wyrazie
     (po podstawieniu $2^n-{n\choose \frac{n}{2}}$ zamiast $2^n$), zaś srodkowy wyraz jest równy ${n\choose \frac{n}{2}}(\log_2(n)-1)$, a więc leży pomiędzy
     ${n\choose \frac{n}{2}}(\log_2(n)-2)$ i ${n\choose \frac{n}{2}}(\log_2(n)-\frac{1}{2})$, co kończy dowód.
    \end{proof}
   \end{spacing}
   \vspace*{-20pt}
  \section{Własność podziałowa}
   Zanim udowodnie mocną własność hiperkostki używaną w następnym rozdziale, zacznę od słabszej jednak o prostszym dowodzie.
   \begin{theorem}\label{wlasnosc podzialowa}
    Dla dowolnego podziału wierzchołków na dwa zbiory : $S\in V(G)$ i $V(Q_n)\backslash S$
    liczba krawędzi między tymi zbiorami jest co najmniej tak duża jak liczność mniejszego ze zbiorów:
    $|\{(u,v)\in E(Q_n):u\in S,v\in V(Q_n)\backslash S\}|\ge\min(|S|,|V(Q_n)\backslash S|)$.
   \end{theorem}
   \begin{proof}
    Załóżmy, że $m=|S|\le 2^{n-1}$.\newline
    Dla każdej pary $(u,v)$ wierzchołków hiperkostki wybieramy jedną $uv$--ścieżkę taką, że jest najkrótsza
    (krawędź każdego kierunku jest używana co najwyżej raz) oraz idąc od $u$ współrzędne kierunkowe krawędzi rosną.
    Każda krawędź (skierowana) w hiperkostce występuje na dokładnie $2^{n-1}$ takich ścieżek
    (dla każdego innego kierunku możemy wybrać czy występuje na ścieżce i zawsze otrzymamy poprawną ścieżkę należącą do tego zbioru).
    Na każdej ścieżce pomiędzy wierzchołkami $u\in S$ i $v\in V(Q_n)\backslash S$ występuje co najmniej jedna krawędź z interesującego nas zbioru,
    zaś wszystkich takich ścieżek jest $m\cdot(2^n-m)$. Pomiędzy $S$ i $V(Q_n)\backslash S$ jest więc co najmniej
    $\frac{m\cdot(2^n-m)}{2^{n-1}}=2m-\frac{m^2}{2^{n-1}}\ge m$ krawędzi. 
   \end{proof}
  \section{Własność ekspansji}\label{ekspansja podrozdzial}
   \begin{spacing}{1.3}
    Hiperkostka posiada także mocniejszą (z punktu widzenia zastosowań) własność zdefiniowaną i udowodnioną w tym podrozdziale.
    \begin{defi}\label{epsilon ekspansja wierzcholkowa}
     Graf $G$ posiada własność \emph{$\varepsilon$--ekspansji wierzchołkowej}, jeżeli dla każdego zbioru wierzchołków $S\subseteq V(G)$ takiego,
     że $|S|\le\frac{|V(G)|}{2}$ zachodzi $|N(S)|\ge\varepsilon\cdot|S|$
    \end{defi}
    W pracy \cite{DFGKR} w dowodzie działania algorytmu testowania spójności wykorzystywane jest posiadanie przez hiperkostkę $Q_n$ własności
    $\frac{c}{\sqrt{n}}$--ekspansji wierzchołkowej, dla pewnej stałej $c$. Poniżej przedstawiam własne dowody mające na celu pokazanie możliwie dokładnej wartości
    $c$ dla otrzymania jak najlepszych rezultatów.
    
    W dowodzie własności $\varepsilon$--ekspancji dla hiperkostki wykorzystywany jest poniższy lemat o bardzo technicznym dowodzie, który można znaleźć w pracy \cite{HAR}.
    \begin{lemma}\label{HAR1}
     Zbiór $S_l$ -- pierwszych $l$ wierzchołków hiperkostki według numerowania warstwowego posiada maksymalne wnętrze wśród zbiorów wielkości $l$.  
    \end{lemma}
    \begin{lemma}\label{S->S_k}
     Dla hiperkostki do udowodnienia własności $\varepsilon_n$--ekspansji wierzchołkowej wystarczy rozważyć zbiory $S$ postaci $S_k,k\le 2^{n-1}$.
    \end{lemma}
   \end{spacing}
   \vspace*{-20pt}
   \begin{spacing}{1.7}
    \begin{proof}
     Weźmy dowolne $S\subseteq V(G),$ $l=|S|+|N(S)|$ z Lematu \ref{HAR1} wynika, że
     $\frac{|N(S)|}{|S|}=\frac{|N(S)|+|S|}{|S|}-1\ge\frac{|S_l|}{|In(S_l)|}-1=\frac{|S_l\backslash In(S_l)|}{|In(S_l)|}\ge\frac{|N(In(S_l))|}{|In(S_l)|}$.
     Z definicji $S_l$ wynika, że $In(S_l)=S_k$ dla $k=|In(S_l)|$.\newline
     Pozostaje udowodnić, że wystarczy rozważyć te $S_k$, że $k\le2^{n-1}$\newline
     Dla $l=|N(S)|+|S|\ge(\varepsilon_n+1)\cdot 2^{n-1}$ mamy $|S|>2^{n-1}$ lub $|N(S)|\ge\varepsilon_n\cdot|S|$, wystarczy więc rozważyć przypadek
     $l<(\varepsilon_n+1)\cdot 2^{n-1}$.\newline
     Dla $n=2m+1$ weźmy $k=2^{n-1}=\sum_{i=0}^{m}{2m+1 \choose i}$, wtedy $S_k$ = pełne $m+1$ pierwszych warstw i $N(S_k)$ = warstwa $m+1$.
     Przykład ten pokazuje, że $\varepsilon_n\le\frac{{2m+1 \choose m+1}}{2^{2m}}$,
     więc $l<2^{2m}+{2m+1 \choose m+1}\Rightarrow$ $S_l$ mieści się w pierwszych $m+2$ warstwach
     $\Rightarrow$ $S_k=In(S_l)$ mieści się w pierwszych $m+1$ warstwach $\Rightarrow k\le 2^{2m}=2^{n-1}$.\newline
     Dla $n=2m$ weźmy $k=2^{n-1}=\sum_{i=0}^{m-1}{2m \choose i}+\frac{1}{2}{2m\choose m}$, wtedy $S_k$ = pełne $m$ pierwszych warstw + połowa środkowej.
     W środkowej warstwie pierwsze ${2m-1\choose m-1}=\frac{1}{2}{2m \choose m}$ wierzchołków to dokładnie te, których ciągi binarne kończą się na $1$.
     Wtedy też $S_k\cup N(S_k)$ to dokładnie pełne $m+1$ pierwszych warstw plus te wierzchołki z warstwy $m+2$, których ciągi binarne kończą się na $1$
     $\Rightarrow |N(S_k)|={2m-1\choose m}+{2m-1\choose m}={2m-1\choose m-1}+{2m-1 \choose m}={2m\choose m}$.
     Przykład ten pokazuje, że $\varepsilon_n\le\frac{{2m \choose m}}{2^{2m-1}}$,
     więc $l<2^{2m-1}+{2m \choose m}\Rightarrow$ $S_l$ mieści się w pierwszych $m+1$ warstwach plus tych wierzchołkach z warstwy $m+2$, które kończą się na $1$
     $\Rightarrow k\le2^{n-1}$.
    \end{proof}
    \begin{theorem}\label{ekspansja kostki}
     Hiperkostka $Q_n$ posiada własność $\frac{1}{\sqrt{\pi n}}$--ekspansji wierzchołkowej.
    \end{theorem}
    \vspace*{-15pt}
    \begin{proof}
     Jeśli $k=\sum_{i=0}^{r}{n\choose i}$ (pełne $r+1\le\lfloor\frac{n}{2}\rfloor+1$ warstw), to
     $\frac{|N(S_k)|}{|S_k|}=\frac{{n\choose r+1}}{\sum_{i=0}^{r}{n \choose i}}\ge\frac{{n\choose r+1}{n\choose \lfloor\frac{n}{2}\rfloor}}{2^{n-1}{n\choose r+1}}
     =\frac{{n\choose \lfloor\frac{n}{2}\rfloor}}{2^{n-1}}>\frac{2^{n}}{2^{n-1}\sqrt{\pi (\lceil\frac{n}{2}\rceil+\frac{1}{2})}}=
     \frac{2}{\sqrt{\pi (\lceil\frac{n}{2}\rceil+\frac{1}{2})}}
     \ge\frac{2\sqrt{2}}{\sqrt{\pi (n+\frac{3}{2})}}\ge\frac{2}{\sqrt{\pi n}}$ (dla $n\ge2$).\newline
     ($n=2m+1,r=m$ rozważone w \ref{S->S_k})\newline
     Jeśli $k=\sum_{i=0}^{r}{n\choose i}+{n-1 \choose r}$
     (pełne $r+1\le\lfloor\frac{n}{2}\rfloor+1$ warstw plus te wierzchołki z warstwy $r+1$, których ciągi binarne kończą się na $1$).\newline
     $|N(S_k)\cup S_k|=\sum_{i=0}^{r+1}{n\choose i}+{n-1 \choose r+1}\Rightarrow |N(S_k)|=2\cdot{n-1\choose r+1}$\newline
     $\frac{|N(S_k)|}{|S_k|}=\frac{2\cdot{n-1 \choose r+1}}{\sum_{i=0}^{r}{n\choose i}+{n-1 \choose r}}>
     \frac{2\cdot{n-1\choose r+1}{n\choose\lfloor\frac{n}{2}\rfloor}}{2^{n-1}\cdot{n\choose r+1}+{n-1\choose r}\cdot{n\choose\lfloor\frac{n}{2}\rfloor}}=
     \frac{2\cdot{n-1\choose r+1}{n\choose\lfloor\frac{n}{2}\rfloor}}{2^{n-1}\cdot({n-1\choose r}+{n-1\choose r+1})+{n-1\choose r}\cdot{n\choose\lfloor\frac{n}{2}\rfloor}}>\newline
     \frac{2\cdot{n-1\choose r+1}{n\choose\lfloor\frac{n}{2}\rfloor}}{2^{n}\cdot{n-1\choose r+1}+{n-1\choose r}\cdot{n\choose\lfloor\frac{n}{2}\rfloor}}=
     \left(\frac{2^{n-1}}{{n\choose\lfloor\frac{n}{2}\rfloor}}+\frac{{n-1\choose r}}{2\cdot{n-1\choose r+1}}\right)^{-1}>
     \left(\frac{\sqrt{\pi(n+\frac{3}{2})}}{2\sqrt{2}}+\frac{1}{2}\right)^{-1}=\frac{2\sqrt{2}}{\sqrt{\pi(n+\frac{3}{2})}+\sqrt{2}}>\frac{2}{\sqrt{\pi n}}$ (dla $n\ge 7$).\newline
     W pozostałych przypadkach można łatwo otrzymać sporo słabsze ograniczenie wiedząc tylko, że dodanie wierzchołka do $S$ zmniejszy $N(S)$ o co najwyżej $1$.\newline
     Weźmy teraz $\sum_{i=0}^{r}{n\choose i}<k<\sum_{i=0}^{r}{n\choose i}+{n-1 \choose r}$\newline
     $\frac{|N(S_k)|}{|S_k|}>\frac{{n\choose r+1}-{n-1\choose r}}{\sum_{i=0}^{r}{n\choose i}+{n-1 \choose r}}=
     \frac{{n-1\choose r+1}}{\sum_{i=0}^{r}{n\choose i}+{n-1 \choose r}}\ge
     \frac{\frac{1}{2}{n \choose r+1}}{\sum_{i=0}^{r}{n\choose i}+\frac{1}{2}{n \choose r+1}}\ge
     \left(\frac{\sqrt{\pi (n+\frac{3}{2})}}{\sqrt{2}}+1\right)^{-1}>\frac{1}{\sqrt{\pi n}}$ (dla $n\ge 7$).\newpage
     \noindent
     Analogicznie dla $\sum_{i=0}^{r}{n\choose i}+{n-1 \choose r}<k<\sum_{i=0}^{r+1}{n\choose i}$\newline
     $\frac{|N(S_k)|}{|S_k|}>\frac{2{n-1\choose r+1}-{n-1\choose r+1}}{\sum_{i=0}^{r+1}{n\choose i}}=
     \frac{{n-1\choose r+1}}{\sum_{i=0}^{r+1}{n\choose i}}>
     \frac{{n-1\choose r+1}}{2^{n-1}\frac{{n\choose r+1}}{{n\choose\lfloor\frac{n}{2}\rfloor}}+{n\choose r+1}}\ge
     \frac{{n\choose r+1}{n\choose\lfloor\frac{n}{2}\rfloor}}{(2^{n-1}+{n\choose\lfloor\frac{n}{2}\rfloor}){n\choose r+1}}=
     \frac{{n\choose\lfloor\frac{n}{2}\rfloor}}{2^{n-1}+{n\choose\lfloor\frac{n}{2}\rfloor}}>
     \left(\frac{\sqrt{\pi (n+\frac{3}{2})}}{\sqrt{2}}+1\right)^{-1}>\frac{1}{\sqrt{\pi n}}$ (dla $n\ge 7$).\newline
     Dla przypadków $n\le6$ można ręcznie sprawdzić wszystkie $2^{n-1}$ przypadków, aby również otrzymać oszacowanie $\frac{1}{\sqrt{\pi n}}$.
    \end{proof}
    \vspace*{-20pt}
    \begin{corollary}\label{ograniczenie ekspansji}
     Hiperkostka wymiaru $n$ nie posiada własności $\frac{2\sqrt{2}}{\sqrt{\pi n}}$--ekspansji wierzchołkowej.
    \end{corollary}
    \vspace*{-20pt}
    \begin{proof}
     Dla $n=2m+1$:\newline
     $\frac{|N(S_{2^{2m}})|}{|S_{2^{2m}}|}=\frac{{2m+1 \choose m+1}}{2^{2m}}<\frac{2^{2m+1}}{2^{2m}\sqrt{\pi(m+1)}}=\frac{2}{\sqrt{\pi(m+1)}}=
     \frac{2}{\sqrt{\pi(\frac{n}{2}+\frac{1}{2})}}=\frac{2\sqrt{2}}{\sqrt{\pi(n+1)}}<\frac{2\sqrt{2}}{\sqrt{\pi n}}$.\newline
     Dla $n=2m$:\newline
     $\frac{|N(S_{2^{2m-1}})|}{|S_{2^{2m-1}}|}=\frac{{2m-1 \choose m}+{2m-1\choose m}}{2^{2m-1}}=\frac{{2m \choose m}}{2^{2m-1}}<
     \frac{2^{2m}}{2^{2m-1}\sqrt{\pi m}}=\frac{2}{\sqrt{\pi m}}=\frac{2}{\sqrt{\pi\cdot\frac{n}{2}}}=\frac{2\sqrt{2}}{\sqrt{\pi n}}$.
    \end{proof}
   \end{spacing}
   \vspace{-20pt}
  \section{Inne własności}
   \subsection{Automorfizmy}
    \begin{fact}$ $\\
    \vspace{-20pt}
     \begin{itemize}
      \item Hiperkostka $Q_n$ ma dokładnie $2^n\cdot n!$ automorfizmów.
      \item Automorfizmy te można uzyskać poprzez permutację współrzędnych na $n!$ sposobów oraz wybranie na $2^n$ sposobów
       zanegowania części współrzędnych lub równoważnie -- wybrania, który wierzchołek otrzyma współrzędne $(0,...,0)$.
     \end{itemize}
    \end{fact}
    Oznacza to, że przy dowolnych rozważaniach, w których wyróżniamy dwa wierzchołki odległe o $k$, możemy założyć, że pierwszy z nich jest wierzchołkiem $\overline{0}$
    zaś drugi ma współrzędne $(0,..,0,1,..,1)$ (gdzie jedynek jest $k$), jeżeli tylko zamienimy odpowiednio współrzędne innych wierzchołków -- w tym wadliwych.
   \subsection{Kolorowanie}
   
    Jako graf dwudzielny hiperkostka jest dwukolorowalna wierzchołkowo. Po usunięciu części wierzchołków ilość potrzebnych kolorów może się tylko zmniejszyć,
    jednak jedynym grafem jednokolorowalnym jest graf bez krawędzi.
 
    $n$ kolorowanie krawędziowe $Q_n$ można uzyskać kolorując krawędzie według numeru kierunku, jaki przyjmują, jest to jednocześnie jedyne takie kolorowanie.
    Z twierdzenia Vizinga dla grafu dwudzielnego ilość potrzebnych kolorów jest równa stopniowi grafu, a to można sprawdzić w czasie liniowym od ilości
    usuniętych wierzchołków (dodatkowo musi być ich co najmniej $\frac{2^n}{n}$, aby każdy wierzchołek stracił sąsiada).
   \subsection{Ilość drzew rozpinających}
    \vspace{5pt}
    \begin{defi}
     Macierz Laplace'a grafu $G$, o $n$ wierzchołkach, to macierz $[a_{i,j}]_{i,j\in[n]}$, taka że:\quad
     $a_{i,i}=deg(v_i)$, $a_{i,j}=-1\cdot [(v_i,v_j)\in E(G)]$ (dla $i\neq j$)\quad(Oznaczenie $L(G)$)
    \end{defi}
    \vspace{5pt}
    \begin{theorem}\label{Kirchhoff}
     (Kirchhoffa)\newline
     Dla spójnego grafu o $n$ wierzchołkach ilość drzew rozpinających jest równa $\frac{1}{n}\cdot\lambda_1\cdot\lambda_2\cdot...\cdot\lambda_{n-1}$,
     gdzie $\lambda_1,\lambda_2,...,\lambda_{n-1}$ to niezerowe wartości własne jego macierzy Laplace'a (z krotnościami).
    \end{theorem}
    \vspace{5pt}
    \begin{lemma}
     Wartościami własnymi $L(Q_n)$ są liczby $2i$ dla $i\in[n+1]$, z krotnością ${n\choose i}$.
    \end{lemma}
    \begin{proof}
     Indukcyjnie -- dla $n=1$ macierz
     $\left|\begin{array}{cc}
      1&-1\\
      -1&1\\
     \end{array}\right|$
     ma wartości własne $0$ i $2$ z ortogonalnymi wektorami własnymi $[1,1]$ i $[1,-1]$.\newline
     Mając macierz $A_n=L(Q_n)$  macierz $A_{n+1}$ jest równa
     $\left|\begin{array}{cc}
      A_n+I_{2^n}&-I_{2^n}\\
      -I_{2^n}&A_n+I_{2^n}\\
     \end{array}\right|$.\newline
     Niech $v$ będzie wektorem własnym macierzy $A_n$ dla wartości własnej $\lambda$. Wtedy:\newline\newline
     $A_{n+1}\cdot
     \left|\begin{array}{c}
      v\\
      v\\
     \end{array}\right|
     =
     \left|\begin{array}{cc}
      A_n+I_{2^n}&-I_{2^n}\\
      -I_{2^n}&A_n+I_{2^n}\\
     \end{array}\right|
     \cdot
     \left|\begin{array}{c}
      v\\
      v\\
     \end{array}\right|
     =
     \left|\begin{array}{c}
      \lambda\cdot v+v-v\\
      -v+\lambda\cdot v+v\\
     \end{array}\right|
     =\lambda\cdot
     \left|\begin{array}{c}
      v\\
      v\\
     \end{array}\right|
     $\newline\newline\newline
     $A_{n+1}\cdot
     \left|\begin{array}{c}
      v\\
      -v\\
     \end{array}\right|
     =
     \left|\begin{array}{cc}
      A_n+I_{2^n}&-I_{2^n}\\
      -I_{2^n}&A_n+I_{2^n}\\
     \end{array}\right|
     \cdot
     \left|\begin{array}{c}
      v\\
      -v\\
     \end{array}\right|
     =
     \left|\begin{array}{c}
      \lambda\cdot v+v+v\\
      -v-\lambda\cdot v-v\\
     \end{array}\right|
     =(\lambda+2)\cdot
     \left|\begin{array}{c}
      v\\
      -v\\
     \end{array}\right|
     $.\newline\newline
     Ponieważ wszystkie $2^n$ wektorów własnych z poprzedniego kroku były ortogonalne, to otrzymane teraz $2^{n+1}$ również takie będzie.
     Stąd dla wartości własnej $2i$ macierzy $A_{n+1}$ otrzymujemy ${n\choose i}+{n\choose i-1}={n+1\choose i}$ ortogonalnych wektorów własnych.
    \end{proof}
    \vspace{5pt}
    \begin{corollary}
     Korzystając z twierdzenia Kirchhoffa, $Q_n$ ma $\frac{1}{2^n}\prod_{i=1}^{n}(2i)^{{n\choose i}}$ drzew rozpinających.
    \end{corollary}
    \noindent
    Macierz Laplace'a grafu pełnego o $n$ wierzchołkach jest równa $n\cdot I_n-J_n$ (gdzie $J_n$, to macierz $n\times n$ złożona z samych $1$), więc 
    ma wartość własną $n$ ($n-1$)-krotną (+ $0$ jednokrotną).\newline
    Macierz Laplace'a grafu pełnego dwudzielnego, w którym każda część dwudzielna ma po $n$ wierzchołków jest równa
    $
     \left|\begin{array}{cc}
      nI_n&-J_n\\
      -J_n&nI_n\\
     \end{array}\right|
    $, więc ma wartość własną $n$ ($2n-2$)-krotną i $2n$ jednokrotną (+ $0$ jednokrotną).\newline
   Jeśli więc porówna się logarytmy ilości drzew rozpinających ($t(G)$) tych trzech grafów o tej samej liczbie wierzchołków, to wychodzi:\newline
   $\log_2(t(K_{2^n}))=n(2^n-2)$\newline
   $\log_2(t(K_{2^{n-1},2^{n-1}}))=(n-1)(2^n-2)$\newline
   $\log(t(Q_n))=2^n-n-1+\sum_{i=1}^{n}{n\choose i}\log(i)\approx (\log_2(n))2^n$ (na podstawie lematu \ref{sum of log}).\newpage
   \subsection{Rysowanie na płaszczyźnie}
    %\begin{wrapfigure}{1}{0.3\textwidth}
     %\includegraphics[scale=0.4]{img/Q3_planar.jpg}\includegraphics[scale=0.4]{img/Q4_not_planar.jpg}
    %\end{wrapfigure}
   $Q_n$ jest grafem planarnym, jedynie dla $n\le3$.\newline
   Dla $n=4$ na płaszczyźnie powstaje minimum $8$ przecięć krawędzi,
   jest ona jednak grafem toroidalnym (daje się narysować na powierzchni torusa bez przecięć). Oba przypadki są zilustrowane poniżej.
   \begin{center}
    \includegraphics[scale=0.6]{img/Q3_planar.jpg}\includegraphics[scale=0.6]{img/Q4_not_planar.jpg}
   \end{center}
   Ogólnie $Q_n$ ma genus (liczba całkowita charakteryzująca rozmaitość topologiczną równa liczbie otworów w rozmaitości, na której daje się narysować graf bez przecięć)\newline
   równy ${\gamma(Q_n)=(n-4)2^{n-3}+1}$ (na podstawie \cite{HHH}).\newline
   
   Każda hiperkostka jest grafem jednostkowej odległości -- można go narysować tak, żeby każda krawędź miała tę samą długość euklidesową.
   Rysunek taki można łatwo dostać indukcyjnie -- malując dwie kopie rysunku $Q_{n-1}$ o tej własności w odległości $1$ w dowolnym kierunku
   (ponieważ kierunków jest continuum wiele zawsze można wybrać taki, by żadne dwa wierzchołki się nie pokrywały).
   Rysunki takie stają się jednak coraz gęstsze, jako że  trzeba zmieścić $2^n$ wierzchołków i $n2^{n-1}$ krawędzi na kole o średnicy $n$
   (obwód grafu równy $2n$), przez co stają się nieczytelne dla dużych $n$.\newline
   \includegraphics[scale=1]{img/unitary_1.jpg}
   \includegraphics[scale=1]{img/unitary_2.jpg}
   \includegraphics[scale=1]{img/unitary_3.jpg}
   \includegraphics[scale=1]{img/unitary_4.jpg}
   \includegraphics[scale=1]{img/unitary_5_1.jpg}\newline
   (Rysunki grafów jednostkowej odległości dla $n\le 5$.)
   
 \chapter{Spójność wadliwej hiperkostki}
  \textbf{Ten rozdział jest napisany w większości na podstawie \cite{DFGKR}.}
  \begin{observation}\label{spojnosc przy usunietych}
   Badanie spójności grafu $G-F$ dla spójnego grafu $G$ wystarczy ograniczyć do sprawdzienia czy wciąż istnieje ścieżka pomiędzy dowolnymi dwoma wierzchołkami,
   które oryginalnym grafie sąsiadowały z jakimś spośród usuniętych wierzchołków (czy wszystkie takie wierzchołki należą do jednej spójnej składowej).
  \end{observation}
  \begin{proof}
   Aby udowodnić spójność trzeba pokazać, że istnieje ścieżka pomiędzy dowolnymi dwoma wierzchołkami, jednak skoro w oryginalnym grafie taka ścieżka istniała,
   to w nowym grafie jedyną przeszkodą jest to, że mogły na niej występować wierzchołki, które zostały usunięte. 
   Ścieżkę taką można naprawić -- przed pierwszym odwiedzeniem i po ostatnim odwiedzeniu wierzchołka wadliwego odwiedzani są jacyś ich sąsiedzi --
   wystarczy ten fragment ścieżki zastąpić ścieżką między tymi dwoma sąsiadami, która istnieje w wadliwym grafie.
  \end{proof}
  \section{Podejście ekspansywne}\label{podejscie ekspansywne}
   Poniżej przedstawie algorytm testowania spójności bazujący na własności $\varepsilon$--ekspansji wierzchołkowej
   udowodnionej przeze mnie dla hiperkostki w podrozdziale \ref{ekspansja podrozdzial}.
   \begin{theorem}\label{Spójność ekspansywna}
    Niech graf $G$ posiada własność $\varepsilon$--ekspansji wierzchołkowej z $\varepsilon>0$ i maksymalny stopień wierzchołka $\Delta$,
    oraz dana jest wyrocznia zwracająca dla danego wierzchołka listę jego sąsiadów.
    Wtedy istnieje algorytm, który otrzymuje na wejściu zbiór $F\subseteq V(G)$ oraz $\varepsilon$
    i testuje spójność $G-F$ w czasie $O\left(\frac{|F|^2\cdot\Delta^2\cdot\log(|V(G)|)}{\varepsilon}\right)$
   \end{theorem}
   \begin{proof}
   \begin{claim}\label{klasyfikacja skladowych}
    Spójna składowa $S\subseteq V(G)\backslash F$ grafu $G-F$ jest jednego z dwóch typów:
    \begin{itemize}
     \item główna -- $|S|>\frac{|V(G)|}{2}$
     \item mała -- $|S|\le\frac{|F|}{\varepsilon}$
    \end{itemize}
   \end{claim}
   
   \begin{proof} Stwierdzenia:\newline
    Weźmy spójną składową $S$ grafu $G-F$ ($N_{G-F}(S)=\emptyset$), jeżeli $S\le\frac{|V(G)|}{2}$, to z własności $\varepsilon$--ekspansji wierzchołkowej grafu $G$
    zachodzi $|N_G(S)|\ge\varepsilon\cdot|S|$ (gdzie $S$ jest teraz traktowane, jako podzbiór wierzchołków grafu $G$). Gdyby zachodziło $|S|>\frac{|F|}{\varepsilon}$,
    to mielibyśmy $|N_G(S)|>\frac{\varepsilon\cdot|F|}{\varepsilon}=|F|$, co daje sprzeczność, ponieważ aby w grafie $G-F$ otoczenie to było puste z grafu
    $G$ trzeba usunąć co najmniej $N_G(S)$ wierzchołków.
   \end{proof}
   
    Może być tylko jedna składowa główna.
    Co prawda dla dużego $|F|$ i małego $\varepsilon$ może być tak, że składowa jest jednocześnie główna i mała, jednak po pierwsze jest to przypadek mało
    interesujący, gdyż wtedy zwykłe przeszukiwanie grafu spełnia tezę twierdzenia, a po drugie przypadek ten nie psuje w żaden sposób otrzymywanego algorytmu.
    W stwierdzeniu istotne jest to, że w grafie nie ma składowych średnich wielkości.\newline
    
    Kontynuując dowód twierdzenia
    chcemy sprawdzić, czy wszyscy sąsiedzi wierzchołków usuniętych należą do tej samej spójnej składowej. Na podstawie lematu \ref{klasyfikacja skladowych},
    jeśli składowa zawierająca taki wierzchołek jest większa niż $\frac{|F|}{\varepsilon}$, to jest to składowa główna.
    Jeżeli wszystkie takie wierzchołki spełniają ten warunek, to $G-F$ jest spójny.
    Jeżeli natomiast, któraś z tych składowych okaże się mała, to $G-F$ nie jest spójny.\newline
    Wystarczy więc uruchomić liniowe przeszukiwanie grafowe w każdym wierzchołku sąsiadującym z wierzchołkiem wadliwym i przerywać po przejrzeniu
    $\frac{|F|}{\varepsilon}$ wierzchołków.\newline
    Algorytm liniowego przeszukiwania grafowego uruchamiany jest co najwyżej $|F|\cdot\Delta$ razy.
    Za każdym razem przeglądamy co najwyżej $\frac{|F|}{\varepsilon}$ wierzchołków.
    Dla każdego przeglądanego wierzchołka sprawdzamy co najwyżej $\Delta$ sąsiadów, a przeczytanie odpowiedzi wyroczni zajmuje $O(log(|V(G)|))$ czasu.
    Daje to złożoność z tezy twierdzenia.
   \end{proof}
   \begin{corollary}\label{ekspansywna spojnosc dla kostki}
    Zgodnie z twierdzeniem \ref{ekspansja kostki} hiperkostka $Q_n$ posiada własność
    $\frac{1}{\sqrt{\pi n}}$--ekspansji wierzchołkowej.
    Ponieważ można w niej znaleźć wszystkich sąsiadów wierzchołka w czasie liniowym od ich ilości powyższy algorytm testuje spójność wadliwej hiperkostki w czasie
    ${O(|F|^2\cdot n^{3.5})}$ (wyrażonego w ilości operacji arytmetycznych).
   \end{corollary}
    Ze względu na długość zapisu identyfikatora wierzchołka liniową od wymiaru hiperkostki nie da się przeprowadzać operacji na wierzchołkach w czasie szybszym niż
    $n$. To dolne ograniczenie jest osiągalne przy przechowywaniu przejrzanych wierzchołków w hashmapie
    (czas oczekiwany operacji $O(n)$, złożoność pamięciowa całej struktury $O(n^{0.5}|F|)$),
    lub w drzewie prefiksowym (czas pesymistyczny operacji $O(n)$, złożoność pamięciowa całej struktury $O(n^{1.5}|F|)$).
    Pozwala to w łatwy sposób uzyskać efektywną wyrocznię, a więc i algorytm o złożoności z wniosku.
   \subsection{pseudokod i uwagi}
    W algorytmie wykorzystywana jest struktura $T$ z operacjami
    \begin{itemize}[noitemsep,topsep=4pt]
     \item $Insert(v,T)$ wstawiającą wierzchołek $v$ do struktury $T$
     \item $Retrieve(v,T)$ zwracająca binarną informacje o obecności wierzchołka $v$ w strukturze $T$
     (w ogólniejszych rozważaniach również dowiązanie do tego wierzchołka)
    \end{itemize}
    które wymagają $O(n)$ czasu na wykonanie (aby uzyskać złożoność z wniosku \ref{ekspansywna spojnosc dla kostki}).\newline
    Przeszukiwanie grafowe odbywa się przy pomocy funkcji rekurencyjnej o pseudokodzie:\newline\newline
    \hspace*{100pt}$DFS(v)\{$\newline
    \hspace*{116pt}	$counter++;$\newline
    \hspace*{116pt}	$Insert(v,T);$\newline
    \hspace*{116pt}	if$(counter\ge size)\quad $return$(TRUE);$\newline
    \hspace*{116pt}	foreach$(u\in N(v))$\newline
    \hspace*{132pt}		if$(Retrieve(u,T)==FALSE)$\newline
    \hspace*{148pt}			if$(DFS(u))\quad $return$(TRUE);$\newline
    \hspace*{116pt}	return$(FALSE);$\newline
    \hspace*{100pt}$\}$\newline
    Spójność sprawdzana jest przy pomocy funkcji głównej o pseudokodzie:\newline\newline
    \hspace*{100pt}$Conectivity(n,F)\{$\newline
    \hspace*{116pt}	$T2=empty\_structureT();$\newline
    \hspace*{116pt}	$counter=0;$\newline
    \hspace*{116pt}	foreach$(f\in F)$\newline
    \hspace*{132pt}		$Insert(f,T2);$\newline
    \hspace*{132pt}		$counter++;$\newline
    \hspace*{116pt}	$size=sqrt(\pi*n)*counter;$\newline
    \hspace*{116pt}	foreach$(f\in F)$\newline
    \hspace*{132pt}		foreach$(v\in N(f))$\newline
    \hspace*{148pt}			if$(Retrieve(v,T2)==FALSE)$\newline
    \hspace*{164pt}				$counter=0;$\newline
    \hspace*{164pt}				$T=T2;$\newline
    \hspace*{164pt}				if$(DFS(v)==FALSE)\quad $return$(FALSE);$\newline
    \hspace*{116pt}	return$(TRUE);$\newline
    \hspace*{100pt}$\}$\newline
    
     Aby przeiterować po $N(v)$ wystarczy przeiterować się po współrzędnych uzyskując sąsiada poprzez zanegowanie tej współrzędnej w zapisie binarnym $v$.
     
     Można użyć dodatkowej struktury $T$, w której przechowywane są wszystkie wierzchołki z poprzednich wywołań $DFS(v)$ z funkcji głównej.
     Wtedy przy kolejnych użyciach $DFS(v)$ można sprawdzać, czy wierzchołek nie był już wcześniej w jakiejś składowej (można wtedy od razu zwrócić $TRUE$).
     Teoretycznie może to zwiększyć złożoność dwukrotnie, jednak w praktyce będzie to dużo szybsze
     (już nawet z tego względu, że albo inni sąsiedzi tego samego $f$ są oddaleni o 2, albo na drodze staje inny wierzchołek z $F$),
     w szczególności przy użyciu bardziej wyszukanych kolejności przeszukiwania (np. próba dojścia do wierzchołka $\overline{0}$).
     
  \section{Redukcja do lokalnej spójności przy pomocy transformacji ścieżek}\label{spojnosc 2}
   Algorytm przedstawiony w poprzednim podrozdziale jest dowodem na to, że testowanie spójności wadliwej hiperkostki może być zrobione wielomianowo
   ze względu na ilość wad i wymiar hiperkostki. Algorytm ten wykorzystuje jednak bardzo płytko potencjał tak regularnego grafu.
   W tym podrozdziale przedstawię algorytm, który dzięki głębszemu wykorzystaniu własności hiperkostki otrzymuje lepsze rezultaty złożonościowe.
   
   Używane w podrozdziale ścieżki w przeciwieństwie do reszty pracy mogą zawierać powtórzenia. 
   \begin{defi}\label{podgrafy kostki}
    Dla $F\subseteq V(Q_n)$ definiuję (za pracą \cite{DFGKR}) pomocniczy podgraf $Q_n$.\newline Niech $G(F)=(A\cup B\cup F,E)$,
    gdzie $A=N(F),\quad B=N(A)\backslash F,\newline E=\{uv\in E(Q_n):u\in A\cup F\}$ (podgraf zawierający wierzchołki w odległości $\le 2$ od wierzchołków wadliwych,
    plus krawędzie w których jeden z końców jest wadliwy lub z takim sąsiaduje).
   \end{defi}
   \begin{theorem}\label{spojnosc z lokalnej spojnosci}
    Dla $F\subseteq V(Q_n)$ graf $Q_n-F$ jest spójny wtedy i tylko wtedy, gdy dla każdej $C$ -- spójnej składowej $Q_n^2[F]$ spójny jest graf $G(C)-C$.
   \end{theorem}
   Na potrzeby dowodu twierdzenia zdefiniuję kilka obiektów i operacji.
  \subsection{Transformacje ścieżek w hiperkostce}
   \begin{defi}\label{sekwencja tranzycji}
     Dla ścieżki $W=(v_0,v_1,...,v_n)$ (z możliwymi powtórzeniami) w hiperkostce \emph{sekwencją tranzycji} nazywamy ciąg $\tau=(d_1,d_2,...,d_n)$,
     gdzie $d_i$ jest współrzędną, na której różnią się ciągi binarne wierzchołków $v_{i-1}$ i $v_i$.
    \end{defi}
    \begin{fact}\label{sekwencja tranzycji - parzystość}
     $\tau$ jest sekwencją tranzycji pewnej $uv$--ścieżki z powtórzeniami w $Q_n$ wtedy i tylko wtedy, gdy:\newline
     $u\Delta v=\{i\in[n]:\#(\tau,i)$ nieparzyste$\}$,
     gdzie $\#(\tau,i)$ to ilość wystąpień $i$ w sekwencji $\tau$.
    \end{fact}
    \noindent
    Dla $\tau$ -- sekwencji tranzycji $uv$--ścieżki $W$ definiujemy trzy operacje:
    \begin{itemize}
     \item $swap(\tau_1,i,j,\tau_2)=(\tau_1,j,i,\tau_2)$\quad dla $\tau=(\tau_1,i,j,\tau_2)$
     \item $insert_i(\tau_1,\tau_2)=(\tau_1,i,i,\tau_2)$\quad dla $\tau=(\tau_1,\tau_2),i\in[n]$
     \item $delete(\tau_1,i,i,\tau_2)=(\tau_1,\tau_2)$\quad dla $\tau=(\tau_1,i,i,\tau_2)$
    \end{itemize}
    Wszystkie te operacje nie zmieniają parzystości wystąpień współrzędnych, dlatego też dowolnie w ten sposób zmodyfikowana sekwencja
    wciąż jest sekwencją tranzycji pewnej $uv$--ścieżki w $Q_n$.
    \begin{defi}\label{sciezki rownowazne}
     Dwie ścieżki, których sekwencje tranzycji $\tau,\rho$ spełniają $\forall_{i\in[n]}\#(\tau,i)=\#(\rho,i)$ nazywamy \emph{równoważnymi}.
    \end{defi}
    \begin{fact}\label{przeksztalcanie sciezek}
     Dla dowolnych dwóch $uv$--ścieżek w $Q_n$ istnieje sekwencja operacji\newline
     $swap, insert, delete$ (w tej kolejności, bez przeplotów),
     która przemienia sekwencję tranzycji pierwszej w sekwencję tranzycji drugiej.
    \end{fact}
    \begin{proof}
     Jeśli dwie ścieżki są równoważne, to można jedną przekształcić w drugą przy pomocy samych operacji $swap$.\newline
     W przypadku, gdy sekwencje mają różne liczności wystąpień współrzędnych, to można je doprowadzić do takich $\tau',\rho'$,
     że $\forall_{i\in[n]}\#(\tau',i)=\#(\rho',i)$ przy pomocy samych operacji $insert$ (używanych w dowolnie wybranych wierzchołkach).
    \end{proof}
    Na potrzeby dowodu twierdzenia \ref{spojnosc z lokalnej spojnosci} wprowadzam definicję:
    \begin{defi}\label{port}
     Dla $uv$--ścieżki $W=(w_0,w_1,...,w_k)$ (gdzie $w_0=u,w_k=v$) wierzchołek
     $w_i$ nazywamy portem, jeśli nie jest wierzchołkiem wadliwym, ale dokładnie jeden z jego sąsiadów w ścieżce należy do $F$ (port musi więc należeć do $A$).
    \end{defi}
    W przypadku tej definicji należy rozróżnić przeplatające się pojęcia wierzchołka grafu i jego wystąpienia na ścieżce --
    portem nazywane jest konkretne wystąpienie na ścieżce, inne jego wystąpienia nie muszą być portami.\newline
    Dla $C$ spójnej składowej $G(F)-F$ przez $p(C,W)$ oznaczamy ilość portów w części $W$ nalezącej do $C$.
    \begin{lemma}\label{parzystosc portow swap}
     Operacja swap zachowuje parzystość $p(C,W)$.
    \end{lemma}
    \begin{proof}
     Dowód stanowi rysunkowe rozpatrzenie wszystkich możliwych przypadków, w których w wyniku operacji $swap$ powstaje i/lub znika pewien port
     (przypadki przy końcach ścieżki można "dopełnić" zwykłymi wierzchołkami do przypadków ze środka, ponieważ wierzchołki końcowe nie są wadliwe).
     \begin{multicols}{3}
      \begin{center}
       \includegraphics[scale=0.75]{img/Q_swap_l1.jpg}
       -- wierzchołek zwykły\newline
       \includegraphics[scale=0.75]{img/Q_swap_l4.jpg}
       -- krawędź ścieżki\newline
      \end{center}
      \begin{center}
       \includegraphics[scale=0.75]{img/Q_swap_l2.jpg}
       -- wierzchołek wadliwy\newline
       \includegraphics[scale=0.75]{img/Q_swap_l5.jpg}
       -- krawędź spoza ścieżki\newline
      \end{center}
      \begin{center}
       \includegraphics[scale=0.75]{img/Q_swap_l3.jpg}
       -- port\newline
       \includegraphics[scale=0.75]{img/Q_swap_l6.jpg}
       -- nieistotna reszta ścieżki\newline
      \end{center}
     \end{multicols}
     \noindent
     \hspace*{30pt}\includegraphics[scale=0.75]{img/Q_swap_1.jpg}\newline\newline
     \hspace*{30pt}\includegraphics[scale=0.75]{img/Q_swap_2.jpg}\newline\newline
     \hspace*{30pt}\includegraphics[scale=0.75]{img/Q_swap_3.jpg}\newline\newline
     \hspace*{30pt}\includegraphics[scale=0.75]{img/Q_swap_4.jpg}\newline\newline
     \hspace*{30pt}\includegraphics[scale=0.75]{img/Q_swap_5.jpg}\newline\newline
     \hspace*{30pt}\includegraphics[scale=0.75]{img/Q_swap_6.jpg}\newline\newline
     \hspace*{30pt}\includegraphics[scale=0.75]{img/Q_swap_7.jpg}\newline\newline
     Wszystkie rozrysowane tu wierzchołki należą do $G(F)$, a rozrysowane części niewadliwe tworzą podgraf spójny, dlatego też wszystkie te zmiany odbywają się w
     jednej spójnej składowej $G(F)-F$. Za każdym razem ilość portów zmienia się o $0$ lub $2$, a więc parzystość pozostaje bez zmian.
    \end{proof}\newpage
   \subsection{Dowód twierdzenia \ref{spojnosc z lokalnej spojnosci} o lokalnej spójności}
    \begin{lemma}\label{Q_n-F spojne => G(F)-F spojne (1 skladowa)}
     Niech $F\subseteq V(Q_n)$ będzie takie, że $G(F)$ jest spójne. Jeśli $Q_n-F$ jest spójne, to $G(F)-F$ również.
    \end{lemma}
    \begin{proof}
     Załóżmy przeciwnie -- istnieją wierzchołki $u,v\in A\cup B=V(G(F)-F)$, dla których istnieje ścieżka $P$ w $Q_n-F$, ale nie ma takiej w  $G(F)-F$.
     Skoro w $G(F)-F$ nie ma takiej ścieżki, to w $P$ musi występować wierzchołek $x$ spoza $A\cup B$.\newline
     $G(F)$ jest spójne, więc musi istnieć też druga ścieżka $R$ łącząca $u$ z $v$ w tym właśnie grafie, na której występuje wierzchołek $y\in F$.\newline
     Podobnie jak w dowodzie lematu \ref{przeksztalcanie sciezek} 
     ścieżki te mogą być napompowane w wierzchołkach $x$ i $y$ odpowiednio sekwencjami operacji $insert$.\newline
     Ponieważ $x$ jest oddalone od $F$ dodane do ścieżki $P$ wierzchołki nie uczynią z żadnego wystąpienia $x$ portu i same również nie staną się portami.
     Ponieważ $y$ należy do $F$ dodane do ścieżki $R$ wierzchołki będą miały dokładnie dwóch sąsiadów z $F$ (tego samego dwukrotnie), a więc nie będą portami.\newline
     Oznacza to, że uzyskane ścieżki równoważne mają tyle samo portów we wszystkich spójnych składowych $G(F)-F$ co odpowiadające im nieprzekształcone.
     Na podstawie lematów \ref{przeksztalcanie sciezek} i \ref{parzystosc portow swap} ich parzystości  się zgadzają (czyli zgadzają się również dla $P$ i $R$).
     Na ścieżce $P$ nie ma żadnych portów, ponieważ nie występuje w niej żaden wierzchołek wadliwy.
     Ścieżka $R$ w składowych $G(F)-F$, w których występują $u$ i $v$ ma nieparzyste ilości portów
     (np. można jako $R$ dobrać taką ścieżkę, która wchodzi do/opuszcza składowe co najwyżej raz), co daje sprzeczność.
    \end{proof}
    \begin{corollary}\label{Q_n-F spojne => Q_n^2[F]-F spojne (1 skladowa)}
     Dla $F\subseteq V(Q_n)$ takiego, że $Q_n^2[F]$ jest spójne ze spójności $Q_n-F$ wynika spójność $G(F)-F$.
    \end{corollary}
    \begin{proof}
     Jeśli dwa wierzchołki $Q_n^2[F]$ są połączone krawędzią, to w oryginalnym grafie musiały być w odległości $\le 2$, a więc w $G(F)$ muszą być połączone
     albo bezpośrednią krawędzią albo poprzez pojedynczy wierzchołek ze zbioru $A$, a więc graf $G(F)$ jest spójny, co sprawia,
     że spełnione są założenia lematu \ref{Q_n-F spojne => G(F)-F spojne (1 skladowa)}.
    \end{proof}
    \begin{lemma}\label{Q_n^2[F]-F spojne => Q_n-F spojne}
     Niech $F\subseteq V(Q_n)$ będzie taki, że $G(C)-C$ jest spójne dla każdej $C$ -- spójnej składowej $Q_n^2[F]$. Wtedy $Q_n-F$ również jest spójny.
    \end{lemma}
    \begin{proof}
     Dla dowolnie wybranych dwóch wierzchołków $u,v\in V(Q_n-F)$ weźmy $W$ -- ścieżkę między nimi w pełnym $Q_n$. Jeśli $W$ nie zawiera wadliwego wierzchołka,
     to jest poprawną ścieżką w $Q_n-F$. W przeciwnym przypadku znajdujemy na tej ścieżce pierwsze wystąpienie wierzchołka wadliwego.
     Poprzedni wierzchołek na ścieżce oraz pierwszy kolejny spoza zbioru $F$ są dwoma niewadliwymi wierzchołkami należącymi do $G(C)$, gdzie $C$
     jest spójną składową $Q_n^2[F]$ (oddalone o 1 od wadliwych wierzchołków, które są połączone ścieżką samych wadliwych wierzchołków).
     
     W $G(C)$ nie ma wadliwych wierzchołków spoza $C$, ponieważ oznaczałoby to, że taki wierzchołek jest oddalony o $\le 2$ od pewnego wierzchołka z $C$,
     a więc byłby z nim połączony w $Q_n^2$, dlatego też ścieżka ze spójnego z założenia $G(C)-C$ nie zawiera wadliwego wierzchołka.
     
     Wystarczy więc tę wadliwą część ścieżki $W$ zastąpić odpowiednią ścieżką z $G(C)-C$, aby zmniejszyć ilość wad na ścieżce.
     Po być może kilkukrotnym powtórzeniu tej operacji otrzymujemy poprawną ścieżkę w $Q_n-F$.
    \end{proof}\newpage
    \subsubsection{Właściwa część dowodu}
     Lemat \ref{Q_n^2[F]-F spojne => Q_n-F spojne} jest implikacją w jedną stronę.
     Implikacja w drugą stronę dla spójnego $Q_n^2[F]$ jest dana wnioskiem \ref{Q_n-F spojne => Q_n^2[F]-F spojne (1 skladowa)}.
     Wystarczy udowodnić, że nic nie psuje się w przypadku, gdy $Q_n^2[F]$ ma więcej niż jedną spójną składową.
     
     Dla $C$ -- spójnej składowej $Q_n^2[F]$ jeśli $Q_n-F$ jest spójne, to jest takie również $Q_n-C$
     (dla wierzchołków spoza $Q_n-F$ te same ścieżki są dobre, dla tych z  $F\backslash C$ dowolny sąsiad należy do $Q_n-F$, więc również łatwo zbudować ścieżkę),
     a więc spójne jest również $G(C)-C$, co kończy dowód.
   \subsection{Algorytm}
    Stosując powyższe twierdzenie można uzyskać wielomianowy algorytm używając jedynie przeszukiwania grafowego
    podobnie jak w podrozdziale \ref{podejscie ekspansywne}. Można jednak uzyskać lepsze rezultaty używając dodatkowo struktury $Find$--$Union$ i sprawdzając
    spójności już w trakcie budowania podgrafów $G(C)-C$.\newline
    W algorytmie używane są:
    \begin{itemize}[noitemsep,topsep=4pt]
     \item struktura $Find$--$Union$ $D$ z operacjami:
      \begin{itemize}[noitemsep,topsep=0pt]
       \item $Make(v,D)$ tworzącą singleton $\{v\}$
       \item $Find(v,D)$ zwracającą wskaźnik na zbiór zawierający $v$
       \item $Union(u,v,D)$ łączącą zbiór zawierający $u$ ze zbiorem zawierającym $v$
      \end{itemize}
      których zamortyzowany czas można ograniczyć przez $O(log m)$ (a da się nawet uzyskać $O(log^*m)$), gdzie $m$ jest ilością użyć
      operacji $Make(v,D)$.
      Dodatkowo struktura zapewnia możliwość sprawdzenia, czy zawiera więcej niż jeden zbiór (wystarczy pojedynczy licznik inkrementowany przy
      $Make(v,D)$ i dekrementowany przy $Union(u,v,D)$).
      \vspace*{4pt}
     \item strukturę $T$ do przechowywania informacji o niektórych wierzchołkach
     (zaimplementowaną przy pomocy binarnego drzewa prefiksowego lub tablicy hashującej),
      przechowywującą dla wierzchołka $v_T$ informacje:
      \begin{itemize}[noitemsep,topsep=0pt]
       \item wskaźnik do wierzchołka $v$ w strukturze $D$
       \item informacje o wadliwości/braku wadliwości wierzchołka
       \item binarną informacje o tym czy wierzchołek należy do $F\cup N(F)$ i był już odwiedzony
      \end{itemize}
      wspierającą operacje:
      \begin{itemize}[noitemsep,topsep=0pt]
       \item $Insert(v,T)$ wstawiającą wierzchołek $v$ do struktury $T$ i zwracającą wskaźnik na $v_T$
       \item $Retrieve(v,T)$ zwracającą $v_T$ lub $NULL$ w przypadku, gdy $v$ nie ma w strukturze
      \end{itemize}
      które wymagają czasu $O(n)$ na wykonanie.
    \end{itemize}
    
    Definiuję pomocniczą funkcję uzyskiwania wierzchołków ze struktury $T$ i inicjalizowania w razie nieobecności:\newline\newline
    \hspace*{100pt}$Retrieve1(v,T)\{$\newline
    \hspace*{116pt}	$v_T=Retrieve(v,T);$\newline
    \hspace*{116pt}	if$(v_T==NULL)$\newline
    \hspace*{132pt}		$v_T=Insert(v,T);$\newline
    \hspace*{132pt}		$v_T.healthy=TRUE;$\newline
    \hspace*{132pt}		$v_T.visited=FALSE;$\newline
    \hspace*{132pt}		$Make(v,D);$\newline
    \hspace*{116pt}	return$(v_T);$\newline
    \hspace*{100pt}$\}$\newline
    
    Najistotniejszą częścią algorytmu jest procedura $DFS(f)$ znajdująca spójne składowe $G(C)-C$,
    dla $C$ -- spójnej składowej $Q_n^2[F]$ zawierającej wadliwy wierzchołek $f$, o następującym pseudokodzie:\newline\newline    
    \hspace*{0pt}$DFS(f)\{$\newline
    \hspace*{16pt}	foreach$(u\in N(f))$\newline
    \hspace*{32pt}		$u_T=Retrieve1(u,T);$\newline
    \hspace*{32pt}		if$(u_T.visited==FALSE)$\newline
    \hspace*{48pt}			$u_T.visited=TRUE;$\newline
    \hspace*{48pt}			if$(u_T.healthy)$\newline
    \hspace*{64pt}				foreach$(v\in N(u))$\newline
    \hspace*{80pt}					$v_T=Retrieve1(v,T);$\newline
    \hspace*{80pt}					if$(v_T.healthy)\{$\newline
    \hspace*{96pt}						if$(Find(u,D)\neq Find(v,D))\quad Union(u,v,D);\quad\backslash\backslash uv\in E(G(C)-C)$\newline
    \hspace*{80pt}					else if$(v_T.visited==FALSE)$\newline
    \hspace*{96pt}						$v_T.visited=TRUE;$\newline
    \hspace*{96pt}						$DFS(v);\quad\backslash\backslash$ wadliwy wierzchołek należący do $C$ (oddalony od $f$ o 2)\newline
    \hspace*{48pt}			else$\quad DFS(u);\quad\backslash\backslash$ wadliwy wierzchołek należący do $C$ (oddalony od $f$ o 1)\newline
    \hspace*{0pt}$\}$\newline
    
    \noindent
    Powyższa prodedura uruchamiana jest z funkcji głównej:\newline\newline
    \hspace*{100pt}$Conectivity(n,F)\{$\newline
    \hspace*{116pt}	$T=empty\_structureT();$\newline
    \hspace*{116pt}	foreach$(f\in F)$\newline
    \hspace*{132pt}		$f_T=Insert(f,T);$\newline
    \hspace*{132pt}		$f_T.healthy=FALSE;$\newline
    \hspace*{132pt}		$f_T.visited=FALSE;$\newline
    \hspace*{116pt}	foreach$(f\in F)$\newline
    \hspace*{132pt}		$f_T=Retrieve(f,T);$\newline
    \hspace*{132pt}		if$(f_T.visited==FALSE)$\newline
    \hspace*{148pt}			$f_T.visited=TRUE;$\newline
    \hspace*{148pt}			$D=empty\_structureD();$\newline
    \hspace*{148pt}			$DFS(f);$\newline
    \hspace*{148pt}			if$(D.counter>1)\quad return(FALSE);$\newline
    \hspace*{116pt}	return$(TRUE);$\newline
    \hspace*{100pt}$\}$\newline
   \subsection{Analiza złożoności}
    \begin{corollary}\label{zlozonosc lokalnej spojnosci}
     Algorytm ma pesymistyczną złożoność czasową i pamięciową $O(|F|\cdot n^3)$.
    \end{corollary}
    \begin{proof}
     Dla każdego wierzchołka z $F$ każdy sąsiad jest przeglądany po jeden raz.
     Dla każdego wierzchołka należącego do $N(F)$ również przeglądani są wszyscy sąsiedzi
     po razie. Przeglądnięcie jednego wierzchołka (znalezienie odpowiedniego wierzchołka w $T$ i $D$) zajmuje $O(n)$,
     ustawienie właściwości w $D$ zajmuje stały czas po posiadaniu dowiązania do odpowiedniego wierzchołka -- daje to złożoność tej części $O(|F|\cdot n^3)$.\newline
     Operacja $Make(v,D)$ używana jest dla każdego wierzchołka z $G(C)-C$ po razie dla każdego $C$
     (może być użyta więcej niż raz dla wierzchołków oddalonych o 2 od $F$ i występujących w różnych $G(C)$).
     W przypadku $Find(v,D)$ i $Union(u,v,D)$ uruchamiane są one maksymalnie odpowiednio dwa i jeden raz dla każdego z sąsiadów wierzchołków $N(F)$
     -- daje to złożoność $O(|F|\cdot n^2\log(n))$.\newline
     Preprocessing i Postprocessing (tworzenie i usuwanie struktur $T$ i $D$)
     może być zrobione w czasie liniowym od ich wielkości (w przypadku $D$ i hashmapy można trzymać dodatkowo nieuporządkowaną listę dowiązań do wszystkich
     elementów). W przypadku struktury $T$ wielkość tą można ograniczyć przez $O(|F|\cdot n^3)$ przy użyciu drzewa prefiksowego
     (lub $O(|F|\cdot n^2)$ przy użyciu hashmapy, która nie pozwala jednak uzyskać odpowiedniej złożoności czasowej przy pesymistycznym scenariuszu),
     zaś w przypadku struktur $D$ łącznie $O(|F|\cdot n^2)$.
    \end{proof}
  \section{Wnioski i zastosowania}
   \subsection{Cykl Eulera}
    Mając dostępny efektywny algorytm badania spójności wysnułem następujący wniosek: 
 
    \begin{theorem}\label{cykl Eulera}
     Dla $F\subseteq V(Q_n)$ można rozstrzygnąć, czy w $Q_n-F$ jest cykl Eulera w czasie $O(|F|\cdot n^3)$
    \end{theorem}
    \begin{proof}
     Kryterium istnienia cyklu Eulera jest to, że po pierwsze graf jest spójny, a po drugie z każdego wierzchołka wychodzi parzyście wiele krawędzi.
     Spójność można sprawdzić w czasie $O(|F|\cdot n^3)$ przy pomocy algorytmu z poprzedniego podrozdziału.
     Wierzchołek niemający wadliwego sąsiada ma stopień $n$, wystarczy więc policzyć tylko parzystość dla tych, którzy takiego sąsiada mają.
     
     W czasie i pamięci $O(|F|\cdot n^2)$ można wstawić wszystkich niewadliwych sąsiadów wierzchołków wadliwych
     do drzewa prefiksowego zapamiętując w liściach krotność.
     Po wszystkim wystarczy dla $^2|n$ sprawdzić czy wszystkie wstawione wierzchołki mają krotność parzystą, zaś dla $^2\nmid n$ trzeba po pierwsze sprawdzić,
     że wszystkie wstawione wierzchołki mają krotność nieparzystą, a po drugie, że jest ich dokładnie $2^n-|F|$.
    \end{proof}
   \subsection{Istnienie ścieżki między dwoma punktami}
    Pomimo gorszej złożoności algorytmu ekspansywnego ma on wciąż przydatne zastosowania. Jeśli nie chcemy badać spójności całego grafu, a jedynie sprawdzić
    czy istnieje w nim ścieżka między dwoma wybranymi wierzchołkami algorytm ten łatwo zredukować:
    \begin{theorem}
     Dla $F\subseteq V(Q_n)$ i dwóch wierzchołków $u,v\in Q_n-F$ można w czasie $O(n^{2.5}|F|)$ rozstrzygnąć, czy w $Q_n-F$ istnieje $uv$--ścieżka.
    \end{theorem}
    \begin{proof}
     Chcemy sprawdzić czy dwa wierzchołki należą do jednej spójnej składowej grafu $Q_n-F$. Zgodnie z lematem \ref{klasyfikacja skladowych}
     jeśli spójne składowe, w których znajdują się te wierzchołki są większe niż $\frac{|F|}{\varepsilon}$, to są składową główną (a więc tą samą),
     wystarczy więc z obu wierzchołków wystartować przeszukiwanie grafowe i zakończyć je kiedy zbada się tyle wierzchołków.
     Jeśli oba wyszukiwania zakończą się dzięki temu kryterium, lub jeśli zostanie napotkany ten drugi wierzchołek, to w $Q_n-F$ istnieje $uv$--ścieżka,
     w przeciwnym przypadku nie.
     
     Tak przedstawiony algorytm działa dla dowolnego grafu (spójnego, z którego usuwamy wierzchołki)
     z odpowiednią wartością $\varepsilon$. W przypadku hiperkostki wystarczy przeszukać $2\sqrt{\pi n}\cdot|F|$ wierzchołków,
     plus krawędzi z nich wychodzące (każdy ma ich $n$),
     podczas gdy każda taka operacja kosztuje $O(n)$, co daje złożoność z twierdzenia.
    \end{proof}    
   \subsection{Długość ścieżki między dwoma punktami}
    Dzięki sprowadzeniu spójności do lokalnej spójności natychmiast zyskujemy konstrukcję ścieżki długości wielomianowej (od $n$ i $|F|$) pomiędzy dowolnymi
    wierzchołkami w spójnej hiperkostce z wadami (choć da się uzyskać również ścieżki o wiele krótsze).
    \begin{fact}
     Dla spójnego grafu $Q_n-F$ pomiędzy każdymi dwoma wierzchołkami jest ścieżka długości nie większej niż $n+n^2\cdot|F|$.
    \end{fact}
    \begin{proof}
     Jest to bezpośredni wniosek z dowodu lematu \ref{Q_n^2[F]-F spojne => Q_n-F spojne}. Dwa dowolne wierzchołki są połączone w pełnej hiperkostce $Q_n$
     ścieżką długości co najwyżej $n$. W dowodzie zastępujemy (być może kilka razy)
     część takiej ścieżki inną ścieżką w grafie $G(C)-C$. Łącznie długość wszystkich takich zastąpień nie może wynosić więcej niż wynosi rozmiar $G(F)-F$,
     co dowodzi tezy.
    \end{proof}

 \chapter{Długie ścieżki i cykle w grafie}
  W poprzednim rozdziale przedstawiony był przykład problemu na wadliwej hiperkostce, dla którego można było znaleźć rozwiązanie wielomianowe od $n$ i $|F|$.
  W tym rozdziale przedstawię kilka problemów, dla których już przedstawienie wyników wymagałoby wykładniczej pamięci,
  jednak samo rozstrzygnięcie czy rozwiązanie istnieje (sprawdzenie warunków twierdzenia) jest możliwe w czasie
  $O(|F|\cdot n)$ dla odpowiednio małych $|F|$ (wartości podane w sformułowaniach twierdzeń).
  W przypadku podwójnych ścieżek twierdzenie daje jedynie warunek wystarczający, dlatego algorytm otrzymany dzięki niemu nawet dla tych małych $|F|$ potrafi
  jedynie rozstrzygnąć pomiędzy "istnieją długie ścieżki" i "kryterium nie rozstrzyga".
  \section{Definicje}
   \begin{defi}\label{dluga sciezka}
    Wolną od wad (nieprzechodzącą przez wierzchołki wadliwe)
    ścieżkę bez powtórzeń (drogę) w hiperkostce $Q_n$ z wadami ze zbioru $F\subseteq V(Q_n)$ nazwiemy długą, jeśli ma długość co najmniej $2^n-2|F|-2$.
   \end{defi}
   \begin{defi}\label{dlugi cykl}
    Wolny od wad cykl bez powtórzeń w hiperkostce $Q_n$ z wadami ze zbioru $F\subseteq V(Q_n)$ nazwiemy długim, jeśli ma długość co najmniej $2^n-2|F|$.
   \end{defi}
   \begin{observation}\label{dluga sciezka- nie da sie dluzszej}
    Dla $F\cup\{u,v\}$ należącego do jednej dwudzielnej części $Q_n$ nie da się skonstruować $uv$--ścieżki wolnej od wad o długości większej niż $2^n-2|F|-2$
    (stąd długość w definicji).
   \end{observation}
   \begin{proof}
    Skoro $Q_n$ jest dwudzielna, to każda ścieżka musi odwiedzić prawie tyle samo wierzchołków w obu jej częściach (różnica nie większa niż $1$),
    ponieważ w części z $u$ i $v$ odwiedza co najwyżej $2^{n-1}-|F|$, to w drugiej co najwyżej $2^{n-1}-|F|-1$ -- daje to długość $2^n-2|F|-2$.
   \end{proof}
   \begin{defi}\label{wierzcholek otoczony}
    Wierzchołek $v\in V(Q_n)$ jest \emph{otoczony} przez $F\subseteq V(Q_n)$ gdy $N(v)\subseteq F$ ($F$ zawiera wszystkich sąsiadów $v$).
   \end{defi}
   \begin{defi}\label{para zablokowana}
    Dla $u,v\in V(Q_n), F\subseteq V(Q_n)$
    trójka $(u,v,F)$ jest \emph{zablokowana w $Q_n$} gdy $u$ jest otoczony przez $\{v\}\cup F$ lub $v$ jest otoczony przez $\{u\}\cup F$.
   \end{defi}
  \section{Długie ścieżki}
   \begin{theorem}\label{tw o dlugich sciezkach, male n}
    Niech $F\subseteq V(Q_n)$, $2\le n \le 5$ i $|F|\le 2n-4$ oraz $u,v\in V(Q_n)\backslash F$ takie, że trójka $(u,v,F)$
    nie jest zablokowana w $Q_n$. Długa $uv$--ścieżka bez wad nie istnieje tylko wtedy, gdy $n=4$ oraz istnieją takie $a,b\in V(Q_n)$,
    że $d(a,b)=4$  i $F\cup\{u,v,a,b\}$  jest dwudzielną częścią $Q_n$.
   \end{theorem}
   \begin{proof}
    Łatwo rozpatrzyć wszystkie przypadki.
    
    Na wyróżnienie zasługują przypadki negatywne, które z dokładnością do automorfizmów kostki są dokładnie 2.
    Rozpatruje je na obrazkach z zaznaczonymi wierzchołkami i uproszczoną ich wersją z pominiętymi wierzchołkami wadliwymi.\newline\newline
    \hspace*{100pt}\includegraphics[scale=0.7]{img/Q_4_niezablokowane_1.jpg}\newline
    \hspace*{100pt}\includegraphics[scale=0.7]{img/Q_4_niezablokowane_2.jpg}\newline
    Jak widać w obu przypadkach najdłuższa możliwa $uv$-ścieżka ma długość $4<16-2\cdot4-2$
   \end{proof}
   \begin{theorem}\label{tw o dlugich sciezkach}
    Dla $Q_n$ i $F\subseteq V(Q_n)$, takich  że  $n\ge 6$ i $|F|\le 2n-4$ dla każdych $u,v\in V(Q_n)\backslash F$ jeśli $(u,v,F)$ nie jest zablokowane w $Q_n$,
    to istnieje długa $uv$--ścieżka bez wad.
   \end{theorem}
   \begin{proof}
    (krótki szkic dowodu z pracy \cite{FG})\newline
    Dowód oparty jest na indukcji po wymiarze.
    Podstawę indukcji stanowi twierdzenie \ref{tw o dlugich sciezkach, male n}.
    Dla $n\ge6$, $|F|\le 2n-4$ można podzielić $Q_n$ na dwie kostki $Q_{n-1}$ wybierając jedną z $n$ współrzędnych 
    i definiując podkostki $Q^0_{n-1}$ i $Q^1_{n-1}$ jako podgrafy indukowane przez wierzchołki mające na tej współrzędnej odpowiednio $0$ i $1$.
    
    Dla $|F|\le 2n-5$ łatwo jest dobrać współrzędną tak, żeby każda z podkostek miała co najwyżej $2n-6=2(n-1)-4$ wadliwych wierzchołków
    (wystarczy wybrać dowolne $f_1,f_2\in F$ i podzielić według jednej ze współrzędnych różniących ich ciągi binarne).
    
    Dla $|F|=2n-4$ można rozpatrzyć macierz $|F|\times n$, w której w wierszach wypisane są ciągi binarne wszystkich wierzchołków wadliwych.
    Trzeba wybrać taką kolumnę, w której zarówno $0$ jak i $1$ jest co najmniej po 2. Gdyby nie dało się dokonać takiego wyboru oznaczałoby to, że
    w każdej kolumnie jest albo co najwyżej jedno $0$ albo co najwyżej jedna $1$, przez proste zanegowanie jednej współrzędnej w całej kostce
    (ta operacja nie zmienia nic poza numerowaniem) można uzyskać przypadek, że w każdej kolumnie jest co najwyżej jedna $1$. 
    Ponieważ kolumn jest tylko $n$, zaś każda zawiera co najwyżej jedną $1$, to oznaczałoby to, że macierz może mieć tylko $n+1$ różnych wierszy
    $\Rightarrow 2n-4=|F|\le n+1\Rightarrow n\le5$ (a więc ponieważ $n\ge 6$, to zawsze istnieje wybór współrzędnej).\newline
    Dowód kończymy przyużyciu lematów:
    \begin{itemize}
     \item Dla $|F|\le 2n-3$ co najwyżej jeden z wierzchołków jest otoczony przez $F$.
     \item Dla $|F|\le 2n-4$ i ustalonego nieotoczonego wierzchołka $u$ istnieje co najwyżej jeden wierzchołek $v$ taki, że $(u,v,F)$ jest zablokowana.
     \item Dla $|F|\le 2n-5$ (nieotaczającego żadnego wierzchołka) tylko jedna trójka $(u,v,F)$ może być zablokowana, i to taka, że $uv\in E(Q_n-F)$.
    \end{itemize}
    i wykorzystując fakt, że w podkostkach poza kilkoma przypadkami istnieją odpowiednie długie kostki rozważa się dużą liczbę przypadków
    (rozbicie ze względu na należenie $u$ i $v$ do tej samej/różnej podkostki, bycia otoczonym/zablokowanym/wolnym w podkostce).
    Dla każdego z tych przypadków da się pokazać metodę łączenia długich ścieżek z podkostek.
   \end{proof}   
   \includegraphics[scale=0.5]{img/sciezka_laczenie1.jpg}\quad\quad\quad\quad
   \includegraphics[scale=0.5]{img/sciezka_laczenie2.jpg}\newline
   (Rysunek ilustruje dwa najpopularniejsze przypadki występujące przy łączeniu ścieżek z podkostek.)\newline
   \begin{observation}\label{dluga sciezka 2n-3 za duzo}
    Dla $Q_n$, $F\subseteq V(Q_n)$, $|F|=2n-3$ teza twierdzenia \ref{tw o dlugich sciezkach} przestaje być prawdziwa.
   \end{observation}
   \begin{proof}
   Dla każdego $n\ge 4$ istnieje po kilka przypadków, w których $|F|=2n-3$, $(u,v,F)$ nie jest zablokowane, ale nie ma długiej $uv$--ścieżki bez wad.\newline
   3 przykłady :\newline
   \includegraphics{img/Q_niezablokowane_1.jpg}\quad\quad\quad
   \includegraphics{img/Q_niezablokowane_2.jpg}\quad\quad\quad
   \includegraphics{img/Q_niezablokowane_3.jpg}
   \end{proof}
   \vspace*{30pt}
  \section{Długie cykle}
   \begin{defi}\label{przekrój kostki}
    Dla zbioru $D\subseteq[n], d=|D|$ oraz $u\in\{0,1\}^{n-d}$ definiujemy kostkę $Q_D(u)$ jako $d$ wymiarową podkostkę $Q_n$, której współrzędne spoza $D$
    są ustalone przez wektor $u$. Definiujemy również $V_D(u)=\{(u,v)_D:v\in\{0,1\}^d\}$ (wierzchołki z oryginalnej kostki wzięte do $Q_D(u)$), oraz
    $F_D(u)=F\cap V_D(u)$.
   \end{defi}
   \vspace*{10pt}
   \begin{lemma}\label{dlugi cykl - podzial kostki}
    Niech $F\subseteq V(Q_n)$ takie, że $|F|\ge 2n$ i niech $d=\lceil\frac{n^2}{2|F|-n-2}\rceil$.
    Wtedy istnieje zbiór $D\subseteq[n],|D|=d$, taki że $|F_D(u)|\le d+1$ dla każdego $u\in\{0,1\}^{n-d}$.
   \end{lemma}
   \vspace*{10pt}
   Lemat pochodzi z pracy \cite{Wie} i został zmodyfikowany do tej postaci w pracy \cite{FG} aby lepiej pasować do dowodu poniższego twierdzenia.
   \vspace*{10pt}
   \begin{theorem}\label{dlugi cykl - tw}
    Dla $n\ge15$ i $F\subseteq V(Q_n)$, takiego że $|F|\le\frac{n^2}{10}+\frac{n}{2}+1$ istnieje długi cykl bez wad.
   \end{theorem}
   \begin{proof}
    (krótki szkic dowodu z pracy \cite{FG})\newline
    Na podstawie lematu \ref{dlugi cykl - podzial kostki} znajdujemy zbiór $D\subseteq[n]$, taki że $|F_D(u)|\le 2d-4$ dla każdego $u\in\{0,1\}^{n-d}$.
    Dla dowolnego cyklu Hamiltona $(u_0,u_1,...,u_{2^{n-d}}=u_0)$ w $Q_{n-d}$ dobieramy w kostce $Q_D(u_i)$ dwa niewadliwe
    wierzchołki $a_i$ oraz $b_i$, takie że $a_ib_{i+1}\in E(Q_n)$ dla każdego $i\in[2^{n-d}]$ (modulo $2^{n-d}$),
    oraz $(a_i,b_i,F_D(u^i))$ nie zablokowane (choć jest to nietrywialne to da sie takie dobrać).
    Na podstawie twierdzenia \ref{tw o dlugich sciezkach} wierzchołki $a_i$ i $b_i$ są łączone długimi ścieżkami dając cykl długości $\ge 2^n-2|F|$.
    Ograniczenie $|F|\le\frac{n^2}{10}+\frac{n}{2}+1$ potrzebne jest po to, aby $\lceil\frac{n^2}{2|F|-n-2}\rceil\ge5$ umożliwiając ominięcie złych
    przypadków z twierdzenia \ref{tw o dlugich sciezkach, male n}.
   \end{proof}
  \section{Długie pary ścieżek}
   \begin{lemma}\label{zawsze dlugie sciezki}
    Dla $n\ge2$, $F\subseteq V(Q_n)$, $|F|\le n-2$ dla każdych dwóch $u,v\in V(Q_n-F)$ istnieje długa $uv$--ścieżka bez wad.
   \end{lemma}
   \begin{proof}
    Bezpośrednio z \ref{tw o dlugich sciezkach}, gdzie ze względu na rozmiar $F$ trójka $(u,v,F)$ nie może być zablokowana.
   \end{proof}
   \begin{theorem}\label{pary sciezek}
    Dla $F\subseteq V(Q_n)$, $F\le n-3$ niech $A$ i $B$ będą różnymi dwuelementowymi podzbiorami $V(Q_n)-F$, takimi że $A\cup B$ nie należy do
    jednej części dwudzielnej hiperkostki. Wtedy istnieje para wierzchołkowo rozłącznych
    ścieżek o łącznej długości $\ge 2^n-2|F|-3$ zaczynających się w wierzchołku z $A$ i kończących na wierzchołku z $B$.
   \end{theorem}
   \begin{proof}
   (krótki szkic dowodu z pracy \cite{FG2})\newline
    Dowód podobnie jak inne przebiega indukcyjnie -- małe przypadki ($n\le 5$) można sprawdzić ręcznie (tutaj trochę więcej sprawdzania niż w poprzednich dowodach),
    dla większych łatwo jest podzielić $Q_n$ na dwie $Q_{n-1}$ tak, żeby każda z nich miała nie więcej niż $n-4$ wierzchołków wadliwych.
    Dalej rozpatrywane jest dużo przypadków w zależności od podziału wierzchołków z $A$ i $B$ na dwie podkostki i w każdym z możliwych przypadków łączy się
    podwójne i pojedyncze ścieżki istniejące na mocy indukcji i twierdzenia \ref{tw o dlugich sciezkach}.
   \end{proof}
    Jeśli $A=\{u,w\}, B=\{v,w\}$, to jedna ze ścieżek musi mieć długość $0$ i być zaczepiona w wierzchołku $w$.
    Twierdzenie \ref{pary sciezek} daje wtedy $uv$-ścieżkę
    wolną od wad długości $2^n-2|F|-3=2^n-2|F\cup\{w\}|-1$, a więc o jeden dłuższą niż w twierdzeniu \ref{tw o dlugich sciezkach}
    (możliwe jest to tylko dlatego, że $u$ i $v$ nie należą do jednej części dwudzielnej $Q_n$).
   
 \chapter{Ścieżka i cykl Hamiltona}\label{Ham}
  \section{Kostka bez wadliwych wierzchołków}

   Pełna kostka $Q_n$ ($n\ge2$) jest grafem hamiltonowskim (posiadającym cykl Hamiltona)
   -- bardzo wiele różnych cykli Hamiltona można uzyskać już łącząc dwa cykle
   z podkostek $Q_{n-1}$ (wystarczy, że w dwóch takich cyklach jest taka sama krawędź).
   \begin{center}
    \includegraphics[scale=0.5]{img/Q4_hamilton.jpg}
   \end{center}
   Cykle Hamiltona na hiperkostce mają również zastosowania praktyczne. Jednym z przykładów użycia takich cykli są kody Graya
   -- ciągi wszystkich wektorów binarnych określonej długości (po dokładnie raz), których sąsiednie wyrazy różnią się na dokładnie jednym bicie.
   Kody te znajdują zastosowanie między innymi przy efektywnej iteracji, minimalizacji kodowania i obwodów binarnych,
   oraz w algorytmach genetycznych i kodach korekcyjnych.\newline
   Pojawia się pytanie, czy po usunięciu kilku wierzchołków nadal da się zbudować takie cykle lub chociaż ścieżki Hamiltona.
  \section{Ogólne podejście do problemu}
   Sprawdzenie czy graf posiada cykl Hamiltona jest problemem NP--zupełnym, dlatego przy dowolnym wyborze zbioru wierzchołków usuniętych
   ciężko spodziewać się algorytmu w czasie $O(2^n)$, a co dopiero w $O(n^c)$. Jeśli jednak zbiór usuniętych wierzchołków
   jest odpowiednio mały lub tworzy graf z jakiejś wąskiej klasy, to istnieją dowody posiadania takiego cyklu przez wadliwą hiperkostkę.
   \begin{defi}\label{graf zbalansowany}
    Graf dwudzielny nazywamy \emph{zbalansowanym}, jeśli jego dwie części dwudzielne mają ten sam rozmiar,
    zaś \emph{prawie zbalansowanym} jeśli ich rozmiary różnią się o 1.
   \end{defi}    
   
   Hiperkostka jest grafem dwudzielnym, więc i po usunięciu części wierzchołków takim pozostanie.
   Jako że dowolny cykl w takim grafie odwiedza tyle samo wierzchołków w obu częściach dwudzielnych (z krotnościami),
   to podstawowym warunkiem koniecznym jest to, żeby z obu tych części usunąć dokładnie tyle samo wierzchołków.
   W przypadku ścieżki Hamiltona koniecznym jest żeby graf był zbalansowany lub prawie zbalansowany.
   
   \begin{defi}\label{Hamiltonian lacable}
    Graf dwudzielny (z częściami $U$ i $V$) nazywamy hamiltonowsko wiązalnym (Hamiltonian laceable), jeśli zachodzi jeden z warunków:
    \begin{itemize}
     \item jest zbalansowany i dla każdej pary $u\in U,v\in V$ istnieje ścieżka Hamiltona z $u$ do $v$.
     \item jest prawie zbalansowany i dla każdej pary $u,u'\in U$ (gdzie $U$ to ta większa składowa) istnieje ścieżka Hamiltona
      z $u$ do $u'$.
    \end{itemize}
   \end{defi}
   \noindent
   Jeśli graf jest zbalansowany i hamiltonowsko wiązalny, to istnieje w nim cykl Hamiltona (jeśli ma więcej niż 2 wierzchołki) -- wystarczy jako $u$ i $v$ wybrać
   wierzchołki połączone krawędzią.\newline
   Hiperkostka jest grafem hamiltonowsko wiązalnym.\newline

   W 2001 Stephen Locke postawił hipotezę:
   \begin{conjecture}\label{Locke conjecture}
    Niech $F\subseteq V(Q_n)$ będzie zbiorem po $k$ wierzchołków z obu dwudzielnych składowych $Q_n$. Wtedy $Q_n-F$ posiada cykl Hamiltona, jeśli
    $n\ge k+2$.
   \end{conjecture}
   Od tego czasu udowodnione zostały szczególne przypadki tej hipotezy ($k\le3$, $k\le\frac{n-5}{6}$,...),
   a w 2009 podane zostało ich uzupełnienie -- dowód indukcyjny po $n$ w pracy \cite{Locke}, który jednak po kilka przypadków granicznych odsyła
   do drugiej pracy, którą autorzy mają dopiero opublikować.
   
  \section{Szkice dowodów dla szczególnych przypadków}
   \begin{defi}\label{podgraf izometryczny}
    Podgraf H grafu G nazywamy \emph{izometrycznym}, jeśli $d_H(u,v)=d_G(u,v)$ dla każdych $u,v\in V(H)$.
   \end{defi}
   Łatwo zauważyć, że podgraf izometryczny jest zawsze podgrafem indukowanym przez podzbiór wierzchołków,
   implikacja w drugą stronę jednak nie zachodzi (na przykład $P_4$, jako podgraf $C_5$).
   \subsection{Podkostka}
    \begin{fact}\label{hamilton dla F=Q_m}
     Jeśli jako $F\subseteq V(Q_n)$, $n\ge3$ weźmiemy zbiór wierzchołków podkostki $Q_m$ ($1\le m<n$), to $Q_n-F$ posiada cykl Hamiltona.
    \end{fact}
    \begin{proof}
     Dla $m\ge2$:\newline 
     Pożądany cykl Hamiltona w $Q_n-F$ można otrzymać poprzez podział grafu na $2^{n-m}-1$ kostek $Q_m$, wybranie tego samego cyklu Hamiltona dla tych podkostek
     a następnie połączenie ich w następujący sposób:\newline
     Wybieramy jedną z podkostek, która wraz z usuniętą tworzy kostkę $Q_{m+1}$ -- taka kostka różni się od poprzedniej na dokładnie jednej współrzędnej
     (takiej która jest stała na całej podkostce). Łączymy obie i otrzymujemy w ten sposób wadliwą kostkę $Q_{m+1}$ z cyklem Hamiltona.
     Pozostałe kostki łączymy parami, razem z cyklami (wystarczy wybrać taką samą krawędź w obu i dokonać "przepięcia") -- wszystkie tak samo.
     Cykl z wadliwej hiperkostki ma tylko jedną krawędź, której nie ma w innych cyklach -- można dokonać połączenia wadliwej $Q_{m+1}$ z całą i dalej aż
     do otrzymania $Q_n-F$ z cyklem Hamiltona.\newline
     Dla $m=1$ postępujemy analogicznie, jednak po pierwszym kroku otrzymana $Q_2-Q_1$ nie ma cyklu a jedynie ścieżkę Hamiltona. $Q_3-Q_1$
     posiada już jednak cykl Hamiltona, który można przedłużyć do cyklu Hamiltona na $Q_3$ zawierający tę krawędź z $Q_2-Q_1$.
     Dalej można postępować już tak samo -- zawsze da się wybrać krawędź, według której można dokonać "przepięcia".
    \end{proof}
    Podobnym sposobem można otrzymać cykle dla $F$ równego sumie kilku kostek ($m\ge2$), które nie są połączone (tak długo, jak jest ich na tyle mało,
    aby dało się wybierać krawędzie do "przepięcia" -- to da się sprawdzić algorytmem wielomianowym od $n$ i $F$).\newline
   
   Poniżej przedstawię szkice dowodów twierdzeń dla innych szczególnych przypadków $F$ opisanych w pracy \cite{Pegr}.
   Ze względu na liczność mało istotnie różniących się przypadków po dokładne obrazki i opisy wyborów wierzchołków,
   które łączymy ścieżkami Hamiltona w podkostkach odsyłam do tejże pracy.
   \subsection{Ścieżka izometryczna}
    \begin{fact}\label{sciezka izo - rozne kierunki}
     W ścieżce izometrycznej w $Q_n$ każda z krawędzi ma inny kierunek, a więc ścieżka taka może mieć długość co najwyżej $n$.
    \end{fact}
    \begin{proof}
     Gdyby dwie krawędzie miały ten sam kierunek, to w $Q_n$ można by było połączyć początek i koniec ścieżki inną ścieżką, krótszą o 2
     (krawędzi o tych samych kierunkach, oprócz tych dwóch powtarzających się).
    \end{proof}
    \begin{theorem}\label{hamilton - sciezka, nieparz}
     Dla $F\subseteq V(Q_n)$ indukującego ścieżkę izometryczną nieparzystej długości $Q_n-F$ jest hamiltonowsko wiązalna, jeśli $n\ge4$.
    \end{theorem}
    \begin{proof}
     Dla ścieżek długości większej niż $1$ zakładamy indukcyjnie hipotezę dla mniejszych kostek.
     Dzielimy hiperkostkę na dwie względem kierunku drugiej krawędzi na ścieżce. W każdej podkostce znajduje się część ścieżki,
     która jest izometryczną nieparzystą ścieżką, więc obydwie wadliwe podkostki są zbalansowane.
     
     Problem rozbijamy na dwa przypadki -- w pierwszym wierzchołki $u$ i $v$, które chcemy połączyć ścieżką Hamiltona znajdują się w różnych podkostkach.
     Wystarczy wtedy wybrać parę wierzchołków odpowiadających sobie w podkostkach, o tych własnościach, że żaden nie jest wadliwy, są różne od $u$ i $v$, oraz
     są w innych dwuspójnych częściach niż $u$ lub $v$ z ich podkostki. Ze względu na ilość wierzchołków zawsze da się znaleźć taką parę,
     a z warunku indukcyjnego istnieją odpowiednie ścieżki Hamiltona w podkostkach.
     
     W przypadku, gdy $u$ i $v$ wylądują w jednej podkostce, można je w niej połączyć ścieżką Hamiltona i w odpowiednim miejscu
     (wzdłuż dowolnej krawędzi ścieżki Hamiltona, która istnieje w drugiej wadliwej podkostce) dopiąć ścieżkę Hamiltona z drugiej podkostki.
     \begin{center}
      \includegraphics[scale=0.75]{img/Q_hamilton_p1.jpg}\quad\quad\quad\quad
      \includegraphics[scale=0.75]{img/Q_hamilton_p2.jpg}
     \end{center}
     (Rysunek przedstawia oba przypadki występujące w dowodzie.)\newline
     Dla ścieżki długości $1$ dowód przebiega tak samo, z tym wyjątkiem, że w jednej podkostce ląduje cała ścieżka, zaś druga jest wolna od wad.
     Przypadek $n=4$ jest natomiast rozważany ręcznie
     (tam są możliwe tylko 2 takie ścieżki z dokładnością do automorfizmów kostki).
    \end{proof}
    \begin{theorem}\label{hamilton - sciezka, parz}
     Dla $F\subseteq V(Q_n)$ indukującego ścieżkę izometryczną parzystej długości $Q_n-F$ jest hamiltonowsko wiązalna, jeśli $n\ge5$.
    \end{theorem}
    \begin{proof}
     Ponownie indukcyjnie po $n$, tym razem dzieląc kostkę w kierunku pierwszej krawędzi ścieżki -- zostawiając w pierwszej podkostce pojedynczy wadliwy wierzchołek,
     zaś w drugiej kostce wymiaru $\ge4$ wadliwą ścieżkę nieparzystej długości (przypadek z poprzedniego twierdzenia).
     Tym razem należy rozpatrzyć 3 przypadki (gdy $u$ i $u'$ są w różnych kostkach i 2 przypadki, gdy są w tej samej).
     Dla każdego z tych przypadków ponownie łatwo wybrać odpowiednie ścieżki w podkostkach i połączyć je tak, aby uzyskać ścieżki Hamiltona w
     $Q_n-F$ od $u$ do $u'$ (szczegóły z obrazkami można zobaczyć w pracy \cite{Pegr}).
    \end{proof}
     Na podstawie \cite{SCJY} prawdziwe jest mocniejsze stwierdzenie:\newline
     Niech $P$ będzie dowolną ścieżką długości $\le 2n-4$ w grafie $Q_n$, wtedy wadliwa hiperkostka  $Q_n-P$ jest hamiltonowsko wiązalna.
   \subsection{Cykl izometryczny}
    \begin{fact}\label{cykl izometryczny - drugie pol powtorka}
     W cyklu izometrycznym na hiperkostce każdy kierunek krawędzi występuje dokładnie 2 razy, w antypodycznych krawędziach cyklu.
    \end{fact}
    \begin{proof}
     Każdy kierunek krawędzi musi się pojawić na dowolnym cyklu w hiperkostce parzystą ilość razy. Dla cyklu długości $2k$
     weźmy dwa wystąpienia tego samego kierunku, oddalone od siebie na cyklu o minimalną liczbę krawędzi. Dostajemy ścieżkę długości $l$ zaczynającą
     i kończącą się krawędzią tego samego kierunku. Podobnie jak w dowodzie faktu \ref{sciezka izo - rozne kierunki} mamy w hiperkostce
     inną ścieżkę długości $l-2$ łączącą te same końce. Z izometryczności $l-2\ge min(l,2k-l)$, zaś z minimalności $l-2\le 2k-l\Rightarrow l-1=k$.
    \end{proof}
    \begin{theorem}\label{hamilton - cykl, parz}
     Niech $C$ będzie izometrycznym cyklem w $Q_n$ ($n\ge6$), o długości $2k$ podzielnej przez $4$, wtedy graf $Q_n-P$ jest hamiltonowsko wiązalny.
    \end{theorem}
    \begin{proof}
     Zaczynamy od podzielenia hiperkostki na dwie w kierunku wyznaczonym przez pierwszą (dowolnie wybraną) krawędź cyklu.
     W obu kostkach znajduje się wadliwa izometryczna ścieżka długości nieparzystej, a więc z twierdzenia \ref{hamilton - sciezka, nieparz}
     obie są hamiltonowsko wiązalne. Podobnie jak wcześniej wystarczy rozważyć dwa przypadki w zależności od tego czy
     $u$ i $v$ wpadają do różnych, czy do tej samej podkostki. W obu przypadkach parę wierzchołków do połączenia ścieżek, czy krawędź do podpięcia
     cyklu wybieramy analogicznie jak w dowodzie twierdzenia \ref{hamilton - sciezka, nieparz}.
    \end{proof}     
    \begin{center}
     \includegraphics[scale=0.75]{img/Q_hamilton_c1.jpg}\quad\quad\quad\quad
     \includegraphics[scale=0.75]{img/Q_hamilton_c2.jpg}
    \end{center}
    (Rysunek przedstawia oba przypadki występujące w dowodzie.)\newline
   \subsection{Drzewo izometryczne}
    \begin{fact}\label{drzewo izo - kierunek tylko raz}
     W drzewie izometrycznym w $Q_n$ każda z krawędzi ma inny kierunek.
    \end{fact}
    \begin{proof}
     Załóżmy, że istnieją dwie krawędzie w tym samym kierunku. Wynika z tego, że w drzewie jest ścieżka,
     która zawiera obie krawędzie, co daje sprzeczność z lematem
     \ref{sciezka izo - rozne kierunki} (w drzewie każde dwa wierzchołki są połączone tylko jedną ścieżką, więc nie ma innej krótszej).
    \end{proof}
    \begin{observation}\label{drzewo izo zbal - ma liść z V}
     Drzewo jest grafem dwudzielnym o częściach $U$ i $V$. Jeśli niepuste drzewo jest zbalansowane, lub prawie zbalansowane z przewagą w $V$,
     to do $V$ należy pewien jego liść.
    \end{observation}
    \begin{proof}
     Nie wprost -- jeśli w drzewie każdy liść należy do dwudzielnej części $U$, to można wybrać dowolny wierzchołek z $v\in V$ jako korzeń.
     Usunięcie dowolnej gałęzi bez rozwidleń (od ostatniego rozwidlenia do liścia) nie produkuje nowych liści, może też pogorszyć balans jedynie
     zwiększając przewagę $V$. Gdy zostanie już tylko jedna ścieżka, to kończyć się ona będzie z obu stron wierzchołkami z $U$, co daje sprzeczność.
    \end{proof}
    \begin{lemma}\label{Q_5-izo tree hamilton}
     Dla zbalansowanego drzewa $T$ izometrycznego w $Q_5$ graf $Q_5-T$ jest hamiltonowsko wiązalny.
    \end{lemma}
    Lemat można sprawdzić ręcznie lub komputerowo (są tylko dwa takie drzewa niebędące ścieżkami z dokładnością do umiejscowienia w kostce).
     \begin{center}
      \includegraphics[scale=0.5]{img/Q_hamilton_t1.jpg}\quad\quad\quad\quad
      \includegraphics[scale=0.5]{img/Q_hamilton_t2.jpg}
     \end{center}
    \begin{theorem}\label{Hamilton - drzewo izo}
     Niech $T$ będzie zbalansowanym lub prawie zbalansowanym drzewem izometrycznym w $Q_n$. Graf $Q_n-T$ jest hamiltonowsko wiązalny dla $n\ge5$
     (w przypadku zbalansowanego wystarczy $n\ge4$).
    \end{theorem}
    \begin{proof}
     Dowód przez indukcję po rozmiarze hiperkostki. Jako krok bazowy można traktować lemat \ref{Q_5-izo tree hamilton}, przypadek drzewa prawie zbalansowanego
     (w $Q_5$ istnieje takie tylko jedno niebędące ścieżką) i zbalansowanego dla $n=4$ (tu są tylko ścieżki) można rozpatrzyć analogicznie do tamtego.
     W kroku indukcyjnym dla drzewa zbalansowanego korzystamy z wymiaru o $1$ mniejszego i drzewa prawie zbalansowanego 
     zaś dla drzewa prawie zbalansowanego korzystamy z drzewa zbalansowanego w kostce o mniejszym wymiarze.
     Na podstawie obserwacji \ref{drzewo izo zbal - ma liść z V} wybieramy liść z części dwudzielnej, w której drzewo
     ma co najmniej połowę wierzchołków i dzielimy mniejsze według kierunku krawędzi drzewa wchodzącej do tego wierzchołka.
     Otrzymujemy w ten sposób w jednej podkostce pojedynczy wadliwy wierzchołek (jak w dowodach poprzednich twierdzeń)
     oraz w drugiej izometryczne drzewo zbalansowane lub prawie zbalansowane.
     
     Jak we wcześniejszych dowodach rozpatrujemy przypadki w zależności od tego, w których podkostkach znajdują się wierzchołki
     które chcemy połączyć ścieżką Hamiltona, tworzymy odpowiednie ścieżki w podkostkach i łączymy je.
    \end{proof}

  \chapter{Maksymalna ścieżka indukowana}
   \begin{defi}\label{srednica grafu}
    \emph{Średnicą grafu} nazywamy największą odległość pomiędzy jego dwoma wierzchołkami $d(G)=\max\limits_{u,v\in V(G)}d(u,v)$.
   \end{defi}
   Jedną z najważniejszych zalet hiperkostki $Q_n$, jest to, że pomimo względnie niskiej gęstości jej średnica jest niewielka i równa $n$,
   a do tego po usunięciu części węzłów często pozostaje spójna.
   Graf ten posiada również tę przyjemną własność, że przy losowym usuwaniu wierzchołków odległości pomiędzy nieusuniętymi wierzchołkami rzadko się zwiększają
   (w szczególności dla wierzchołków bardziej oddalonych).
   Przy intencjonalnym usuwaniu, aby zwiększyć odległość pomiędzy dowolnymi dwoma wierzchołkami potrzeba usunąć co najmniej dwa wierzchołki
   (w przypadku dowolnych wierzchołków oddalonych o 2), zaś aby zwiększyć średnicę grafu należy usunąć co najmniej $n-1$ wierzchołków
   (wszystkich prócz jednego sąsiadów dowolnie wybranego wierzchołka).
   Aby uzyskać średnicę $n+2$ należy usunąć $2n-3$ wierzchołków
   (na przykład sąsiedzi wierzchołków numer $1$ i $2$ oprócz $4$ i siebie nawzajem przy numerowaniu klasycznym),
   dalej jednak pojawia się coraz więcej możliwości usuwania wierzchołków i ciężko wybrać najlepszą.
    
   Zainspirowany tym postawiłem sobie pytanie, jaką największą średnicę można uzyskać w ten sposób pozostawiając graf spójnym.
   Średnicę spójnego grafu $Q_n-F$ wyznacza najkrótsza ścieżka między pewnymi dwoma wierzchołkami, jeśli więc dla jakiegoś grafu uda się uzyskać
   taką maksymalną średnicę, to można wybrać tę ścieżkę i odrzucić wszystkie wierzchołki spoza niej (nie zmniejszając średnicy).
   Oznacza, to że maksimum po wszystkich grafach $Q_n-F$ jest uzyskiwane dla grafu będącego ścieżką indukowaną.
   
   Ścieżka taka ma tę własność, że ciągi binarne sąsiadujących w niej wierzchołków różnią się na dokładnie jednej pozycji,
   natomiast dla dowolnych dwóch niebędących sąsiadami ciągi te muszą się różnić na co najmniej dwóch pozycjach.
   Wyjściowy problem jest więc również równoważny problemowi:\newline
   Znajdź najdłuższy ciąg wektorów binarnych długości $n$, taki
   że każde dwa sąsiednie wektory różnią się na co najmniej jednej pozycji, zaś każde dwa inne na co najmniej dwóch.
   \section{Ścieżka Fibonacciego}
    W tym podrozdziale przedstawię własną, bardzo prostą w użyciu i wygodną w implementacji
    (w porównaniu z konstrukcjami spotykanymi w innych pracach) konstrukcję wykładniczo długiej ścieżki indukowanej.\newpage
    \begin{theorem}\label{co najmniej fibo}
     Dla danego $n$ istnieje algorytm konstruujący ciąg wektorów reprezentujący ścieżkę długości $F_{n+1}-1$,
     (gdzie $F_n$ to $n$-ty wyraz ciągu Fibonacciego: $F_0=F_1=1$), która jest podgrafem indukowanym hiperkostki.
     Co więcej prefiks długości $F_{n}$ jest takim ciągiem dla $n-1$ (po odrzuceniu ostatniej współrzędnej wektorów).
    \end{theorem}
    \begin{proof}
     Na potrzeby dowodu nazwę taką ścieżkę $S_n$.
     Dowód tradycyjnie przez indukcję po wymiarze -- dla $n\in\{0,1,2,3,4\}$ konstrukcja jest widoczna na rysunkach:\newline
     \hspace*{0pt}\includegraphics[scale=0.6]{img/Q0_long_path.jpg}
     \hspace*{12pt}\includegraphics[scale=0.6]{img/Q1_long_path.jpg}
     \hspace*{12pt}\includegraphics[scale=0.6]{img/Q2_long_path.jpg}
     \hspace*{12pt}\includegraphics[scale=0.6]{img/Q3_long_path.jpg}
     \hspace*{12pt}\includegraphics[scale=0.6]{img/Q4_long_path.jpg}\newline
     Dla tak małych wymiarów jest to jedyny z dokładnością do automorfizmów hiperkostki sposób uzyskania ścieżki tej długości.
     Już na tych przypadkach widać, że ścieżka tak powstała wygląda identycznie od początku i od końca
     (po zamianie punktów początkowego z końcowym oraz przepermutowaniu współrzędnych). Dodatkowo widać też,
     że tylko 3 podkostki $Q_{n-2}$ zawierają wybrane do ścieżki wierzchołki, dokładniej pierwsza podkostka zawiera $S_{n-2}$, druga $S_{n-3}$
     i trzecia znów $S_{n-2}$ (znów z dokładnością do automorfizmów).\newline
     Konstrukcja indukcyjna $S_n$ przebiega następująco:\newline
     Startuję od $S_{n-1}$, który używa tylko współrzędne od $0$ do $n-2$, kolejnym wierzchołkiem w ciągu będzie sąsiad aktualnie ostatniego w kierunku
     $n-1$. Mam dane już ponad pół ścieżki, jeśli więc chcę uzyskać $S_n$ takie samo od początku jak i od końca wystarczy znaleźć odpowiednią permutację.
     Co do tej nowej części mamy pewną dowolność przy wyborze permutacji, jednak musi się ona zgadzać na tej środkowej części
     ($S_{n-2}$ zaczynające się po krawędzi w kierunku $n-3$, a kończące przed dostawioną $n-2$ -- te współrzędne zostaną ze sobą zamienione w permutacji
     i w całej ścieżce $S_n$ krawędzie w tych kierunkach występują tylko po razie). Z kroku indukcyjnego $S_{n-2}$ jest symetryczne w ten sposób,
     więc istnieje odpowiednia permutacja na kierunkach w niej używanych, wystarczy więc ją rozszerzyć o tą zamianę $n-1$ z $n-2$
     (permutacja jest już w pełni zadana, bo to $S_{n-2}$ używa wszystkie pierwsze $n-2$ kierunki).
     Z konstrukcji wynika, że zbudowana tak ścieżka posiada własność: dowolne dwa wierzchołki ze ścieżki są sąsiadami w hiperkostce
     tylko wtedy, gdy są sąsiadami na ścieżce wewnątrz trzech swoich części. Pomiędzy pierwszą i drugą częścią własność ta jest spełniona z kroku indukcyjnego,
     zaś pomiędzy trzecią i drugą jest tak samo z symetrii. Pozostają części pierwsza i trzecia, ale skoro są one całkowicie zanurzone w dwóch podkostkach,
     które nie mają między sobą krawędzi, to warunek ten jest również spełniony w tym przypadku.
    \end{proof}
     Ciąg Fibonacciego rośnie w tempie wykładniczym, oznacza to, że ilość użytych wierzchołków w tak skonstruowanej wynosi
     $\frac{1}{\sqrt{5}}((\frac{1+\sqrt{5}}{2})^{n+1}-(\frac{1-\sqrt{5}}{2})^{n+1})\approx \frac{(1.62)^{n+1}}{\sqrt{5}}=\Omega(1.61^n)$, przy wielkości kostki $2^n$.
   \section{Wyniki eksperymentalne}
    \begin{observation}\label{da sie dluzsze}
     Dla wymiarów większych niż $4$ da się otrzymać ścieżki jeszcze dłuższe -- dla $n=5$ najdłuższą możliwą ścieżką jest:\newline
     00000
     00001
     00011
     00111
     01111
     01110
     01100
     11100
     11101
     11001
     11011
     11010
     10010
     10110.\newline
     Ma ona długość 13, jest więc o 1 dłuższa od ścieżki otrzymanej na podstawie twierdzenia \ref{co najmniej fibo}
    \end{observation}
    Aby znaleźć możliwie najdłuższe ścieżki dla większych $n$ przeprowadziłem eksperymenty przy pomocy komputera.
    Wyszukiwanie ścieżek prowadziłem startując od wierzchołka $\overline{0}$ i dobierając do istniejącej ścieżki nowy wierzchołek,
    który jest sąsiadem aktualnego końca i nie jest sąsiadem żadnego poprzedniego, aż do momentu, gdy taki wybór nie jest możliwy 
    -- w ten sposób otrzymuje "grę". Znalezienie wszystkich możliwych ścieżek otrzymywalnych w ten sposób
    (a są to wszystkie takie poprawne ścieżki, które zaczynają się w wierzchołku $\overline{0}$) otrzymywane jest poprzez przejrzenie
    drzewa gry, której wierzchołkami są ścieżki. Dla dowolnego wierzchołka drzewa ścieżki w przodkach tego wierzchołka są prefiksami ścieżki
    przez niego przechowywanej. Drzewo tej gry ma maksymalny stopień rozgałęzienia równy $n$, zaś głębokość $|P|$.
    Daje to złożoność algorytmu przeglądającego wszystkie ścieżki mocno pesymistycznie ograniczoną przez $O(n^{|P|})$. Nawet po zastosowaniu
    różnych usprawnień, algorytm dokładny szybko staje się zbyt wolny, dlatego dla $n\ge7$ musiałem skorzystać z metod losowych.
    \vspace*{-20pt}
    \subsection{Nested Monte Carlo Search (NMCS)}
     Aby otrzymać pewną ścieżkę, która posiada tę własność, że nie da się jej już wydłużyć do innej poprawnej można przeprowadzić zejście po drzewie
     od korzenia do liścia poprzez losowe wybieranie jednego z dzieci za każdym razem, gdy taki wybór jest możliwy.\newline
     Przeprowadzając odpowiednio dużo takich losowych zejść po drzewie i wybierając najdłuższą ścieżkę ze znalezionych możemy uzyskać już w miarę dobry
     wynik, istnieją jednak metody, które pozwalają na lepsze ukierunkowanie losowości.\newline
     Jedną z takich metod jest przeszukiwanie Monte Carlo polegające na tym, że mając dany stan gry (aktualną ścieżkę) dla każdego możliwego ruchu
     (wyboru kolejnego wierzchołka) szacujemy jego wartość poprzez uruchomienie przeszukiwania losowego w tym kierunku i następnie wybranie najlepszego z nich.
     Metoda ta jest wykorzystywana głównie w przypadkach gier, w tym takich w których gracz ma tylko częściowy wpływ na rozgrywkę (istnieje przeciwnik),
     jednak w przypadku szukania możliwie długiej ścieżki również się sprawdza.
     
     Rozszerzenie tej metody zostało zaprezentowane w pracy \cite{NMCS} zwane Nested Monte Carlo Search, polegające na tym, że do oszacowania
     wartości możliwych ruchów zamiast zwykłych przeszukiwań losowych używamy zwykłego przeszukiwania Monte Carlo.
     Dokładniej metoda ta zakłada, że tworząc rozwiązanie przy pomocy NMCS z poziomem $k$ używamy do wyboru ruchu NMCS z poziomem $k-1$ i tak dalej rekurencyjnie
     aż do poziomu $0$, który oznacza przeszukiwanie losowe. Można to przedstawić następującym pseudokodem:\newline\newline
     \hspace*{100pt}$NMCS(v,level)\{$\newline
     \hspace*{116pt}	while$(TRUE)$\newline
     \hspace*{132pt}		$val.push\_back(v);$\newline
     \hspace*{132pt}		if$(|v.children|==0)$ return $val;$\newline
     \hspace*{132pt}		if$(level==0)$\newline
     \hspace*{148pt}			$v=random(v.children);$\newline
     \hspace*{132pt}		else\newline
     \hspace*{148pt}			$best=-1;$\newline    
     \hspace*{148pt}			foreach$(u\in v.children)$\newline
     \hspace*{164pt}				$val2=NMCS(u,level-1);$\newline
     \hspace*{164pt}				if$(value(val2)>best)$\newline
     \hspace*{180pt}					$best=value(val2);$\newline
     \hspace*{180pt}					$move=u;$\newline
     \hspace*{148pt}			$v=move;$\newline
     \hspace*{100pt}$\}$
    \subsection{Uwagi praktyczne}
     Ze względu na ogromną różnicę w złożoności algorytmu $NMCS$ w zależności od użytego poziomu można wprowadzić lekką modyfikację -- w przypadku,
     gdy wyższy poziom jest zbyt kosztowny, zaś niższy za słaby zamiast uruchamiać program z niskim poziomem wiele razy można poszerzyć przeszukiwanie.
     Poszerzenie takie można łatwo uzyskać uruchamiając dla każdego dziecka symulację poziomu niżej nie raz a $T$ razy
     (plus uruchomienie algorytmu najwyższego poziomu tyle razy).
     
     Mając na uwadze algorytm z tą modyfikacją można oszacować pesymistyczną złożoność algorytmu dla szukania długiej indukowanej ścieżki w hiperkostce
     w zależności od wymiaru $n$, poziomu $L$, liczby razy przy sprawdzaniu dzieci $T$ oraz oczekiwanej długości maksymalnej ścieżki $|P|$.
     Złożoność ta jest wyrażona poprzez równanie rekurencyjne:\newline
     $NMCS(n,0,T,|P|)=n\cdot T\cdot |P|$\newline
     $NMCS(n,L,T,|P|)=n\cdot T\cdot NMCS(n,L-1,T,|P|-1)+NMCS(n,L,T,|P|-1)$\newline
     Które daje rozwiązanie:\newline
     $NMCS(n,L,T,|P|)=O(T^{L+1}\cdot n^{L+1}\cdot|P|^{L+1})$     
     
     W przypadku szukania indukowanej ścieżki w hiperkostce należy jeszcze doliczyć czas i pamięć potrzebne na sprawdzanie czy nowo wybrany (wylosowany)
     wierzchołek nie jest sąsiadem jednego z wierzchołków z początku ścieżki.
     
     Jedną z możliwych metod jest za każdym razem przeiterowanie po aktualnej ścieżce i porównanie ciągów binarnych, daje to jednak złożoność
     czasową rzędu $n\cdot |P|$. Drugą metodą jest podobnie jak w algorytmach z poprzednich rozdziałów użycie jako struktury przechowującej wszystkich sąsiadów
     wierzchołków ze ścieżki drzewa prefiksowego lub tablicy hashującej. W ten sposób otrzymujemy czas $O(n)$ na odczytanie i wstawienie wierzchołka,
     potrzebując $O(\min(n^{1.5}\cdot|P|,2^n))$ lub $O(\min(n\cdot|P|,2^n))$ pamięci w zależności od użytej struktury.
     Możliwe jest też o wiele prostsze rozwiązanie -- przechowywanie tych sąsiadów w tablicy binarnej wielkości $2^n$.
     Rozwiązanie to o ile nie lepsze asymptotycznie (ten sam czas i niemniejsza pamięć),
     ze względu na wnioskowaną z testów dużą długość $P$ (porównywalna z $\frac{2^n}{n}$) nie jest też gorsze,
     a łatwiejsze do zoptymalizowania i uzyskania istotnie mniejszych stałych.
     
     W przypadku użycia struktury zapamiętującej sąsiadów wierzchołków ze ścieżki należy jeszcze zwrócić uwagę na zmienianie jej stanu przy wchodzeniu
     i wracaniu z niższych poziomów wywołania. Można albo za każdym razem pamiętać wszystkie wstawione wierzchołki i przy zwracaniu wyniku
     do poziomu wyżej dokonywać ich usunięcia, albo można za każdym razem kopiować całą strukturę (dosyć efektywne przy użyciu tablic).
     W drugim przypadku wystarczy mieć $L+1$ kopii struktury na raz (po jednej na każdy poziom) i przy powrocie z poziomu niżej
     zastępować starą wersję z tamtego poziomu kopią z poziomu o 1 wyżej.
     
     Mając to na uwadze można zaprogramować algorytm tak, aby miał złożoności czasową $O(T^{L+1}\cdot n^{L+2}\cdot|P|^{L+1})$
     i pamięciową $O(L\cdot \min(n\cdot |P|,2^n))$.\newline
     
     Przeszukiwanie można dodatkowo zawęzić jeśli program weźmie pod uwagę, że kierunki, które nie pojawiły się jeszcze na ścieżce są równoważne
     (wystarczy pamiętać liczbę oznaczającą ilość kierunków już użytych i nie używać kierunków większych niż ta liczba + 1).
     W ten sposób ustalone są pierwsze 3 ruchy ($1\rightarrow2\rightarrow4\rightarrow8$ przy numerowaniu klasycznym),
     zaś na czwarty są tylko dwie możliwości ($8\rightarrow7,8\rightarrow17$). Daje to duże przyśpieszenie przy małych przypadkach,
     jednak przy większych traci na znaczeniu.
    \subsection{Uzyskane wartości}
     Przy pomocy komputera i przedstawionych wyżej algorytmów przeprowadziłem przeszukiwanie w celu znalezienia możliwie najdłuższej takiej ścieżki.
     W przypadku $n\le6$ przy pomocy przeszukiwania dokładnego, a dalej NMCS z 4 poziomami dla $n=7$, 3 dla $n=8$, 2 dla $9\le n\le12$ i 1 dla $13\le n\le18$
     z ilością wywołań niższego poziomu takich żeby podobny był czas dla wszystkich $n$. Dalej użyłem zwykłego błądzenia losowego i wybierania najdłuższej
     uzyskanej tak ścieżki (ilość przeszukiwań znów dostosowana do tego czasu).
     Program użyty do takiego przeszukiwania wraz z zapisami ścieżek dla mniejszych $n$ (dla większych brak ze względu na wielkość plików)
     i dokładnymi parametrami jest dołączony do pracy.\newline
     
     Poniższa tabelka i wykres logarytmiczny prezentują porównanie najlepszych rezultatów uzyskanych dla poszczególnych $n$ ($|P|$) do wielkości ścieżki
     Fibonacciego i całego grafu.
     $\begin{array}{|c|c|c|c|c|c|c|c|c|c|c|c|c|c|c|}
      \hline
      n&0&1&2&3&4&5&6&7&8&9&10&11&12&13\\
      \hline
      F_{n+1}&1&2&3&5&8&13&21&34&55&89&144&233&377&610\\
      \hline
      |P|+1&1&2&3&5&8&14&27&51&86&146&245&423&749&1373\\
      \hline
      2^n&1&2&4&8&16&32&64&128&256&512&1024&2048&4096&8192\\
      \hline
      \sqrt[n]{|P|+1}&&2&1.73&1.71&1.68&1.7&1.73&1.75&1.75&1.74&1.73&1.73&1.74&1.74\\
      \hline
     \end{array}$\newline
     $\begin{array}{|c|c|c|c|c|c|c|c|c|c|}
      \hline
      n&14&15&16&17&18&19&20&21&22\\
      \hline
      F_{n+1}&987&1597&2584&4181&6765&10946&17711&28657&46368\\
      \hline
      |P|+1&2568&4778&9017&16612&28287&34801&63271&118344&216033\\
      \hline
      2^n&16384&32768&65536&131072&262144&524288&1048576&2097152&4194304\\
      \hline
      \sqrt[n]{|P|+1}&1.75&1.76&1.77&1.77&1.77&1.73&1.74&1.74&1.75\\
      \hline
     \end{array}$\newline
     $\begin{array}{|c|c|c|c|c|c|c|c|}
      \hline
      n&23&24&25&26&27&28&29\\
      \hline
      F_{n+1}&75025&121393&196418&317811&514229&832040&1346269\\
      \hline
      |P|+1&410719&748502&1373384&2543441&4774783&9098482&16788856\\
      \hline
      2^n&8388608&16777216&33554432&67108864&134217728&268435456&536870912\\
      \hline
      \sqrt[n]{|P|+1}&1.75&1.76&1.76&1.76&1.77&1.77&1.77\\
      \hline
     \end{array}$\newline
      $\begin{array}{|c|c|c|c|c|c|}
      \hline
      n&30&31&32&33&34\\
      \hline
      F_{n+1}&2178309&3524578&5702887&9227465&14930352\\
      \hline
      |P|+1&32747927&61637291&117676035&204031449&386051791\\
      \hline
      2^n&1073741824&2147483648&4294967296&8589934592&17179869184\\
      \hline
      \sqrt[n]{|P|+1}&1.78&1.78&1.79&1.79&1.79\\
      \hline
     \end{array}$\newline
     \begin{center}
      \includegraphics[scale=0.55]{img/plot1.jpg}
     \end{center}
     Na podstawie tabelki i wykresu można zauważyć, że pomimo spadającej jakości wyników (w szczególny sposób widoczej przy zmianie poziomu NMCS) pierwiastek
     $n$--tego stopnia z wyniku ma tendencję wzrostową, co nasuwa wniosek, że dla optymalnych wyników ciąg takich pierwiastków zbiega do 2. 
     Widać również, że dla większych $n$ ścieżka Fibonacciego  osiąga już bardzo słabe wyniki.
   \section{Znane ograniczenia długości węży}
    Problemem najdłuższej ścieżki indukowanej i najdłuższego cyklu indukowanego w hiperkostce zajmowali się już wcześniej inni badacze, nazywając
    problemy odpowiednio "wąż w pudełku" (snake-in-a-box) i "zwój w pudełku" (coil-in-a-box) i osiągając lepsze rezultaty.\newline
    Dla $n\le7$ wyniki otrzymane przeze mnie są optymalne, jednak dla dalszych są znane lepsze\newline
    (8-98,9-190,10-370,11-708,12-1357,13-2687).
    Znane są również ograniczenia dolne i górne lepsze od tych oczywistych, oraz tych opisanych przeze mnie w poprzednich podrozdziałach.
    \subsection{Ograniczenia dolne}
     Od kiedy zdefiniowany został problem pojawiały się coraz lepsze ograniczenia dolne
     ($c\cdot2^{\frac{n}{2}}$,$(1.5)^{n}$,$c\cdot6^{\frac{n}{4}}$,$c\cdot(2.5)^{\frac{n}{2}}$), później długo dominowało ograniczenie z pracy \cite{Snake1}
     Dające ograniczenie $\frac{7}{4}\frac{2^n}{n-1}$ (które jest już lepsze od mojego wynikającego z konstrukcji ścieżek Fibonacciego).\newline
     Jeszcze lepsze ograniczenie wraz z konstrukcją ścieżki i cyklu pojawia się w pracy \cite{Snake2} (przedstawię szkic tego rozwiązania poniżej).
     \begin{fact}
      Cykl można przerobić na ścieżkę poprzez usunięcie dowolnego wierzchołka tracąc 2 na długości.
     \end{fact}
     \begin{theorem}\label{snake 9/64 2^n}
      Dla $Q_n$, $n\ge7$ można skonstruować cykl indukowany o długości $\frac{9}{64}\cdot2^n$
     \end{theorem}
     \begin{proof}(szkic konstrukcji i dowodu z pracy \cite{Snake2})
      $Q_n$ dzielimy na kostki wymiaru $5$ otrzymując graf $Q_{n-5}\times Q_5$. W grafie $Q_{n-5}$ bierzemy cykl Hamiltona otrzymany indukcyjnie:
      dla $n-5=2$ jedyny możliwy ($(0,0)\rightarrow(0,1)\rightarrow(1,1)\rightarrow(1,0)\rightarrow(0,0)$),
      dalej dla otrzymania cyklu dla $Q_{m+1}$ bierzemy dwie kostki $Q_m$ z takimi cyklami i łączymy je usuwając krawędź, która wchodzi do wierzchołka
      $\overline{0}$ i łącząc te dwie ścieżki w jeden cykl w $Q_{m+1}$ (konstrukcja jak w rozdziale \ref{Ham}).
      Kostki $Q_5$ występujące na tym cyklu można ponumerować startując od $\overline{0}$ w kierunku $(0,...,0,1)$.\newline
      Kolejnym krokiem jest zdefiniowanie ciągu $2^{n-5}$ ścieżek indukowanych w $Q_5$ (ponumerowanych według cyklu Hamiltona)
      spełniających następujące własności:
      \begin{itemize}
       \item ścieżka $m$ kończy się tam, gdzie zaczyna się ścieżka $m+1 (\mod 2^{n-5})$ i jest to jedyny ich wspólny wierzchołek.
       \item ścieżki z dwóch kostek $Q_5$ będących sąsiadami w $Q_{n-5}$, ale nie na cyklu Hamiltona nie zawierają wspólnych wierzchołków.
       \item ścieżki o numerach $m=1$ lub $2(\mod 4)$ mają długość $3$, zaś o numeracji $m=0$ lub $3(\mod 4)$ długość $4$.
      \end{itemize}\newpage
      Ścieżki konstruowane są poprzez chodzenie po grafie G (widocznym na obrazku) zanurzonym w $Q_5$:
      \begin{center}
       \includegraphics[scale=1]{img/snake_G.jpg}
      \end{center}
      Zanurzenie jest zadane dla ścieżki $b_1..a_1...b_2$ wraz z pośrednimi wierzchołkami poprzez ścieżkę
      $(0,0,0,0,0),(0,0,0,0,1),(0,0,0,1,1),(0,1,0,1,1),(0,1,0,1,0),(0,1,1,1,0),(1,1,1,1,0),$\newline
      $(1,1,1,1,1)$, zaś dla pozostałych czterech analogicznie po cyklicznej zamianie współrzędnych\newline
      ($a_1=(0,1,0,1,1),a_2=(1,0,1,0,1),a_3=(1,1,0,1,0),a_4=(0,1,1,0,1),a_5=(1,0,1,1,0)$).
      Ścieżkami długości $3$ są odpowiedniki $a_i\rightarrow b_1$ i $b_1\rightarrow a_j$, zaś ścieżek długości $4$ ścieżki
      $a_j\rightarrow b_2$ i $b_2\rightarrow a_i$. Kolejne ścieżki tworzone są poprzez chodzenie w ten sposób po grafie $G$ z odpowiednim
      wyborem jednej z 5 ścieżek kiedy startujemy w $b_i$.
      Postępując w ten sposób otrzymujemy pożądany cykl indukowany (dowód drugiego warunku można znaleźć w pracy \cite{Snake2}).
      Po połączeniu wszystkich tych ścieżek otrzymujemy cykl długości $(\frac{3+4}{2}+1)\cdot 2^{n-5}=\frac{9}{64}\cdot2^n$.
     \end{proof} 
     \begin{corollary}
      Długość maksymalnej ścieżki indukowanej w $Q_n$ jest ograniczona z dołu przez $\frac{9}{64}\cdot 2^n-2$.
     \end{corollary}
    \subsection{Ograniczenia górne}
     \begin{defi}\label{pseudowąż}
      \emph{Pseudowężem} nazwiemy dowolny indukowany podgraf hiperkostki, w którym stopień każdego wierzchołka jest nie większy niż 2. 
     \end{defi}
     Każdy wąż i każdy zwój jest pseudowężem (pseudowąż składa się ze zbioru indukowanych ścieżek i cykli oraz wierzchołków izolowanych).
     \begin{fact}
      Obcięcie pseudowęża do części z jednej podkostki jest pseudowężem w tej podkostce (w całej hiperkostce też).
     \end{fact}
    \begin{theorem}
     dla $n\ge5$ każdy dowolny pseudowąż (a więc i wąż) w $Q_n$ ma rozmiar co najwyżej $2^{n-1}$.
    \end{theorem}
    \begin{proof}
     (Dowód za \cite{Snake1}).
     Dla $n=4$ jest tylko jeden z dokładnością do automorfizmów kostki pseudowąż rozmiaru $9$ (większych nie ma).
     \begin{center}
      \includegraphics[scale=0.4]{img/pseudoS_4.jpg}
     \end{center}
     Gdyby w $Q_5$ był pseudowąż rozmiaru $17$, to przy dowolnym podziale na dwie podkostki $Q_4$ w jednej z nich musiałby sie znajdować pseudowąż
     rozmiaru co najmniej $9$, a więc ten z rysunku. W drugiej kostce wierzchołki odpowiadające tym z cyklu nie mogą zostać wybrane,
     czyli w skład węża musiałby wchodzić wszystkie pozostałe, jednak taki graf nie byłby pseudowężem.\newline
     Dalej dla $n\ge6$ indukcyjnie. Przy dowolnym podziale na dwie podkostki $Q_{n-1}$ w jednej z nich musiałby się znajdować pseudowąż rozmiaru co najmniej
     $2^{n-2}+1$, co daje sprzeczność.
    \end{proof}
    \begin{observation}
     W pracy \cite{Snake3} pojawiło się mocniejsze ograniczenie górne na długość węża w pudełku:
     $2^{n-1}\cdot(1-\frac{1}{n^2-n+2})$ (dla $n\ge7$), które zostało później jeszcze ulepszone do\newline
     $2^{n-1}\cdot(1-\frac{1}{20n-41})$ (dla $n\ge12$) w pracy \cite{Snake4}.
    \end{observation}


\begin{thebibliography}{99}
\addcontentsline{toc}{chapter}{Bibliografia}

  \bibitem{Ruskey} Frank Ruskey,
   "Combinatorial Generation"
   \textit{Working Version, October 1, 2003}

  \bibitem{LPV} L. Lov\'{a}sz, J. Pelik\'{a}n and K. Vesztergombi.
   "Discrete Mathematics: Elementary and Beyond."
   \textit{Undergraduate Texts in Mathematics. New York: Springer, first edition, 2003}   

  \bibitem{HAR} L. H. Harper,
   "Optimal Numberings and Isoperimetric Problems on Graphs"
   \textit{Journal of Combinatorial Theory 1, 385-393 (1966)}
   
  \bibitem{HHH} Frank Harary, John P. Hayes and Horng--Jyh Wu,
   "A survey of the theory of hypercube graphs"
   \textit{Comput. Math. Applic. Vol. 15, No 4, pp. 277-289, 1988}
   
  \bibitem{DFGKR} Tom\'{a}\v{s} Dvo\v{r}\'{a}k, Ji\v{r}\'{\i} Fink, Petr Gregor, V\'{a}clav Koubek and Tomasz Radzik,
   "Efficient connectivity testing of hypercubic networks with faults"
   
  \bibitem{FG} Ji\v{r}\'{\i} Fink and Petr Gregor,
   "Long paths and cycles in hypercubes with faulty vertices"
   
  \bibitem{Wie} G. Wiener,
   "Edge multiplicity and other trace functions."
   \textit{In Proceedings of European Conferenc on Combinatorics, Graph Theory and Applications (EuroComb 2007), volume 29 of
   Electronic Notes in Discrete Mathematics, pages 491–495, 2007.}
   
  \bibitem{FG2} Ji\v{r}\'{\i} Fink and Petr Gregor,
   "Long pairs of paths in faulty hypercubes"
  
  \bibitem{Locke} Nelson Casta\~{n}eda and Ivan S. Gotchev,
   "Proof of Locke's Conjecture, I"

  \bibitem{Pegr} David P\v{e}g\v{r}\'{\i}mek,
   "Hamiltonian cycles in hypercubes with removed vertices",
   \textit{Charles University, Prague 2013}
  
  \bibitem{SCJY} Sun, Chao-Ming and Jou, Yue-Dar,
   "Hamiltonian laceability of hypercubes with some faulty elements",
   \textit{In Proceedings of 2009 IEEE International Conference on Networking, Sensing and Control, Okayama, Japan,
   March 26-29, 2009, pp.626-630}
  
  \bibitem{NMCS} Tristan Cazenave,
   "Nested Monte-Carlo Search"
   \textit{LAMSADE Universite Paris-Dauphine Paris, France}  
  
  \bibitem{Snake1} Ludwig Danzer and Victor Klee,
   "Lengths of Snakes in Boxes",
   \textit{Journal of combinatorial theory 2, 258-265 (1967)}

  \bibitem{Snake2} J. Wojciechowski,
   "A new lower bound for Snake-in-the-Box Codes",
   \textit{Combinatorica 9 (1) (1989) 91--99}
  
  \bibitem{Snake3} F. I. Solov'jeva ,
   "Upper bound for the length of a cycle in an n-dimensional unit cube",
   \textit{Methods of Diskrete Analiz 45 (1987)}

  \bibitem{Snake4} Hunter S. Snevily ,
   "The snake-in-the-box problem: A new upper bound",
   \textit{Discrete Mathematics 133 (1994) 307-314}
   
   %------------------------------------------------------------pompowanie 
   
  \bibitem{HL} Frank Harary, Marilynn Livingston,
   "Independent domination in hypercubes"
   \textit{Appl. Math. Lett. Vol. 6, No 3, pp. 27-28, 1993}
   
  \bibitem{RS} Wojciech Rytter $\&$ Bartosz Szreder,
   "Wprowadzenie do kombinatoryki algorytmicznej"
   
  \bibitem{Knuth} Donald E. Knuth,
   "Generating All Tuples and Permutations"
   \textit{The Art of Computer Programming volume 4, fascicle 2}
   
\end{thebibliography}

\end{document}