\documentclass{pracamgr}
\usepackage{polski}
\usepackage[utf8]{inputenc}


\author{Wiktor Zuba}

\nralbumu{320501}

\title{Metody ulepszania systemów rekomendacyjnych}

\tytulang{Methods of recommendation systems improving}

\kierunek{Matematyka}

\opiekun{prof. Hung Son Nguyen\\
Instytut Informatyki}

\date{??? 2016}

\dziedzina{ 
11.0 Matematyka, Informatyka:?????????????????????????\\
11.2 Statystyka??????????????????\\
11.4 Sztuczna inteligencja????????????\\ 
}


\klasyfikacja{62-XX Statistics??????????????????????????????\\
68-XX Computer science????????????????????????????????????\\
}

\keywords{???????????????????????}



\begin{document}
\maketitle

\begin{abstract}
???????????????????????????????????????
\end{abstract}


\tableofcontents

\chapter*{Wprowadzenie}
\addcontentsline{toc}{chapter}{Wprowadzenie}


 \chapter{Sformułowanie problemu}
  W każdym modelu danych wykorzystywanym do wyliczania możliwych ocen przedmiotów lub sporządzania list rekomendacji
  występują użytkownicy i przedmioty, oraz pewne niepełne informacje na temat ich powiązań ze sobą.
  Aby móc przewidzieć czy dany nie oceniony jeszcze przedmiot zainteresuje konkretnego użytkownika trzeba dla obu określić preferencje na podstawie 
  załączonych opisów, znanych interakcji tego użytkownika z innymi przedmiotami i przedmiotu z innymi uzytkownikami, lub też obu informacji na raz.
  
  
  \section{Podstawowe definicje}
   \textbf{Definicja 1.1.1.} Użytkownik -- osoba korzystająca z serwisu, dla której staramy się stworzyć rekomendacje lub też,
    która tylko dostarcza informacji użytecznych przy ich sporządzaniu dla innych użytkowników.\newline\newline
   %
   \textbf{Definicja 1.1.2.} Przedmiot -- rzecz która jest używana (czasem też oceniana) przez użytkowników.\newline
   
    Przedmiot musi być reużywalny (np. filmy do oglądnięcia, książki do przeczytania, strony internetowe do odwiedzenia),
    lub występujący w wielu identycznych egzemplarzach (przedmioty w sklepach), aby użycie przez innego uzytkownika nie wyczerpało jego zasobu
    (wtedy rekomendacja nie miała by sensu, skoro przedmiot nie istnieje) i dało miarodajne informacje na jego temat.
    Przedmiot jest rzeczą, której ocenę przez użytkownika chcemy przewidzieć, lub którą chcemy mu zarekomendować.\newline\newline
   %
   \textbf{Definicja 1.1.3.} Użycie przedmiotu -- zarejestrowana informacja o interakcji użytkownika z przedmiotem,
   dostępna w danych wykorzystywanych do tworzenia rekomendacji -- niekoniecznie prawdziwe używanie przedmiotu
   (kupienie produktu nie oznacza jego używania a wypożyczenie książki jej przeczytania). \newline\newline
   %
   \textbf{Definicja 1.1.4.} Informacja explicite -- wyraźne informacje dostarczone przez użytkownika odnośnie przedmiotu -- głównie
    ocena, recenzja lub przypisanie tagów.\newline\newline
   %
   \textbf{Definicja 1.1.5.} Informacja implicite -- bezwarunkowe informacje o użyciu przedmiotu przez użytkownika
    (kupienie towaru, obejrzenie filmu, wypożyczenie książki, odwiedzenie strony).\newline
    
    Informacje bezwarunkowe, których główną zaletą jest łatwość ich zbierania
    (dzieje się to automatycznie, bez dodatkowych akcji ze strony uzytkownika, a czasem i bez jego wiedzy),
    są niestety bardzo narażone na przekłamania odnośnie preferencji użytkownika.
    Po pierwsze zarejestrowanie uzycia przez system wcale nie musi oznaczać faktycznego użycia
    (przypadkowe kliknięcie linku, pomyłka we wpisywaniu nazwy, zrezygnowanie po przeczytaniu opisu), 
    a po drugie z użycia nie musi wynikać pozytywny odbiór.
    W szczególności użycie przedmiotu wynikające z nietrafionych rekomendacji może pogłębiać zawodność systemu.
    Pomimo licznych wad informacje implicite często są jedynymi posiadanymi, a nawet przy dostępnych danych explicite ze względu na
    znacznie większą ilość znajdują istotne zastosowanie w systemach rekomendacyjnych.
    Dane o wielokrotnym użyciu przedmiotu są często spłaszczone do binarnych (przedmiot użyty/nie użyty),
    co pozwala na zmniejsznie rozmiaru pamięci potrzebnej do ich przechowywania, oraz użycie niektórych metod rekomendacji.\newline\newline
   %
   \textbf{Definicja 1.1.6.} Lista rekomendacji -- (zazwyczaj uporządkowany) podzbiór przedmiotów nie użytych dotychczas przez użytkownika,
    które powinny się mu spodobać (być najwyżej ocenione, chętnie użyte).
    
  \section{Przeznaczenie systemów}
   Istnieją dwa główne cele postawione przed systemami rekomendacyjnymi:
   \begin{itemize}\itemsep1pt \parskip0pt \parsep0pt
    \item zbudowanie listy rekomendacyjnej konkretnej długości
    \item przewidzenie niewprowadzonych ocen przedmiotów
   \end{itemize}
   Oczywiście jeżeli system przewidzi oceny wszystkich nieocenionych przedmiotów,
   łatwo można je uszeregować malejąco i obciąć listę w odpowiednim miejscu otrzymując listę rekomendacji pożądanej długości.\newline
   Konwersja w drugą stronę jest znacznie trudniejsza, a bez dodatkowych informacji praktycznie niemożliwa.\newline
   Posiadanie przewidzianych ocen przedmiotów może być bardzo przydatne -- serwis proponujący przedmioty może przedstawiać użytkownikowi
   rekomendacje mocniejsze i słabsze w inny sposób, lub też obciąć listę rekomendacji tylko do przedmiotów o przewidzianej ocenie powyżej pewnego poziomu.
   Dodatkową motywacją do wyliczenia przewidywanych ocen jest możliwość podania tych informacji użytkownikowi wraz z rekomendacjami
   dla mocniejszego zainteresowania użytkownika przedmiotem, oraz zwiększenia satysfakcji z systemu.\newline
   Niestety w niektórych przypadkach na przykład przy posiadaniu jedynie informacji implicite określenie ocen przedmiotów jest niemożliwe
   (chyba że wprowadzimy sztuczne oceny na podstawie miejsc w liście rankingowej). Dodatkowo algorytm generowania rekomendacji którego wynikiem
   jest tylko lista rekomendacji (bez ocen) może posiadać pożądane przez nas zalety jak szybkość lub oszczędność pamięci,
   są więc one wykorzystywane w sytuacjach w których posiadanie przewidzianych ewaluacji nie ma dodatkowych zastosowań.
   
   
   
  \section{Problemy stojące przed systemami rekomendacyjnymi}
   
   
   
   
   
   
   


 \chapter{Podstawowe techniki}
  \section{Filtrowanie kolaboratywne}
   Jedną z najbardziej podstawowych technik przewidywania 
 

\begin{thebibliography}{99}%TODO
\addcontentsline{toc}{chapter}{Bibliografia}






\end{thebibliography}

\end{document}