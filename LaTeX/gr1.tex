\documentclass{article}
\usepackage[utf8]{inputenc}
\usepackage{polski}
\usepackage{amsmath}
\usepackage{anysize}
\usepackage{amssymb}
\begin{document}

Wiktor Zuba 320501
\newline

Zadanie 1.1.
\newline
\newline
SL(2)=$\left\{\left|\begin{array}{cc}a&b\\c&d\end{array}\right|:ad-bc=1\right\}$\newline
istnieje atlas złożony z dwóch map z SL(2) w $\mathbb{R}^3$ :
$\varphi(\left|\begin{array}{cc}a&b\\c&d\end{array}\right|)=(a,b,c)$ dla $a\neq0$,
$\psi(\left|\begin{array}{cc}a&b\\c&d\end{array}\right|)=(a,b,d)$ dla $b\neq0$.
Mapy łącznie pokrywają całe SL(2), gdyż dla $a=b=0$ wyznacznik również byłby równy $0$
Parametryzacje przyjmują postaci $\varphi^{-1}(a,b,c)=\left|\begin{array}{cc}a&b\\c&\frac{1+bc}{a}\end{array}\right|$,
$\psi^{-1}(a,b,d)=\left|\begin{array}{cc}a&b\\\frac{ad-1}{b}&d\end{array}\right|$,
a więc wyrażone są tylko poprzez funkcje wymierne, które są gładkie, więc SL(2) jest gładką rozmaitością różniczkową.\newline
$\psi\circ\varphi^{-1}(a,b,c)=(a,b,\frac{1+bc}{a})$, jest dyfeomorfizmem gładkim, odwrotny do niego również jest gładki ($(a,b,\frac{ad-1}{b})$), więc mapy są zgodne.
\newline

Zadanie 1.3.
\newline
\newline
$\eta(x,y,z)=(\frac{x}{1-z},\frac{y}{1-z})$\newline
$f\circ\eta(x,y,z)=(\frac{x^3-3xy^2}{(1-z)^3},\frac{3x^2y-y^3}{(1-z)^3})$\newline
$\eta^{-1}\circ f\circ\eta(x,y,z)=
(\frac{2(x^3-3xy^2)(1-z)^3}{(x^2+y^2)^3+(1-z)^6},\frac{2(3x^2y-y^3)(1-z)^3}{(x^2+y^2)^3+(1-z)^6},\frac{(x^2+y^2)^3-(1-z)^6}{(x^2+y^2)^3+(1-z)^6})$\newline
$F$ jest oczywiście gładkie w $S^2\backslash\{N\}$. Granica funkcji po argumentach dążących do $\{N\}$ jest równa $(0,0,1)$ (z lematu d'Hospitala),
jest więc ono ciągłe też tam, a jako funkcja wymierna jest więc też gładkie. Przekształcenie $\eta$ jest na $\mathbb{R}^2$, zaś $\eta^{-1}$ na $S^2\backslash\{N\}$,
więc z surjektywności $f$ wynika również suriektywność $F$
\newline

Zadanie 1.4.
\newline
\newline
a) Gdyby dla pewnego $p$ zbiór punktów $\pi^{-1}\circ\pi(p)$ był nieskończony, to gdyby ułożyć go w ciąg ze zwartości $W$
istniałby w tym ciągu podciąg zbieżny. Oznaczałoby to, że w dowolnie bliskim otoczeniu punktu do którego zbiega ten podciag
nieskończenie wiele punktów przyjmuję tę samą wartość przy przekształceniu $\pi$, a więc $\pi$ nie byłoby lokalnie różnowartościowe,
co przeczy byciu lokalnym dyfeomorfizmem.\newline
b) (ilość$\circ\pi^{-1}$) jest przekształceniem ciągłym (a przynajmniej lokalnie), ze zbioru spójnego w dyskretny,
a więc z jednej z definicji spójności przekształceniem stałym.\newline
c) z treści zadania mamy lokalną dyfeomorficzność,
zaś z podpunktów a) i b) równoliczność przeciwobrazów punktów i ich dyskretność, a więc $\pi$ jest nakryciem.\newline
d) identyczność z otwartej półprostej $W=(0,\infty)$ w $M=\mathbb{R}$.
Jest lokalnym dyfeomorfizmem i niewątpliwie jest to przekształcenie gładkie,
jednak nie jest nakryciem, gdyż $\pi^{-1}$ moze przyjmować 1 lub 0 wartości.
\end{document}