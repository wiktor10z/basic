\documentclass{article}
\usepackage[utf8]{inputenc}
\usepackage{polski}

\begin{document}

Wiktor Zuba 320501 grupa 4
\newline
Zadanie 2.
\newline


$
a>b>0
$
$
\int\limits_{0}^{\infty}
\frac{e^{-ax}-e^{-bx}}{x} dx
$
\newline
"pochodna" funkcji podcałkowej względem a : 
$
\bigl( \frac{e^{-ax}-e^{-bx}}{x} \bigr)^{\prime}
=
\frac{-xe^{-ax}}{x}=-e^{-ax}
$
\newline
mamy funkcję od y 
$
-e^{-yx} \quad
\int\limits_{b}^{a}
-e^{-yx} dy
=
\frac{e^{-ax}-e^{-bx}}{x}
$
więc:
$
\int\limits_{0}^{\infty}
\frac{e^{-ax}-e^{-bx}}{x} dx
=
\int\limits_{0}^{\infty}
\bigl(
\int\limits_{b}^{a}
-e^{-yx} dy
\bigr)
dx
=
- \int\limits_{0}^{\infty}
\bigl(
\int\limits_{b}^{a}
e^{-yx} dy
\bigr)
dx
$
funkcja podcałkowa jest mierzalna i nieujemna więc spełnia założenia twierdzenia Fubiniego:
\newline
$
- \int\limits_{0}^{\infty}
\bigl(
\int\limits_{b}^{a}
e^{-yx} dy
\bigr)
dx
=
\int\limits_{b}^{a}
\bigl(
\int\limits_{0}^{\infty}
-e^{-yx} dx
\bigr)
dy
=
\int\limits_{b}^{a}
\bigl[
\frac{e^{-yx}}{y}
\bigr]_{x=0}^{x=\infty} dy
=
\int\limits_{b}^{a} -\frac{1}{y} dy
=
\bigl[-\ln{y}
\bigr]_{b}^{a}
=
\ln{b}-\ln{a}=
\ln{\frac{b}{a}}
$

\end{document}
