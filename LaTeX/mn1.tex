\documentclass{article}
\usepackage[utf8]{inputenc}
\usepackage{polski}
\usepackage{amsmath}
\usepackage{anysize}
\usepackage{amssymb}
\begin{document}
 
Wiktor Zuba 320501 grupa 1
\newline

Zadanie 1.
\newline
$
\varphi(\alpha)=\alpha,\varphi'(\alpha)=0,\cdots,\varphi^{(n)}(\alpha)=0,\varphi^{(n+1)}(\alpha)\neq0\newline
x_{k+1}=\varphi(x_k)\newline\newline
$
Rozwijając $\varphi(x_k)$ do wzoru Taylora wokół $\alpha$ otrzymujemy(dla pewnego $\zeta_k$ z przedziału $[\alpha,x_k]$):\newline
$
x_{k+1}=\varphi(\alpha)+\varphi'(\alpha)(x_k-\alpha)+\frac{\varphi''(\alpha)(x_k-\alpha)^2}{2}+\cdots+\frac{\varphi^{(n)}(\alpha)(x_k-\alpha)^n}{n!}+\frac{\varphi^{(n+1)}(\zeta_k)(x_k-\alpha)^{n+1}}{(n+1)!}\newline
x_{k+1}=\alpha+\frac{\varphi^{(n+1)}(\zeta_k)(x_k-\alpha)^{n+1}}{(n+1)!}\newline
x_{k+1}-\alpha=\frac{\varphi^{(n+1)}(\zeta_k)(x_k-\alpha)^{n+1}}{(n+1)!}\newline
$
Zakładając, że $\varphi^{(n+1)}$ jest ciągła na pewnym otoczeniu $\alpha$ (zawierającym $x_k$) możemy przyjąć, że jest ograniczona przez pewne M
($|\varphi^{(n+1)}(\zeta)|\le M$). Biorąc $\varepsilon_k=x_k-\alpha$ otrzymujemy:\newline
$
\varepsilon_{k+1}=\frac{\varphi^{(n+1)}(\zeta_k)}{(n+1)!}(\varepsilon_k)^{n+1}\quad\quad\quad
|\varepsilon_{k+1}|\le|\frac{M}{(n+1)!}(\varepsilon_k)^{n+1}|=\frac{M}{(n+1)!}(|\varepsilon_k|)^{n+1}\newline
$
A więc od pewnego "startowego" $x_k$ dobranego tak, by ciąg $\varepsilon_k$ zbiegał ($\varepsilon_k\le\sqrt[n]{\frac{(n+1)!}{M}}$)
wykładnik zbieżności dla wyżej zadanej funkcji iteracyjnej jest równy n+1
(nie większy, ponieważ $\varphi^{(n+1)}(\alpha)\neq0$ to dla pewnego otoczenia $\alpha$ $\varphi^{(n+1)}(x)$ jest również odgrodzona od 0 przez pewne K
po przecięciu z poprzednim otoczeniem otrzymujemy również, że 
$
|\varepsilon_{k+1}|\ge\frac{K}{(n+1)!}(|\varepsilon_k|)^{n+1}
$).
\end{document}