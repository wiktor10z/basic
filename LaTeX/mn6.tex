\documentclass{article}
\usepackage[utf8]{inputenc}
\usepackage{polski}
\usepackage{amsmath}
\usepackage{anysize}
\usepackage{amssymb}
\begin{document}
 
Wiktor Zuba 320501 grupa 1
\newline

Zadanie 6.
\newline
\newline
Sprawdzenie dla macierzy $m\times n$ 4 warunków pseudoodwrotności\newline
($AA^+=(AA^+)^{\top},A^+A=(A^+A)^{\top},AA^+A=A,A^+AA^+=A^+$)
w dwóch przypadkach:\newline\newline
$\text{rz} A=n,A^+=(A^{\top}A)^{-1}A^{\top}:\newline
(AA^+)^{\top}=(A(A^{\top}A)^{-1}A^{\top})^{\top}=A((A^{\top}A)^{\top})^{-1}A^{\top}=A(A^{\top}A)^{-1}A^{\top}=AA^+\newline
(A^+A)^{\top}=((A^{\top}A)^{-1}A^{\top}A)^{\top}=(A^{\top}A)(A^{\top}A)^{-1}=I_n=(A^{\top}A)^{-1}(A^{\top}A)=A^+A\newline
AA^+A=A(A^{\top}A)^{-1}A^{\top}A=A(A^{\top}A)^{-1}(A^{\top}A)=A\cdot I_n=A\newline
A^+AA^+=(A^{\top}A)^{-1}A^{\top}A(A^{\top}A)^{-1}A^{\top}=(A^{\top}A)^{-1}(A^{\top}A)(A^{\top}A)^{-1}A^{\top}=(A^{\top}A)^{-1}A^{\top}=A^+\newline\newline
\text{rz} A=m,A^+=A^{\top}(AA^{\top})^{-1}:\newline
(AA^+)^{\top}=(AA^{\top}(AA^{\top})^{-1})^{\top}=((AA^{\top})^{\top})^{-1}AA^{\top}=(AA^{\top})^{-1}AA^{\top}=I_m=AA^{\top}(AA^{\top})^{-1}=AA^+\newline
(A^+A)^{\top}=(A^{\top}(AA^{\top})^{-1}A)^{\top}=A^{\top}(AA^{\top})^{-1}A=A^+A\newline
AA^+A=AA^{\top}(AA^{\top})^{-1}A=(AA^{\top})(AA^{\top})^{-1}A=A\newline
A^+AA^+=A^{\top}(AA^{\top})^{-1}AA^{\top}(AA^{\top})^{-1}=A^{\top}(AA^{\top})^{-1}(AA^{\top})(AA^{\top})^{-1}=A^{\top}(AA^{\top})^{-1}=A^+\newline\newline
$
Przy przejściach z identycznością korzystam z maksymalności rzędu odpowiednich macierzy.

\end{document}