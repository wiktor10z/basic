\documentclass{article}
\usepackage[utf8]{inputenc}
\usepackage{polski}
\usepackage{amsmath}
\usepackage{anysize}
\usepackage{textcomp}
\usepackage{amssymb}
\marginsize{1cm}{1cm}{1cm}{0,7cm}
\newcommand{\norma}[1]{\lVert{#1}\rVert}
\begin{document}
$\newline
f'(x)=\lim\limits_{h\rightarrow0}\frac{f(x+h)-f(x)}{h}\newline
$Jeżeli $f,g$ różniczkowalne w $x_0$ ,to:\quad
$(f\pm g)'(x_0)=f'(x_0)\pm g'(x_0)\quad
(f\cdot g)'(x_0)=f'(x_0)g(x_0)+f(x_0)g'(x_0)\quad
g(x_0)\neq0\Rightarrow(\frac{f}{g})\in C^1(\{x_0\}),(\frac{f}{g})'(x_0)=\frac{f'(x_0)g(x_0)+f(x_0)+g'(x_0)}{g^2(x_0)}\quad
f(g(x))'=f'(g(x))\cdot g'(x)\newline
g=f^{-1},g\in C(y_0):y_0=f(x_0)\Rightarrow g\in C^1(y_0),g'(y_0)=\frac{1}{f'(x_0)}\newline
c$-ekstremum lokalne funkcji $f\Rightarrow f'(c)=0|\Leftarrow$ jeżeli istnieją kolejne pochodne i pierwsza$\neq0$ - nieparzysta$\newline
$Wł.Darboux $f'(c)<A<f'(d)\Rightarrow\exists_{x_0\in(c,d)\vee(d,c)}:f'(x_0)=A\newline
$Tw.Rolle'a $f:[a,b],f(a)=f(b)\rightarrow \mathbb{R},f\in C^1((a,b))\Rightarrow\exists_{c\in(a,b)}:f'(c)=0\newline
$Tw.Lagrange'a $f:[a,b],f\in C^1((a,b))\Rightarrow\exists_{c\in(a,b)}:f(c)=\frac{f(b)-f(a)}{b-a}\newline
$Wzór Cauchy'ego $f,g:[a,b]\rightarrow \mathbb{R}, f,g\in C^1((a,b))\Rightarrow\exists_{c\in(a,b)}:(f(b)-f(a))g'(c)=(g(b)-g(a))f'(c)\newline
\left(\frac{f(b)-f(a)}{g(b)-g(a)}=\frac{f'(c)}{g'(c)} ,(\text{gdy }g'(c)\neq0)\right)\quad C\Rightarrow L\Rightarrow R\newline
\sum\limits_{n=1}^{\infty}a_nx^n=f(x)\in C((-R,R)):R=(\limsup\sqrt[n]{|a_n|})^{-1}\quad
$dla $R>0$ $f(x)\in C^{\infty}((-R,R))$ i $f'(x)=\sum\limits_{n=1}^{\infty}na_nx^{n-1},f'(x)\in C((-R,R))\newline
$Reguła de l'Hospitala $f(a)=g(a)=\left\{\begin{array}{c}=0\\=\pm\infty\end{array}\right.$ i $\exists\lim\limits_{x\rightarrow a^+(a^-)}\frac{f'(x)}{g'(x)}$,
to $\lim\limits_{x\rightarrow a^+(a^-)}\frac{f(x)}{g(x)}=\lim\limits_{x\rightarrow a^+(a^-)}\frac{f'(x)}{g'(x)}\quad
$jak dalej $\frac{0}{0}\vee\frac{\infty}{\infty}$ to dalsze pochodne$\newline
$Def. funkcja jest n krotnie różniczkowalna w $x_0\in(a,b)$, jeśli jest (n-1) ktornie różniczkowalne w otoczeniu punktu $x_0$
oraz funkcja $f^{(n-1)}(x)$ jest różniczkowalna w punkcie $x_0$ $f^{(n)}=(f^{(n-1)})',f^{(2)}=f''\newline
$Tw(Wzór Taylora) $f:((a,b))\rightarrow \mathbb{R},f\in C^n({x_0}):x_0\in(a,b),
r_n(h)=f(x_0+h)-(f(x_0)+\frac{f'(x_0)}{1!}\cdot h+\frac{f''(x_0)}{2!}\cdot h^2+\cdots+\frac{f^{(n)}(x_0)}{n!}\cdot h^n)\Rightarrow
\lim\limits_{h\rightarrow 0}\frac{r_n(h)}{h^n}=0\newline
$Uwaga $f,x_0$ zachodzi $P_{x_0}(h), deg(P)=n:\frac{f(x_0+h)-P_{x_0}(h)}{h^n}\rightarrow0$ jest tylko jedno takie $P_{x_0}(h)\newline
f\in C({x_0})\Leftrightarrow|f(x_0+h)-f(x_0)|=r_0(h_0)\rightarrow 0,f\in C^n({x_0})\Rightarrow \frac{r_n}{h^n}\rightarrow 0\newline
$Tw-reszty Taylora:$\newline
$Lagrange'a $r_n(h)=\frac{f^{(n+1)}(c)}{(n+1)!}\cdot h^{n+1}:c\in (x_0,x_0+h)\newline
$Cauchy'ego $r_n(h)=\frac{f^{(n+1)}(x_0+\theta h)}{n!}\cdot(1-\theta)^nh^{n+1}:0<\theta<1\newline
$Schlomidta-Roche'a $r_n(h)=\frac{f^{(n+1)}(x_0+\theta h)}{n!k}\cdot(a-\theta)^{n-k+1}\cdot h^{n+1}\quad (k=1\rightarrow C,k=n+1\rightarrow L)\newline
$kiedy funkcja da się rozwinąć w szereg: $f\in C^{\infty}(\text{otocz }x_0),Sz.T(f)=f$ dla $|h|<R\newline
$ inne możliwości- Sz.T zbieżny, ale nie do f/Sz.T rozbieżny $\forall h\neq 0\newline
$Tw.Borela $a_n$- dowolny ciąg liczb $\mathbb{R}$ $\Rightarrow \exists f\in C^{\infty}(\mathbb{R}):f^{(n)}(0)=a_n\newline
$Def. ciąg $f_n$ jest $\underline{\text{zbieżny punktowo}}$ do $f:D\rightarrow \mathbb{R}$ gdy $\forall_{x\in D}f_n(x)^{\longrightarrow}_{n\rightarrow\infty}f(x)\newline
$Def. ciąg $f_n$ jest $\underline{\text{zbieżny jednostajnie}}$ do $f:D\rightarrow \mathbb{R}$ gdy $\forall_{\varepsilon>0}\exists_N\forall_{x\in D}\forall_{n>N}
|f_n(x)-f(x)|<\varepsilon\quad ($oznaczenie $f_n\rightrightarrows f )\newline
$Tw. Granica jednostajnie zbeiżnego ciągu funkcji ciągłych jest funkcją ciągłą $f_n\rightrightarrows f,f_n\in C\Rightarrow f\in C\newline
$Tw. D-odcinek ograniczony, $f_n:D\rightarrow\mathbb{R},f_n\in C^1,f'_n\rightrightarrows g,\exists_{p_0\in D}:f_n(p_0)$ jest ciągiem zbieżnym $
\Rightarrow f_n\rightrightarrows f,f\in C^1,f'=g\newline
$Def. Szereg funkcyjny $\sum\limits_{n=1}^{\infty}f_n(x) (f_n:D\rightarrow\mathbb{R})$ jest zbieżny jednostajnie jeżeli ciąg 
$S_n=\sum\limits_{k=1}^{n}f_k(x)$ jest zbieżny jednostajnie\newline
Stw. Niech $f_n:D\rightarrow\mathbb{R}:|f_n(x)|\le a_n$ (nie zależy od x !), $\sum\limits_{n=1}^{\infty}$ zbieżny 
$\Rightarrow\sum\limits_{n=1}^{\infty}f_n(x)$ jest zbieżny bezwzględnie i jednostajnie\newline
Tw. Abela($\mathbb{R}$) Niech $R>0$ -promień zbieżności szeregu potęgowego $\sum a_nx^n$ jeżeli szereg $\sum a_nR^n(\sum a_n(-R)^n)$
jest zbieżny $\Rightarrow \sum a_nx^n $ jest zbieżny $\underline{\text{jednostajnie}}$ na $[0,R] ([-R,0])\newline
$Tw.Abela($\mathbb(C)$) Niech $R>0$ -promień zbieżności szeregu potęgowego $\sum a_nxz^n$ jeżeli szereg $\sum a_nz_0^n$ jest zbieżny dla $z_0:|z_0|=R
\Rightarrow \sum a_nz^n$ jest zbieżny w $B(z_0,\delta)\cap K$ dla dostatecznie małego $\delta\newline
$Tw.Weierstrassa+Stone'a Weierstrassa:\newline
Stw. Dowolna funkcja $f\in C([a,b])$ jest jednostajną granicą funkcji ciągłych kawałkami liniowych\newline
Tw. Weierstrassa: Dowolna $f\in C([a,b])$ jest jednostajną granicą wielomianów\newline
Tw. Diniego: Niech $f_n\in C([a,b])\rightarrow g, \forall_{x\in[a,b]}f_n(x)$ jest monotoniczny $\Rightarrow$ zbieżność jest jednostajna na [a,b]\newline
Def. $K\subset\mathbb{R}$ jest zbiorem zwartym jeżeli z każdego ciągu $x_n\in K$ można wybrać podciąg zbieżny\newline
Tw.(B-W): odcinek [a,b] jest zwarty\newline
C(K)-przestrzeń funkcji ciągłych na K -algebra\newline
$A\subset C(K)$ jest podalgebrą jeżeli jest podprzestrzenią wektorową oraz $f,g\in A\Rightarrow f\cdot g\in A\newline
$Jeżeli $f\in C(K)$ są o wartościach w $\mathbb(C)$, to mówimy że podalgebra A jest symetryczna jeżeli $f\in A\Rightarrow \overline{f}\in A$\newline
Mówimy, że podalgebra rozdziela punkty zbiory=u K jeśli $\forall_{x_1\neq x_2\in K}\exists f\in A : f(x_1)\neq f(x_2)$\newline
Tw.(Stone'a Weierstrassa(1)) Niech K będzie zbiorem zwartym, $A\subset C(K)$ będzie symetryczną rozdzielającą punkty podalgebrą, taką,
że $1(\text{funkcja stała})\in A \Rightarrow\forall_{f\in C(K)}\exists_{g_n\in A}:g_n\rightrightarrows f, \overline{A}=C(K)\newline
$Def. $K\subset \mathbb{R}^n$ jest zbiorem zwartym $\Leftrightarrow$ z każdego ciągu $x_n\in K$ można wybrać podciąg zbieżny do pewnego $x_0\in K$\newline
Stw. Zbiór K jest zwarty $\Leftrightarrow$ z każdego otwartego pokrycia zbioru K można wybrać podpokrycie skończone 
$K\subset \bigcup\limits_{\alpha}U_{\alpha}$\newline
Def. Funkcję $f:I\rightarrow\mathbb{R}$ nazywamy wypukłą(wklęsłą), gdy $\forall_{x,y\in I}$ i $\forall_{\theta\in[0,1]}$
$f(\theta x+(1-\theta)y)\le\theta f(x)+(1-\theta)f(y)$ $(\ge)$\newline
Stw. $f\in C^1((a,b))$ jest wypukła $\Leftrightarrow f'$ jest rosnąca\newline
Stw. $f\in C^2((a,b))$ jest wypukła $\Leftrightarrow f''(x)\ge 0 \forall_{x\in (a,b)}$\newline
Stw.(Nierówność Jensena); f wypukła na [a,b]$\Rightarrow$ dla dowolnego skończonego zbioru $\{x1,..,x_n\}\subset[a,b],
0\le\theta_i\le 1:\sum\limits_{i=1}^{n}\theta_i=1$ $f(\sum\theta_ix_i)\le\sum\theta_if(x_i)\newline$
Funkcje Analityczne:\newline
Niech $f$ będzie określona w otwartym otoczniu $x_0$\newline
Def. Funkcja $f$ jest analityczna w punkcie $x_0$ gdy $f(x)=\sum a_n(x-x_0)^n$ dla $x-x_0<r_0>0$\newline
Def. Funkcja $f$ jest analityczna na zbiorze otwartym $A$, gdy f jest analityczna w każdym punkcie $x_0\in A$\newline
Tw. Niech $f(x)=a_nx^n$ dla $|x|<R$ $R>0$ oraz $|x_0|<R\Rightarrow f(x)=\sum\frac{f^{(n)}(x_0)}{n!}(x-x_0)^n$ dla $|x-x_0|<R-|x_0|$\newline
Lemat $a_{n,m}\in\mathbb{R}$ Jeżeli $\sum\limits_{n,m=0}^{\infty}|a_{n,m}|<\infty$ (sumowanie w dowolnej kolejności) to dla dowolnej bijekcji 
$\sigma\in \mathbb{N\rightarrow N\times N}$ $\sum\limits_{m=0}^{\infty}\sum\limits_{n=0}^{\infty}a_{n,m}=\sum\limits_{k=0}^{\infty}a_{\sigma(k)}$\newline
Tw.(Jednoznaczność przedłużenia): Niech $f,g$ analityczne na $(a,b)$ Jeśli ciąg $x_n\in(a,b)$ $x_n\rightarrow c\in(a,b)$ $x_n\neq c$ $\forall_n$,
oraz $f(x_n)=g(x_n)$, to $f=g$ na $(a,b)$\newline
Stw. f,g analityczne w $x_0$:\newline
$f\pm g,f\cdot g$ są analityczne w $x_0$, dla $g(x_0)\neq 0$ $\frac{f}{g}$ jest analityczna w $x_0$\newline
Stw. Niech $f$ analityczna w punkcie $x_0$, $g$ analityczna w punkcie $y_0=f(x_0)\Rightarrow g(f(x))$ jest analityczna w $x_0$\newline
Tw.(Analityczność funkcji odwrotnej) Niech f będzie analityczna w $x_0$ oraz $f'(x_0)\neq 0\Rightarrow\exists_{\varepsilon_0>0}:$
$f:(x_0-\varepsilon_0,x_0+\varepsilon_0)_{\longrightarrow}^{^{1-1}}\mathbb{R}$ i $f^{-1}$ jest analityczna w punkcie $y_0=f(x_0)$\newline\newline
Całki nieoznaczone:\newline
Niech $I$ będzie przedziałem otwartym $(a,b)$ lub $(a,+\infty)$ lub $(-\infty,b)$, $f:I\rightarrow\mathbb{R}$\newline
Def. funkcję $F:I\rightarrow\mathbb{R}$ nazywamy funkcją pierwotną funkcji f gdy $\forall_{x\in I}F'(x)=f(x)$\newline
Stw. Jeżeli $F_1,F_2$ są funkcjami pierwotnymi f, to $\exists_{C\in\mathbb{R}}:F_1-F_2=C$\newline
$F(x)+c$ -ogólna postać funkcji pierwotnej\quad Oznaczenie: F=$\int f(x)dx$\newline
Stw. Niech f będzie funkcją analityczną na $I\Rightarrow$ Istnieje funkcja pierwotna $F:I\rightarrow\mathbb{R}$ t.że $F'=f$ i jest analityczna\newline
Tw. Niech $f\in C(I)\Rightarrow$ Istnieje funkcja pierwotna $F:I\rightarrow\mathbb{R}$ t.że $F'=f$\newline
Stw. liniowość całki: $a,b\in\mathbb{R}$ $f,g:I\rightarrow\mathbb{R}\Rightarrow\int(af+bg)=a\int f+b\int g$\newline
Stw. Całkowanie przez części: Niech $f,g\in C^1\Rightarrow \int f\cdot g'=f\cdot g-\int f'\cdot g$\newline
Stw. Całkowanie przez podstawianie: $g$-ciągła,$f\in C^1\Rightarrow\int g(f(x))f'(x)dx=\int g(y)dy$ , gdzie $y=f(x)$\newline
Całka z funkcji wymiernej: $f(x)=\frac{P(x)}{Q(x)}$ $P,Q$- wielomiany $\rightarrow$ $R(x)+\frac{P_2(x)}{Q(x)}$ rozkład na czynniki proste\newline
Tw. Niech $P,Q$ będą rzeczywistymi wielomianami, $\neg(Q\equiv0)$ Wtedy 
$\frac{P}{Q}=\widetilde{P}+\sum\frac{A_j}{(x-a_j)^{k_j}}+\sum\frac{B_ix+C_i}{(x^2+b_ix+c_i)^{l_i}}$\newline
Całki z funkcji wymiernych od funkcji trygonometrycznych: $\int R(\sin{x},\cos{x})dx$- 
podstawienie $t=\tg{\frac{x}{2}}\rightarrow \int R(\frac{2t}{1+t^2},\frac{1-t^2}{1+t^2})\frac{2}{1+t^2}dt$\newline
Tw.(Jednostajne kryterium Dirichleta): Niech $|\sum\limits_{k=1}^{n}b_k(x)|\le M \forall_{n,x}$ orazniech funkcje $a_n(x)$ tworzą ciąg 
$\underline{\text{monotoniczny}}$, $a_n\rightrightarrows 0\Rightarrow$ szereg $\sum a_n(x)b_n(x)$ jest sbieżny jednostajnie\newline
Tw.(Jednostajne kryterium Abela): Niech szereg $\sum b_n(x)$ jest zbieżny jednostajnie a funkcje $a_n(x)$ tworzą ciąg monotoniczny ($\forall_x$)
i wspólnie ograniczony $|a_n(x)|\le M \forall_{n,x}\Rightarrow$ szereg $\sum a_n(x)b_n(x)$ jest zbieżny jednostajnie\newline\newline
Całka Riemanna:\newline
Def. Podziałem odcinka $[a,b]\subset\mathbb{R}$ nazywamy każdy skończony podzbiór $p\subset[a,b]$ zawierający końce odcinka, $p=\{a=x_0<x_1<\cdots<x_n=b\}$\newline
Def. Podział $\widetilde{p}$ jest zagęszczeniem podziału $p$ jeżeli $p\subset \widetilde{p}$\newline
Niech $f:[a,b]\rightarrow\mathbb{R}$ będzie funkcją ogranicczoną (tzn. $\exists$ liczby $m,M$ t.że $m\le f(x)\le M$)\newline
$I_k=[x_{k-1},x_k], m_j=\inf\limits_{x\in I_j}f(x),M_j=\sup\limits_{x\in I_j}f(x)$\newline
$\underline{S}(p,f)=\sum\limits_{j=1}^{n}m_j|I_j|$ -suma dolna $|$ $\overline{S}(p,f)=\sum\limits_{j=1}^{n}M_j|I_j|$ -suma górna\newline
Lemat: Niech f będzie funkcją ograniczoną na $[a,b]$, jeżeli podział $\widetilde{p}$ jest zagęszczeniem podziału $p$, to
$\underline{S}(p,f)\le\underline{S}(\widetilde{p},f)\le\overline{S}(\widetilde{p},f)\le\overline{S}(p,f)$\newline
Wn: Niech $p,\widetilde{p}$ będą dowolnymi podziałami odcinka $[a,b]$, to $\underline{S}(p,f)\le\overline{S}(\widetilde{p},f)$\newline
Oznaczamy $P_{[a,b]}=$zbiór podziałów odcinka $[a,b]$\newline
$\{\overline{S}(p,f):p\in P_{[a,b]}\}$ -ograniczony z dołu $|$ $\{\underline{S}(p,f):p\in P_{[a,b]}\}$ -ograniczony z góry\newline
Def. $\underline{\text{Całką}}$ górną (dolną) funkcji $f$ na $[a,b]$ nazywamy liczbę:
$\overline{\int\limits_{a}^{b}}f(x)dx:=\inf\limits_{p\in P_{[a,b]}}\overline{S}(p,f)$
$\left(\underline{\int\limits_{a}^{b}}f(x)dx:=\sup\limits_{p\in P_{[a,b]}}\underline{S}(p,f)\right)$
$\overline{\int\limits_{a}^{b}}f(x)dx\ge\underline{\int\limits_{a}^{b}}f(x)dx$\newline
Def. Funkcję ograniczoną $f:[a,b]\rightarrow\mathbb{R}$ nazywamy całkowalną w sensie Riemanna na $[a,b]$ jeżeli
$\overline{\int\limits_{a}^{b}}f(x)dx=\underline{\int\limits_{a}^{b}}f(x)dx$\newline
$\int\limits_{a}^{b}f(x)dx:=\overline{\int\limits_{a}^{b}}f(x)dx=\underline{\int\limits_{a}^{b}}f(x)dx$\newline
Zbiór dunkcji całkowalnych w sensie Riemanna na $[a,b]$ oznaczamy $R[a,b]$\newline
Lemat: Niech $f,g$ będą ograniczone na odcinku $[a,b]$. Wtedy:\newline
$\overline{\int\limits_{a}^{b}}(f(x)+g(x))dx\le\overline{\int\limits_{a}^{b}}f(x)dx+\overline{\int\limits_{a}^{b}}g(x)dx$ $|$
$\underline{\int\limits_{a}^{b}}(f(x)+g(x))dx\ge\underline{\int\limits_{a}^{b}}f(x)dx+\underline{\int\limits_{a}^{b}}g(x)dx$\newline
Jeżeli $\lambda>0$, to $\overline{\int\limits_{a}^{b}}(\lambda f(x))dx=\lambda\overline{\int\limits_{a}^{b}}f(x)dx$ $|$ 
$\underline{\int\limits_{a}^{b}}(\lambda f(x))dx=\lambda\underline{\int\limits_{a}^{b}}f(x)dx$\newline
$\overline{\int\limits_{a}^{b}}(-f(x))dx=-\underline{\int\limits_{a}^{b}}f(x)dx$ $|$
$\underline{\int\limits_{a}^{b}}(-f(x))dx=-\overline{\int\limits_{a}^{b}}f(x)dx$\newline
Tw. Funkcja ciągła na $[a,b]$ jest całkowalna w sensie Riemanna.\newline
Stw. Funkcja monotoniczna na $[a,b]$ jest całkowalna w sensie Riemanna.\newline
$f,g\in R[a,b]$ $\int\limits_{a}^{b}(\lambda_1f(x)+\lambda_2g(x))dx=\lambda_1\int\limits_{a}^{b}f(x)dx+\lambda_2\int\limits_{a}^{b}g(x)dx$\newline
Stw. Własności całki $R$:\newline
$f(x)=c\Rightarrow f\in R[a,b],\int\limits_{a}^{b}c=(b-a)c$ $ $ $|$ $ $ liniowość$\uparrow$\newline
$f,g\in R[a,b], f\le g\Rightarrow\int\limits_{a}^{b}f\le\int\limits_{a}^{b}g$ $ $ $|$ $ $
$f\in R[a,b]\wedge f\in R[b,c]\Rightarrow f\in r[a,c],\int\limits_{a}^{c}f=\int\limits_{a}^{b}+\int\limits_{b}^{c}$\newline
$f\in R[a,b]$ oraz $[c,d]\subset[a,b]\Rightarrow f\in R[c,d]$\newline
$f\in R[a,b] m\le |f(x)|\le M dla x\in[a,b]$ oraz niech $\varphi:[m,M]\rightarrow\mathbb{R}$ będzie ciągła $\Rightarrow\varphi\circ f\in R[a,b]$\newline
$f,g\in R[a,b]\Rightarrow f\cdot g\in R[a,b]$ $ $ $|$ $ $
$f\in R[a,b]\Rightarrow |f|\in R[a,b],\left|\int\limits_{a}^{b}f\right|\le\int\limits_{a}^{b}|f|$\newline
$f\in R[a,b]$, to funkcja $F(y)=\int\limits_{a}^{y}f(x)dx$ jest Lipshitzowsko ciągła, tzn. $\exists L$ t.że $|F(y_1)-F(y_2)|\le L|y_1-y_2|$\newline
Tw.(Podstawowe rachunku całkowego):\newline
Jeżeli $f\in C[a,b]$, to funkcja $F(x)=\int\limits_{a}^{x}f(t)dt$ $x\in [a,b]$ jest różniczkowalna na $[a,b]$ oraz $F'=f$\newline
Wn: Jeżeli $f\in C([a,b])$, to $\exists F:\mathbb{R}\rightarrow\mathbb{R},F\in C^1(\mathbb{R}):F'(x)=f(x)\forall_{x\in[a,b]}$
$\widetilde{f}=\left\{\begin{array}{cc}f(a)&\text{dla }x<a\\f&\text{na }[a,b]\\f(b)& \text{dla }x>b\end{array}\right. F(x)=\int\limits_{a}^{x}\widetilde{f}(t)dt$\newline
Wn: Jeżeli $f\in C([a,b]), F\in C([a,b])$, różniczkowalna na $(a,b)$, t.że $F'(x)=f(x)$ dla $x\in(a,b)\Rightarrow\int\limits_{a}^{b}f=F(b)-F(a)$\newline
Def. Jeżeli $p\in P_{[a,b]}$, to $d(p)=\max|I_j|$- długość podziału\newline
Tw: Niech $f\in R[a,b]$ oraz $p_n$ będzie ciągiem podziałów odcinka $[a,b]$ takim, żę $d(p_n)^{--\longrightarrow}_{n\rightarrow\infty}0\Rightarrow$
$\overline{S}(p_n,f)^{--\longrightarrow}_{n\rightarrow\infty}\int\limits_{a}^{b}f$,
oraz $\underline{S}(p_n,f)^{--\longrightarrow}_{n\rightarrow\infty}\int\limits_{a}^{b}f$\newline
Stw. Jeżeli $f:[a,b]\rightarrow\mathbb{R}$ jest ograniczona i ma skończenie wiele punktów nieciągłości, to $f\in R[a,b]$\newline
Twierdzenia o wartości średniej:\newline
I $ $ Niech $f\in C([a,b])$ oraz $g\in R[a,b]$ oraz $g$ ma stały znak$\Rightarrow\exists_{c\in(a,b)}:\int\limits_{a}^{b}f\cdot g=f(c)\int\limits_{a}^{b}g$\newline
II Niech $f\in R[a,b]$ oraz $g$monotoniczna $\Rightarrow\exists_{c\in[a,b]}:\int\limits_{a}^{b}f\cdot g=g(a)\int\limits_{a}^{c}f+g(b)\int\limits_{c}^{b}f$\newline
Tw.(Reszta Taylora w postaci całkowej): Niech $f\in C^{n+1}((x_0-\delta,x_0+\delta))\Rightarrow$
n-ta reszta we wzorze Taylora \newline$r_n=\frac{1}{n!}\int\limits_{0}^{h}(h-t)^nf^{(n+1)}(x_0+t)dt$\newline
Wn-Stw: Niech $f:(a,b)\rightarrow\mathbb{R},f\in C^{\infty}$ t.że $\forall_{x\in(a,b),n\in \mathbb{N}\cup\{0\}}f^{(n)}(x)\ge 0\Rightarrow f$ jest analityczna\newline
Tw: Niech $f_n\in R[a,b]$ oraz $f_n\rightrightarrows f$ (na $[a,b]$)$\Rightarrow f\in R[a,b]$ oraz
$\lim\limits_{n\rightarrow\infty}\int\limits_{a}^{b}f_n(t)dt=\int\limits_{a}^{b}f(t)dt$\newline
Kryterium całkowalności w sensie Riemanna:\newline
Def. Zbiór $A\subset\mathbb{R}$ jest $\underline{\text{miary zero}}$ gdy 
$\forall_{\varepsilon>0}$ $A\subset\bigcup\limits_{j=1}^{\infty}I_j$, gdzie $I_j$-odcinek, oraz $\sum\limits_{j=1}^{\infty}|I_j|<\varepsilon$\newline
Lemat: Każdy zbiór przeliczalny jest miary zero $|$ Jeżeli $A$ jest miary zero i $B\subset A$ to $B$ jest miary zero\newline
Jeżeli $A_k,k=1,2,\cdots$ są zbiorami miary zero, to suma $A=\bigcup\limits_{k=1}^{\infty}A_k$ jest miary zero\newline
Lemat: Niech $f:D\rightarrow\mathbb{R}, x_0\in D$ funkcja $f$ jest ciągła w $x_0\Leftrightarrow
\inf\limits_{\delta>0}(\sup\limits_{|x-x_0|<\delta,|y-x_0|<\delta}|f(x)-f(y)|)=0$\newline
Wn: Zbiór punktów nieciągłości $f$ $NC_f=\{x:\inf\limits_{\delta>0}(\sup(\cdots))>0\}
=\bigcup\limits_{k=1}^{\infty}\{x:\inf\limits_{\delta>0}(\sup(\cdots))\ge\frac{1}{k}\}$\newline
Tw.(Kryterium całkowalności): Niech $f:[a,b]\rightarrow\mathbb{R}$ będzie funkcją ograniczoną.\newline
Wtedy $f\in R[a,b]\Leftrightarrow$ Zbiór punktów nieciągłości $f$ jest miary zero\newline
Tw(Wzór Wallisa): $\lim\limits_{n\rightarrow\infty}\frac{((2n)!!)^2}{(2n-1)!(2n+1)!}=\prod\limits_{n=1}^{\infty}\frac{(2n)^2}{(2n-1)(2n+1)}=\frac{\pi}{2}$\newline
Tw: $\pi\notin\mathbb{Q}$\newline
Tw.(Wzór Stirlinga): $\sqrt{2\pi}<\frac{n!e^n}{n^n\sqrt{n}}<\sqrt{2\pi}\cdot e^{\frac{1}{12n}}$ $(n!\approx\frac{\sqrt{2\pi n}n^n}{e^n})$\newline
Tw. Liczba $e$ nie jest algebraiczna\newline
Tw.(Weierstrassa(wersja 2)): $f\in C([0,1])\Rightarrow\exists$ciąg wielomianów $p_n$ t.że $p_n\rightrightarrows f$ (na $[0,1]$)\newline
$\sin{x}=x\prod\limits_{n=1}^{\infty}(1-\frac{x^2}{n^2\pi^2})$\newline\newline
Całki Niewłaściwe:\newline
$f:[a,b]\rightarrow\mathbb{R}$ ograniczona $\overline{S},\underline{S}\cdots$\newline
Załóżmy, ze $f:[a,b]\rightarrow\mathbb{R}$ nieograniczona, $\sup\limits_{[a,b]}f=+\infty\Rightarrow \overline{S}(p,f)=+\infty$\newline
Niech $f:[0,\infty)\rightarrow\mathbb{R}$ będzie funkcją całkowalną w sensie $R$ na każdym przedzile $[a,$R$]$, jeżeli istnieje granica
$\lim\limits_{\text{R}\rightarrow\infty}\int\limits_{a}^{\text{R}}f(x)dx$, to nazywamy ją całką niewłaściwą z funkcji $f$ na odcinku $[a,\infty]$
ozn. $\int\limits_{a}^{\infty}f(x)dx$ $ $ (analogicznie $\int\limits_{-\infty}^{a}f(x)dx$)\newline
Def. jeżeli granica jest skończona to całkę nazywamy zbieżną\newline
Stw.(Wzór Cauchy'ego): $f:[a,\infty)\rightarrow\mathbb{R},f\in R[a,\text{R}]$ jest zbieżny $\Leftrightarrow\forall_{\varepsilon}\exists_{\text{R}_{\varepsilon}}
forall_{\text{R}_1,\text{R}_2>\text{R}_{\varepsilon}}|\int\limits_{a}^{\text{R}_1}f-\int\limits_{a}^{\text{R}_2}f|
=|\int\limits_{\text{R}_1}^{\text{R}_2}f|<\varepsilon$\newline
Tw.(Kryterium całkowe zbieżności szeregów): $f:[1,\infty)\rightarrow\mathbb{R},f\ge0,f$ malejąca $\Rightarrow\left(\int\limits_{1}^{\infty}f(x)dx<\infty
\Leftrightarrow\sum\limits_{n=1}^{\infty}f(n)<\infty\right)$\newline
Tw.(Kryterium Abela!): Niech $g:[a,\infty)\rightarrow\mathbb{R}$ będzie \underline{monotoniczna} i \underline{ograniczona}, $\int\limits_{a}^{\infty}f$
jest zbieżna $\Rightarrow\int\limits_{a}^{\infty}(f\cdot g)$ jest zbieżna\newline
Tw.(Kryterium Dirichleta!): Niech $g:[a,\infty)\rightarrow\mathbb{R}$ będzie \underline{monotoniczna}, $\lim\limits_{x\rightarrow\infty}g(x)=0,
\exists_M\forall_x|\int\limits_{a}^{x}f(t)dt|\le M\Rightarrow\int\limits_{a}^{\infty}(f\cdot g)$ jest zbieżna\newline
Tw. Niech $f_n:[a,\infty)\rightarrow\mathbb{R},f_n\in R[a,\text{R}]\forall_{\text{R}>n} f_n\rightrightarrows f$ na każdym odcinku [$a$,R]
(zbieżność niemal jednostajna), istnieje $|f_n(x)|\le F(x)\forall_{x\in[a,\infty)}$ oraz $\int\limits_{a}^{\infty}F<\infty
\Rightarrow\int\limits_{a}^{\infty}f$ jest zbieżna oraz $\lim\limits_{n\rightarrow\infty}\int\limits_{a}^{\infty}f_n=\int\limits_{a}^{\infty}f$\newline
Def. Jeżeli całka $\int\limits_{a}^{b}|f|$ jest zbieżna to całkę $\int\limits_{a}^{b}f$ nazywamy zbieżną bezwzględnie\newline
Lem.(Riemanna v 1.0): $f\in C([a,b])\Rightarrow\lim\limits_{n\rightarrow\infty}\int\limits_{a}^{b}f(x)\sin(nx)dx=0
=\lim\limits_{n\rightarrow\infty}\int\limits_{a}^{b}f(x)\cos(nx)dx$\newline
Stw: $f\in C^1([a,b])$ $ $ $\int\limits_{a}^{b}f(x)\sin(nx)dx
=[-f(x)\frac{\cos{nx}}{n}]_{a}^{b}+\frac{1}{n}\int\limits_{a}^{b}f'\cos{nx}_{--\longrightarrow}^{n\rightarrow\infty}0$\newline
Lem.(Riemanna v 2.0): $f\in R[a,b]\Rightarrow\lim\limits_{n\rightarrow\infty}\int\limits_{a}^{b}f(x)\sin(nx)dx=0
=\lim\limits_{n\rightarrow\infty}\int\limits_{a}^{b}f(x)\cos(nx)dx$\newline
Lem: $f\in R[a,b]\Rightarrow\forall_{\varepsilon}\exists g\in C([a,b])$ t.że $\int\limits_{a}^{b}|f-g|<\varepsilon$\newline
$\cos{x}=\frac{e^{ix}+e^{-ix}}{2}$ $ $ $|$ $ $ $\int\limits_{0}^{\infty}e^{-x^2}dx=\frac{\sqrt{\pi}}{2}$\newline
Funkcje 2 zmiennych: $f:Dx\times Dy\rightarrow\mathbb{R}$  $f(x,y)$ $y_0\in Dy$ jest punktem\newline
Jeżeli $f(x,y)_{--\longrightarrow}^{y\rightarrow y_0}f_0(x) \forall_{x\in Dx},\forall_{\varepsilon>0}\exists_{\delta}$ (nie zależy od x) $|y-y_0|<\delta$
$\Rightarrow\forall_{x\in Dx}|f(x,y)-f_0(x)|<\varepsilon$, to mówimy, że $f(x,y)$ dąży do funkcji $f_0$ jednostajnie względem x\newline
Stw: $f(x,y)_{--\longrightarrow}^{y\rightarrow y_0}f_0$ jednostajnie względem x jeżeli $\forall$ ciągu
$y_n\rightarrow y_0(y_n\in Dy)$ $f(x,y_n)\rightrightarrows f_0(x)$\newline
Def. $f:[a,b]\times[c,d]\rightarrow\mathbb{R}$ jest ciągła w punkcie $(x_0,y_0)$ gdy $\forall_{\varepsilon}\exists_{\delta}|x-x_0|<\delta,|y-y_0|<\delta
\Rightarrow|f(x,y)-f(x_0,y_0)|<\varepsilon$\newline
Stw. Jeśli $f$ jest ciągła na $[a,b]\times[c,d]$  to jest jednostajnie ciągła tzn.$\forall_{\varepsilon}\exists_{\delta}|x_1-x_2|<\delta,|y_1-y_2|<\delta
\Rightarrow|f(x_1,y_1)-f(x_2,y_2)|<\varepsilon$\newline
Tw. Niech $f:[a,b]\times[c,d]\rightarrow\mathbb{R}$ będzie ciągła względem x $\forall_{y\in(c,d)}, (Dyf)(x,y)$ jest ciągła na $[a,b]\times[c,d]$ wtedy
$I'(y)=\int\limits_{a}^{b}(Dyf)(x,y)dx$\newline
Warunek Cauchy'ego $\forall_{\varepsilon>0}\exists_{\text{R}_0}\forall_{\text{R},\widetilde{\text{R}}>\text{R}_0}
|\int\limits_{\text{R}}^{\widetilde{\text{R}}}f(x,y)|<\varepsilon$ $ $ $\forall_y$\newline
Kryterium Weierstrassa: Jeżeli $f(x,y)$ jest całkowalna względem $x$ na [$a$,R] $\forall_{\text{R}}$ oraz $\exists_{\varphi(x)}$ całkowalna na $[a,\infty)$
$\Rightarrow$ całka $\int\limits_{a}^{\infty}f(x,y)$ jest zbieżna jednostajnie\newline
Kryterium Abela: Jeżeli $f(x,y)$ jest całkowalna względem $x$ na [$a$,R] $\forall_{\text{R}}$, $g(x,y)$ monotoniczna względem $x$
oraz $\int\limits_{a}^{\infty}f(x,y)$ jest zbieżna jednostajnie względem $y$ oraz g jest jednostajnie ograniczona, tzn. $|g(x,y)|\le L$ $\forall_{x,y}$\newline
$\Rightarrow\int\limits_{a}^{\infty}f(x,y)g(x,y)dx$ jest zbieżna\newline
Kryterium Dirichleta: Jeżeli $f(x,y)$ jest całkowalna względem $x$ na [$a$,R] $\forall_{\text{R}}$, $g(x,y)$ monotoniczna względem $x$
oraz $\int\limits_{a}^{\text{R}}f(x,y)$ są ograniczone jako funkcje R i $y$ tzn. $|\int\limits_{a}^{\text{R}}f(x,y)|\le M$ $\forall_{\text{R}\ge a}\forall_{y\in D_y}$
$g(x,y)\rightrightarrows 0$ $(x\rightarrow\infty)$ (jednostajnie względem y)\newline
$\Rightarrow\int\limits_{a}^{\infty}f(x,y)g(x,y)dx$ jest zbieżna\newline
Stw: Jeżeli $f(x,y)$ jest całkowalna na [$a$,R] $\forall_{\text{R}}$ oraz $f(x,y)_{--\longrightarrow}^{y\rightarrow y_0}f_0(x)$,
jednostajnie względem $x$ na każdym [$a$,R] oraz $\int\limits_{a}^{\infty}f(x,y)$ jest zbieżna jednostajnie względem $y$
$\Rightarrow\lim\limits_{y\rightarrow y_0}\int\limits_{a}^{\infty}f(x,y)dx=\int\limits_{a}^{\infty}f_0(x)dx$\newline
Stw: Niech $f(x,y)$ będzie ciągła $[a,\infty)\times[c,d]$ jeżeli $\int\limits_{a}^{\infty}f(x,y)$ jest zbieżna jednostajnie względem $y\in[c,d]
\Rightarrow I(y)=\int\limits_{a}^{\infty}f(x,y)$ jest ciągła na $[c,d]$\newline
$B(a,b)=\int\limits_{0}^{1}x^{a-1}(1-x)^{b-1}dx$ dla $(a,b>0)$ - funkcja beta\newline
Własności: $B(a,b)=B(b,a)$ $ $ $|$ $ $ $B(a,b)=\frac{b-1}{a+b-1}\cdot B(a,b-1)$ dla $(a>0,b>1)\newline
B(a,n)=\frac{(n-1)!}{a(a+1)\cdots(a+n-1)}=\frac{(n-1)!}{a^{\overline{n}}}$ dla $n\in \mathbb{N}$ $ $ $|$ $ $ $B(n,m)=\frac{(n-1)!(m-1)!}{(m+n-1)!}$ dla $m,n\in \mathbb{N}$\newline
$B(a,1-a)=\frac{\pi}{\sin{\pi a}}$ dla $a\in(0,1)$\newline
$\Gamma(a)=\int\limits_{0}^{\infty}x^{a-1}e^{-x}dx$ dla $a>0$ -funkcja gamma\newline
Stw. $\Gamma\in C^{\infty}((0,\infty))$\newline
Własności: $\Gamma(a+1)=a\cdot\Gamma(a)$ $ $ $|$ $ $ $\Gamma(a+n)=(a+n-1)\cdots(a+1)a\cdot\Gamma(a)=a^{\overline{n}}\Gamma(a)$\newline
$\Gamma(n+1)=n!$ $ $ $|$ $ $ $\Gamma(a)=\lim\limits_{n\rightarrow\infty}n^a\frac{(n-1)!}{a(a+1)\cdots(a+n-1)}=\lim\limits_{n\rightarrow\infty}n^a\frac{(n-1)!}{a^{\overline{n}}}\newline
B(a,b)=\frac{\Gamma(a)\cdot\Gamma(b)}{\Gamma(a+b)}$
\newpage
Def. (V-przestrzeń wektorowa nad $\mathbb{R}$($\mathbb{C}$)) norma na V - $\norma{ }:V\rightarrow\mathbb{R_+}:\norma{x}=0\Leftrightarrow x=0,
\norma{\lambda x}=|\lambda|\cdot\norma{x},\newline\norma{x+y}\le\norma{x}+\norma{y}\quad$ metryka $\rho(x,y)=\norma{x-y}$,
jeżeli $\dim V<\infty$, to wszystkie normy są równoważne\newline
$B_R(x)=\{y:\norma{x-y}<R\}$ $(\overline{B}_R(x)=\{y:\norma{x-y}\le R\})$ kula otwarta (domknięta) w przestrzeni\newline
zbiór otwarty $U$: $\emptyset$ albo: $\forall p\in U$ $\exists R>0:B_R(p)\in U$\newline
zbiór domknięty = uzupełnienie zbioru otwartego\newline
($X\subset V$) zb. otwarty w $X$ := zbiór postaci $U\cap X$, gdzie $U$ otwarty w $V$\newline
Topologia = rodzina wszystkich zbiorów otwartych\newline
normy równoważne $\norma{}_1\sim\norma{}_2\Leftrightarrow\exists c,C>0:c\norma{}_1\le\norma{}_2\le C\norma{}_1\Leftrightarrow
\forall$ kula otwarta (o dodatnim promieniu) wokół dowolnego p w sensie $\norma{}_1$ zawiera kulę otwrtoą o dodatnim promieniu wokół p w sensie $\norma{}_2$
i na odwrót (mamy również określoną tą samą topologię)\newline
Wn. $V=$przestrzeń wszystkich wielomianów stopnia$\le$d (jednej zmiennej t): $\exists c\forall w\in V \sup\limits_{t\in[-2,2]}|w(t)|\le c\sup\limits_{[t\in[-1,1]]}|w(t)|$\newline
$(V,\norma{})$ $x_\nu\rightarrow x\Leftrightarrow\norma{x_\nu-x}\rightarrow0\Leftrightarrow\forall$otoczenia $x$ zbiór $\{\nu:x_\nu\notin\text{otoczenie}\}$jest skończony\newline
$e_1,e_2,...,e_n$ - baza $\mathbb{R}^n$ $x=\sum x_ie_i$ $x_\nu\rightarrow x\Leftrightarrow\forall i$ $x_{\nu,i}\rightarrow x_i$\newline
Warunek Cauchy'ego: $\norma{x_\nu-x_\mu}^{--\longrightarrow}_{\nu,\mu\rightarrow\infty}0\Leftrightarrow\forall\varepsilon\exists N\forall_{\nu,\mu>N}\norma{x_\nu-x_\mu}<\varepsilon$
\quad w $\mathbb{R}^n$ zb.C$\leftrightarrow$W.C\newline
C jest domknięty$\Leftrightarrow\forall x_\nu\in C \lim_{x_\nu}\in C$ (zbieżność w $\mathbb{R}^n$) $A\subset\mathbb{R}^n$ $\overline{A}=\{x:\exists x_\nu\in A$ $x_\nu\rightarrow x \}$\newline
punkt skupienia $A$: każdy taki punkt $x_0:\exists (x_n)\subset A:xn\rightarrow x_0, x_n\neq x_0$\newline
zbiór ograniczony- mieści się w jakiejś kuli\newline
Zwartość: $X\subset V$ jest zwarty$\Leftrightarrow$ każdy ciąg $x_\nu\in X$ zawiera podciąg zbieżny do pewnego punktu $\in X
\Leftrightarrow$ każde pokrycie otwarte X zawiera podpokrycie skończone (w $\mathbb{R}^n\Leftrightarrow$ domknięty i ograniczony)\newline
Spójność: $X$ jest spójny jeśli nie jest niespójny\newline
$X$ jest niespójny gdy $\exists X_1, X_2\neq\emptyset$ otwarte : $X=X_1\cup X_2,X_1\cap X_2=\emptyset$\newline
$A\subset \mathbb{R}^n , f:A\rightarrow\mathbb{R}^m$ $x_0$- punkt skupienia $A$ $f(x)^{--\longrightarrow}_{x\rightarrow x_0} y\in R^m$ - zbieżność w sensie Heinego, $H\Leftrightarrow C$\newline
punkt skupienia $f$ w $x_0$: zbiór wszystkich granic ciągów postaci $f(x_\nu)$ $A\ni x_\nu\rightarrow x_0$ $x\nu\neq x_0$\newline
ciągłośc $f$ w $x_0\in A$: jeśli $x_0\in A$ jest punktem skupienia, to ciągłość ok, jeśli $f(x_0)=\lim f(x)\Leftrightarrow\forall(x_n)\subset A: x_n\rightarrow x_0$ $f(x_n)\rightarrow f(x_0)$
$\Leftrightarrow\forall$otoczenie $U\ni f(x_0)$ w $\mathbb{R}^m$ $f^{-1}(U)$ zawiera otoczenie $x_0$ w $A$\newline

\end{document}




