\documentclass{article}
\usepackage[utf8]{inputenc}
\usepackage{polski}
\usepackage{amsmath}
\usepackage{anysize}
\usepackage{amssymb}
\begin{document}

Wiktor Zuba 320501 grupa 3
\newline

Zadanie 6.
\newline
\newline
b) $\varphi(t)=1+\sin{(t)}$ nie jest funkcją charakterystyczną ponieważ $|\varphi(\frac{\pi}{2})|=2>1$, a więc nie spełnia warunku koniecznego bycia funkcją charakterystyczną.\newline\newline
c) $\varphi(t)=\cos{(t^2)}$ nie jest funkcją charakterystyczną ponieważ $(\varphi)"(0)=(\cos{(t^2)})"(0)=(-2t\sin{(t^2)})'(0)=(-2\sin{(t^2)}-4t^2\cos{(t^2)})(0)=0$,
jednak $\varphi(t)$ nie jest funkcją stałą, co przeczy byciu funkcją charakterystyczną na mocy lematu z ćwiczeń.\newline
\end{document}