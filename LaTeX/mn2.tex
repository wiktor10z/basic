\documentclass{article}
\usepackage[utf8]{inputenc}
\usepackage{polski}
\usepackage{amsmath}
\usepackage{anysize}
\usepackage{amssymb}
\begin{document}
 
Wiktor Zuba 320501 grupa 1
\newline

Zadanie 2.
\newline
$
cond_{f}x=\sup\limits_{\lVert \overline{x}-x\rVert\le\varepsilon\lVert x\rVert}
\left(\frac{\lVert f(\overline{x})-f(x)\rVert}{\lVert f(x)\rVert}:\frac{\lVert \overline{x}-x\rVert}{\lVert x\rVert}\right)
\newline
cond_{a+b}a=\sup\limits_{\lVert \overline{a}-a\rVert\le\varepsilon\lVert a\rVert}
\left(\frac{\lVert \overline{a}+b-a-b\rVert}{\lVert a+b\rVert}:\frac{\lVert \overline{a}-a\rVert}{\lVert a\rVert}\right)=
\sup\limits_{\lVert \overline{a}-a\rVert\le\varepsilon\lVert a\rVert}
\left(\frac{\lVert \overline{a}-a\rVert}{\lVert a+b\rVert}:\frac{\lVert \overline{a}-a\rVert}{\lVert a\rVert}\right)=
\frac{\lVert a\rVert}{\lVert a+b\rVert}\newline
$
Czyli było by dobrze uwarunkowane dla dużego $a+b$, jednak dla $b\approx-a $ zadanie jest bardzo źle uwarunkowane\newline
$
cond_{a+b}b=\sup\limits_{\lVert \overline{b}-b\rVert\le\varepsilon\lVert b\rVert}
\left(\frac{\lVert a+\overline{b}-a-b\rVert}{\lVert a+b\rVert}:\frac{\lVert \overline{b}-b\rVert}{\lVert b\rVert}\right)=
\sup\limits_{\lVert \overline{b}-b\rVert\le\varepsilon\lVert b\rVert}
\left(\frac{\lVert \overline{b}-b\rVert}{\lVert a+b\rVert}:\frac{\lVert \overline{b}-b\rVert}{\lVert b\rVert}\right)=
\frac{\lVert b\rVert}{\lVert a+b\rVert}\newline
$
To samo co wyżej\newline
$
cond_{a-b}a=\sup\limits_{\lVert \overline{a}-a\rVert\le\varepsilon\lVert a\rVert}
\left(\frac{\lVert \overline{a}-b-a+b\rVert}{\lVert a-b\rVert}:\frac{\lVert \overline{a}-a\rVert}{\lVert a\rVert}\right)=
\sup\limits_{\lVert \overline{a}-a\rVert\le\varepsilon\lVert a\rVert}
\left(\frac{\lVert \overline{a}-a\rVert}{\lVert a-b\rVert}:\frac{\lVert \overline{a}-a\rVert}{\lVert a\rVert}\right)=
\frac{\lVert a\rVert}{\lVert a-b\rVert}\newline
$
Źle uwarunkowane dla $b\approx a$\newline
$
cond_{a-b}b=\sup\limits_{\lVert \overline{b}-b\rVert\le\varepsilon\lVert b\rVert}
\left(\frac{\lVert a-\overline{b}-a+b\rVert}{\lVert a-b\rVert}:\frac{\lVert \overline{b}-b\rVert}{\lVert b\rVert}\right)=
\sup\limits_{\lVert \overline{b}-b\rVert\le\varepsilon\lVert b\rVert}
\left(\frac{\lVert -\overline{b}+b\rVert}{\lVert a-b\rVert}:\frac{\lVert \overline{b}-b\rVert}{\lVert b\rVert}\right)=
\frac{\lVert b\rVert}{\lVert a-b\rVert}\newline
$
Jak wyżej\newline
$
cond_{a\cdot b}a=\sup\limits_{\lVert \overline{a}-a\rVert\le\varepsilon\lVert a\rVert}
\left(\frac{\lVert \overline{a}\cdot b-a\cdot b\rVert}{\lVert a\cdot b\rVert}:\frac{\lVert \overline{a}-a\rVert}{\lVert a\rVert}\right)=
\sup\limits_{\lVert \overline{a}-a\rVert\le\varepsilon\lVert a\rVert}
\left(\frac{\lVert \overline{a}-a\rVert\cdot\lVert b\rVert}{\lVert a\rVert\cdot\lVert b\rVert}:\frac{\lVert \overline{a}-a\rVert}{\lVert a\rVert}\right)
=1\newline
$
A więc zadanie jest bardzo dobrze uwarunkowane\newline
$
cond_{a\cdot b}b=\sup\limits_{\lVert \overline{b}-b\rVert\le\varepsilon\lVert b\rVert}
\left(\frac{\lVert a\cdot \overline{b}-a\cdot b\rVert}{\lVert a\cdot b\rVert}:\frac{\lVert \overline{b}-b\rVert}{\lVert b\rVert}\right)=
\sup\limits_{\lVert \overline{b}-b\rVert\le\varepsilon\lVert b\rVert}
\left(\frac{\lVert a\rVert\cdot\lVert \overline{b}-b\rVert}{\lVert a\rVert\cdot\lVert b\rVert}:\frac{\lVert \overline{b}-b\rVert}{\lVert b\rVert}\right)
=1\newline
$
Jak wyżej\newline
$
cond_{\frac{a}{b}}a=\sup\limits_{\lVert \overline{a}-a\rVert\le\varepsilon\lVert a\rVert}
\left(\frac{\lVert \frac{\overline{a}}{b}-\frac{a}{b}\rVert}{\lVert\frac{a}{b}\rVert}:\frac{\lVert \overline{a}-a\rVert}{\lVert a\rVert}\right)=
\sup\limits_{\lVert \overline{a}-a\rVert\le\varepsilon\lVert a\rVert}
\left(\frac{\lVert (\overline{a}-a)\cdot\frac{1}{b}\rVert}{\lVert a\cdot\frac{1}{b}\rVert}:\frac{\lVert \overline{a}-a\rVert}{\lVert a\rVert}\right)=
\sup\limits_{\lVert \overline{a}-a\rVert\le\varepsilon\lVert a\rVert}
\left(\frac{\frac{\lVert\overline{a}-a\rVert}{\lVert b\rVert}}{\frac{\lVert a\rVert}{\lVert b\rVert}}:\frac{\lVert \overline{a}-a\rVert}{\lVert a\rVert}\right)=1\newline
$
dla b niezerowego zadanie jest bardzo dobrze uwarunkowane\newline
$
cond_{\frac{a}{b}}b$ inną metodą (pochodnych)\newline$
cond_{\frac{a}{b}}b=\left|\frac{-a}{b^2}\cdot \frac{b}{\frac{a}{b}}\right|=
|-1|=1
$
Tak więc ten przypadek jest również bardzo dobrze uwarunkowany\newline
Uwaga na temat przejść norma iloczynu - iloczyn norm. Dla liczb rzeczywistych nie ma żadnego problemu, natomiast dla wektorów i iloczynu wektorowego
supremum wyraża się właśnie tym, że zaburzenie jest takie aby wektory mnożone były prostopadłe (dla iloczynu skalarnego równoległe), w przypadku
innych norm wektorowych też nie ma problemu wystarczy w przejściu wziąć "$\le$" zamiast "="


\end{document}