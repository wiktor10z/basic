\documentclass{article}
\usepackage[utf8]{inputenc}
\usepackage{polski}
\usepackage{amsmath}
\usepackage{anysize}
\usepackage{amssymb}
\begin{document}

Wiktor Zuba 320501\newline

Zadanie 1\newline

Algorytm bardzo zgrubny:

Startując od deterministycznego automatu z wartunkiem Mullera o n stanach
czytamy najpierw $n^2$ znaków - czyli $n$ bloków $1(00)^k)$. Następnie czytając po $n!$ bloków $1(00)^k$
zapamiętujemy stan do którego wchodzimy po przeczytaniu $1$,
jeśli po kilku takich krokach napotkamy wcześniej już zapamiętany stan, to znaleźliśmy cykl ruchów po automacie -
czytając słowo dalej będziemy się poruszać zawsze po dokładnie tych stanach które zostały odwiedzone pomiędzy miejscem w którym zapamiętaliśmy
stan i kiedy się on powtórzył(jest to skończona liczba ruchów) sprawdzamy czy stany odwiedzane w tej sekwencji tworzą
zbiór zawarty w rodzinie akceptującej automatu.\newline\newline

Dowód poprawności:\newline
a) Algorytm się zatrzyma po skończonej liczbie kroków - stanów automatu jest $n$, więc po przeczytaniu maksymalnie $n\cdot n!$
bloków musimy napotkać znów ten sam stan, więc przeczytamy maksymalnie $n+n\cdot n!$ bloków, czyli $(n+n\cdot n!)^2$ liter +
zapamiętywanie stanów z przejścia czytającego niewięcej niż $n\cdot n!$ bloków, o analogicznym koszcie co wcześniejsza część algorytmu.\newline
b) Będziemy nieskończenie wiele razy chodzić dokładnie po stanach ze znalezionej sekwencji:\newline
Po przeczytaniu $n^2$ liter czytając resztę słowa będziemy się już poruszać w stałym schemacie:
przeczytanie $1$, dojście po krawędziach o etykietach $0$ do cyklu złożonego z krawędzi o etykietach $0$ - przejście tego cyklu kilka razy
(conajmniej raz) i wyjście z cyklu przy przeczytaniu kolejnej litery $1$, co wynika z deterministyczności automatu i tego,
że musimy przeczytać więcej $0$ pod rząd niż liczba stanów automatu.
Jeśli po przeczytaniu $1$ trafiamy do stanu $x$ a następnie przeczytaniu $2l$ liter $0$ wychodzimy z cyklu krawędzią $1$ do stanu $y$,
to po przeczytniu $2l+k\cdot a$, (gdzie $a$ jest długością cyklu) również musielibyśmy trafić do tego samego stanu $y$ (z deterministyczności),
w szczególności po przeczytaniu $2l+2n!$ stanie się tak zawsze (nie musimy więc znać długości cyklu).
Jeśli po przeczytaniu dokładnie $k\cdot n!$ bloków trafimy do tego samego stanu po przeczytaniu $1$,
to przy przeczytaniu tego bloku przejdziemy po tej samej ścieżce, po tym samym cyklu (więcej razy) i wyjdziemy z niego tą samą krawędzią o etykiecie $1$,
do następnego stanu który znajduje się w dokładnie takiej samej sytuacji (powtórzył się po przeczytaniu dokładnie $k\cdot n!$ bloków).
\end{document}