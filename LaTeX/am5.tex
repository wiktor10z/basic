\documentclass{article}
\usepackage[utf8]{inputenc}
\usepackage{polski}
\usepackage{amsmath}
\usepackage{anysize}
\marginsize{3,5cm}{3,5cm}{2cm}{5cm}
\begin{document}

Wiktor Zuba 320501 grupa 4
\newline
Zadanie 3.1.
\newline



$
\iiint\limits_{x^2+y^2+z^2<x^\frac{1}{3}}x^pdxdydz\quad (p\in R)
$\newline
stosujemy podstawienie sferyczne:$y=r\cos{\alpha}\cos{\beta},z=r\sin{\alpha}\cos{\beta},x=r\sin{\beta},|J\varphi|=r^2\cos{\beta}$
$
\iiint\limits_{r^2<r^\frac{1}{3}(\sin{\beta})^\frac{1}{3}}r^{p+2}(\sin{\beta})^{p}\cos{\beta}
=
\iiint\limits_{0<r<(\sin{\beta})^\frac{1}{5},0<\beta<\frac{\Pi}{2}}r^{p+2}(\sin{\beta})^{p}\cos{\beta}
$
funkcja jest nieujemna możemy więc skorzystać z twierdzenia Fubiniego:
$
\int\limits_{0}^{2\Pi}d\alpha\int\limits_{0}^{\frac{\Pi}{2}}\bigl((\sin{\beta})^{p}\cos{\beta}\int\limits_{0}^{(\sin{\beta})^\frac{1}{5}}r^{p+2}dr\bigr)d\beta
$
dla $p>-3\quad$
$
2\Pi\int\limits_{0}^{\frac{\Pi}{2}}\bigl(\frac{1}{p+3}(\sin{\beta})^{p}\cos{\beta}(\sin)^\frac{p+3}{5}\bigr)d\beta
=
\frac{2\Pi}{p+3}\int\limits_{0}^{\frac{\Pi}{2}}(\sin{\beta})^\frac{6p+3}{5}\cos{\beta}d\beta
$\newline
dla $p>-\frac{4}{3}\quad$
$
\frac{2\Pi}{p+3}\bigl[\frac{5}{6p+8}(\sin{\beta})^\frac{6p+8}{5}\bigr]_{0}^{\frac{\Pi}{2}}
=
\frac{10\Pi}{(p+3)(6p+8)}
=
\frac{5\Pi}{(p+3)(3p+4)}
$\newline
dla $p=-\frac{4}{3}\quad$
$
\frac{2\Pi}{p+3}\bigl[\ln{\sin{\beta}}\bigr]_{0}^{\frac{\Pi}{2}}
=
\frac{2\Pi}{p+3}\cdot \infty
=
\infty
$\newline
dla $-3<p<-\frac{4}{3}$\quad
$
\frac{2\Pi}{p+3}\bigl[\frac{5}{6p+8}(\sin{\beta})^\frac{6p+8}{5}\bigr]_{0}^{\frac{\Pi}{2}}=\frac{2\Pi}{p+3}\cdot\infty=\infty
$\newline
dla p=-3\quad
$
2\Pi\int\limits_{0}^{\frac{\Pi}{2}}(\sin{\beta})^{-3}\cos{\beta}\bigl[\ln{r}\bigr]_{0}^{(\sin^\frac{1}{5})}d\beta
=
2\Pi\int\limits_{0}^{\frac{\Pi}{2}}(\sin{\beta})^{-3}\cos{\beta}\frac{1}{5}(\ln{\sin{\beta}}+\infty)d\beta
$
funkcja dodatnia, dla $\beta$ odgrodzonego od 0 równa $\infty$ (nie ma problemu$-\infty+\infty$),
całka równa $\infty$
\newline
dla $p<-3$\quad
$
2\Pi\int\limits_{0}^{\frac{\Pi}{2}}(\sin{\beta})^{p}\cos{\beta}(\frac{1}{p+3}(\sin{\beta})^\frac{p+3}{5}+\infty)d\beta
$ jak wyżej\newline
Czyli wartość całki to $\infty$ dla $p\le-\frac{4}{3}$ i $\frac{5\Pi}{(p+3)(3p+4)}$ wpp
\newline
\newline




Zadanie 3.2.
\newline



$
\iiint\limits_{0<z<1-\sqrt{\frac{x^2}{2}+\frac{y^2}{3}}}(x^2+y^2)dxdydz\newline
$
stosujemy podstawienie:$x=\sqrt{2}a,y=\sqrt{3}b,|J\varphi|=\sqrt{6}$, które będąc ściskaniem jest oczywiście gładkim dyfeomorfizmem \newline
$
\sqrt{6}\iiint\limits_{0<z<1-\sqrt{a^2+b^2}}(2a^2+3b^2)dadbdz\newline
$
stosujemy podstawienie walcowe:$a=r\cos{\alpha},b=r\sin{\alpha},|J\varphi|=r$\newline
$
\sqrt{6}\iiint\limits_{0<z<1-r}(2r^3+r^3(\sin{\alpha})^2)dadbdz\newline
$
funkcja nieujemna więc z twierdzenia Fubiniego,($0<r<1$, bo dla innych nie ma takich z)\newline
$
\sqrt{6}\int\limits_{0}^{2\Pi}d\alpha\int\limits_{0}^{1}dr\int\limits_{0}^{1-r}dz (2r^3+r^3(\sin{\alpha}^2))
=
\sqrt{6}\int\limits_{0}^{2\Pi}(2+(\sin{\alpha})^2)d\alpha\cdot\int\limits_{0}^{1}r^3(1-r-0)dr
=
\sqrt{6}\cdot(4\Pi+\Pi)\bigl[\frac{r^4}{4}-\frac{r^5}{5}\bigr]_{0}^{1}
=
\sqrt{6}\cdot5\Pi(\frac{1}{4}-\frac{1}{5})
=
\frac{\Pi\sqrt{6}}{4}
$
\newpage

Wiktor Zuba 320501 grupa 4
\newline
Zadanie 3.3.
\newline



$
\lim\limits_{n \to \infty}\iiint\limits_{\sqrt{x^2+y^2+z^2}<\sqrt[3]{z}}\frac{e^{x^2+y^2+z^2}dxdydz}{n\cdot\sin{(\frac{1}{n}\sqrt{x^2+y^2+z^2})}}
$\newline
stosujemy podstawienie sferyczne(dla każdego n niezależnie): $x=r\cos{\alpha}\cos{\beta},y=r\sin{\alpha}\cos{\beta},z=r\sin{\beta},|J\varphi|=r^2\cos{\beta}$
\newline
$
\lim\limits_{n \to \infty}\iiint\limits_{0<r<\sqrt[3]{r}\sqrt[3]{\sin{\beta}}}\frac{e^{r^2}r^2\cos{\beta}drd\alpha d\beta}{n\cdot\sin{(\frac{r}{n})}}
=
\lim\limits_{n \to \infty}\iiint\limits_{0<r<\sqrt{\sin{\beta}},0<\beta<\frac{\Pi}{2}}\frac{\frac{r}{n}}{\sin{(\frac{r}{n})}}e^{r^2}r\cos{\beta}drd\alpha d\beta
$\newline
ograniczanie części w ułamku:
$
\forall n \in N,0<r<1\quad
1=\frac{\frac{r}{n}}{\frac{r}{n}}\le\frac{\frac{r}{n}}{\sin{(\frac{r}{n})}}\le\frac{\frac{r}{n}}{\frac{r}{n}-\frac{1}{6}(\frac{r}{n})^3}
=\frac{1}{1-\frac{1}{6}(\frac{r}{n})^2}\le\frac{1}{1-\frac{1}{6}}=\frac{6}{5}
$
tak więc poniewaź funkcja $e^{r^2}r\cos{\beta}$ jest całkowalna, to funkcja wyjściowa(nieujemna) jest wspólnie ograniczona przez funkcję całkowalną
- można skorzystać z twierdzenia Lebesque'a o zbieżności zmajoryzowanej\newline
$
\iiint\limits_{0<r<\sqrt{\sin{\beta}},0<\beta<\frac{\Pi}{2}}\lim\limits_{n \to \infty}\frac{\frac{r}{n}}{\sin{(\frac{r}{n})}}e^{r^2}r\cos{\beta}drd\alpha d\beta
=
\iiint\limits_{0<r<\sqrt{\sin{\beta}},0<\beta<\frac{\Pi}{2}}e^{r^2}r\cos{\beta}drd\alpha d\beta
$
funkcja nieujemna, więc z twierdzenia Fubiniego:
$
\int\limits_{0}^{2\Pi}d\alpha\int\limits_{0}^{\frac{\Pi}{2}}\bigl(\cos{\beta}\int\limits_{0}^{\sqrt{\sin{\beta}}}e^{r^2}r\bigr)d\beta
=
2\Pi\cdot\int\limits_{0}^{\frac{\Pi}{2}}\cos{\beta}\bigl[\frac{e^{r^2}}{2}\bigr]_{0}^{\sqrt{\sin{\beta}}}d\beta
=
\Pi\cdot\int\limits_{0}^{\frac{\Pi}{2}}\cos{\beta}(e^{\sin{\beta}}-1)d\beta
=
\Pi\bigl(\bigl[e^{\sin{\beta}}\bigr]_{0}^{\frac{\Pi}{2}}-1\bigl)
=
\Pi(e-1-1)=\Pi(e-2)
$
\newline
\newline




Zadanie 3.4.
\newline



$
\iiiint\limits_{t^2+2x^2+3y^2+4z^2<1}\frac{t^2dtdxdydz}{t^2+2x^2+3y^2+4z^2}
$\newline
Podobnie jak w zadaniu drugim podstawienie:
$
x=\frac{a}{\sqrt{2}},y=\frac{b}{\sqrt{3}},z=\frac{c}{2},|J\varphi|=\frac{1}{2\sqrt{6}}\newline
$
$
\frac{1}{2\sqrt{6}}\iiiint\limits_{t^2+a^2+b^2+c^2<1}\frac{t^2dtdadbdc}{t^2+a^2+b^2+c^2}
$\newline
Następne podstawienie hipersferyczne:
$
t=r\cos{\alpha}\cos{\beta}\cos{\gamma},a=r\sin{\alpha}\cos{\beta}\cos{\gamma},b=r\sin{\beta}\cos{\gamma},c=r\sin{\gamma},|J\varphi|=r^3\cos{\beta}(\cos{\gamma})^2
$\newline
$
\frac{1}{2\sqrt{6}}\iiiint\limits_{0\le r<1}\frac{r^5(\cos{\alpha})^2(\cos{\beta})^3(\cos{\gamma})^4drd\alpha d\beta d\gamma}{r^2}
=
\frac{1}{2\sqrt{6}}\iiiint\limits_{0\le r<1}r^3(\cos{\alpha})^2(\cos{\beta})^3(\cos{\gamma})^4drd\alpha d\beta d\gamma
$\newline
$-\frac{\Pi}{2}<\beta\le\frac{\Pi}{2}$ więc funkcja jest nieujemna, więc z twierdzenia Fubiniego:\newline
$
\frac{1}{2\sqrt{6}}\int\limits_{0}^{2\Pi}d\alpha\int\limits_{-\frac{\Pi}{2}}^{\frac{\Pi}{2}}d\beta\int\limits_{-\frac{\Pi}{2}}^{\frac{\Pi}{2}}d\gamma\int\limits_{0}^{1}dr
\cdot r^3(\cos{\alpha})^2(\cos{\beta})^3(\cos{\gamma})^4
=
\frac{1}{2\sqrt{6}}\bigl(\int\limits_{0}^{2\Pi}(\cos{\alpha})^2d\alpha\bigr)\cdot\bigl(\int\limits_{-\frac{\Pi}{2}}^{\frac{\Pi}{2}}(\cos{\beta})^3d\beta\bigr)\cdot
\bigl(\int\limits_{-\frac{\Pi}{2}}^{\frac{\Pi}{2}}(\cos{\gamma})^4d\gamma\bigr)\cdot\bigl(\int\limits_{0}^{1}r^3dr\bigr)
=
\frac{1}{2\sqrt{6}}\bigl(\int\limits_{0}^{2\Pi}\frac{1}{2}\cos{2\alpha}+\frac{1}{2}d\alpha\bigr)\cdot\bigl(\int\limits_{-\frac{\Pi}{2}}^{\frac{\Pi}{2}}\frac{1}{4}(3\cos{\beta}+cos{3\beta})d\beta\bigr)\cdot
\bigl(\int\limits_{-\frac{\Pi}{2}}^{\frac{\Pi}{2}}\frac{1}{8}(4\cos{2\gamma}+cos{4\gamma}+3)d\gamma\bigr)\cdot\frac{1}{4}
=
\frac{1}{8\sqrt{6}}\cdot\Pi\cdot(\frac{3}{2}+\frac{1}{12}\bigl[\sin{3\beta}\bigr]_{-\frac{\Pi}{2}}^{\frac{\Pi}{2}})\cdot\frac{1}{8}(0+0+3\Pi)
=
\frac{1}{8\sqrt{6}}\Pi(\frac{3}{2}-\frac{1}{6})\frac{3\Pi}{8}=\frac{\Pi^2}{16\sqrt{6}}
$
\newpage

Wiktor Zuba 320501 grupa 4
\newline
Zadanie 3.5
\newline



$
f(x)=\lVert x\rVert^{-p}(1-\lVert x\rVert)^{-q}\quad p,q>0$ kiedy funkcja całkowalna na kuli w $R^k$
$\newline
\int\limits_{\lVert x\rVert<1}\bigl|\lVert x\rVert^{-p}(1-\lVert x\rVert)^{-q} \bigr|<\infty
$
 Funkcja nieujemna, więc można pozbyć się wartości bezwzględnych,
$
\int\limits_{\lVert x\rVert<1}\lVert x\rVert^{-p}(1-\lVert x\rVert)^{-q}
=
\int\limits_{\lVert x\rVert<\frac{1}{2}}\lVert x\rVert^{-p}(1-\lVert x\rVert)^{-q}
+
\int\limits_{\frac{1}{2}\le\lVert x\rVert<1}\lVert x\rVert^{-p}(1-\lVert x\rVert)^{-q}
$
Całka wyjściowa jest skończona wtw gdy obie te całki są skończone(bo funkcja nieujemna)
$(1-\lVert x\rVert)^{-q}$ dla $\lVert x\rVert<\frac{1}{2}$ jest ograniczone z dołu i z góry$(q>0)$\newline
$1=1^{-q}\le(1-\lVert x\rVert)^{-q}\le(\frac{1}{2})^{-q}=2^q$ więc i pierwsza całka jest ograniczona przez\newline
$
\int\limits_{\lVert x\rVert<\frac{1}{2}}\lVert x\rVert^{-p}\le
\int\limits_{\lVert x\rVert<\frac{1}{2}}\lVert x\rVert^{-p}(1-\lVert x\rVert)^{-q}
\le2^q\int\limits_{\lVert x\rVert<\frac{1}{2}}\lVert x\rVert^{-p}\newline
$ więc skończona wtw gdy $\int\limits_{\lVert x\rVert<\frac{1}{2}}\lVert x\rVert^{-p}$\newline
,funkcja nieujemna:podstawienie sferyczne(w k-wymiarze)+ z Fubiniego\newline
$
\int\limits_{0}^{2\Pi}d\alpha\int\limits_{-\frac{\Pi}{2}}^{\frac{\Pi}{2}}\cos{\beta}d\beta \cdots \int\limits_{0}^{\frac{1}{2}}r^{k-1-p}dr=
Const\cdot\int\limits_{0}^{\frac{1}{2}}r^{k-1-p}dr=Const\cdot\frac{1}{k-p}(\frac{1}{2})^{k-p}$ dla $p<k$ i $\infty$ dla $p\ge k$

$\lVert x\rVert^{-p}$ dla $\frac{1}{2}\le\lVert x\rVert<1$ jest ograniczone z dołu i z góry$(p>0)$\newline
$1=1^{-p}\le\lVert x\rVert^{-p}\le(\frac{1}{2})^{-p}=2^p$ więc i druga całka jest ograniczona przez\newline
$
\int\limits_{\frac{1}{2}\le\lVert x\rVert<1}(1-\lVert x\rVert)^{-q}\le
\int\limits_{\frac{1}{2}\le\lVert x\rVert<1}\lVert x\rVert^{-p}(1-\lVert x\rVert)^{-q}
\le2^p\int\limits_{\frac{1}{2}\le\lVert x\rVert<1}(1-\lVert x\rVert)^{-q}\newline
$ więc skończona wtw gdy $\int\limits_{\frac{1}{2}\le\lVert x\rVert<1}(1-\lVert x\rVert)^{-q}$\newline
,funkcja nieujemna:podstawienie sferyczne(w k-wymiarze)+ z Fubiniego\newline
$
\int\limits_{0}^{2\Pi}d\alpha\int\limits_{-\frac{\Pi}{2}}^{\frac{\Pi}{2}}\cos{\beta}d\beta \cdots \int\limits_{\frac{1}{2}}^{1}r^{k-1}(1-r)^{-q}dr=
Const\cdot\int\limits_{\frac{1}{2}}^{1}r^{k-1}(1-r)^{-p}dr
$
Kolejna zamiana zmiennych $y=1-r,|J\varphi|=1$\quad
$
Const\cdot\int\limits_{0}^{\frac{1}{2}}(1-y)^{k-1}y^{-q}dy
$ podobne ograniczanie jak wcześniej
$
Const\cdot2^{-k+1}\int\limits_{0}^{\frac{1}{2}}y^{-q}dy\le
Const\cdot\int\limits_{0}^{\frac{1}{2}}(1-y)^{k-1}y^{-q}dy
\le Const\cdot\int\limits_{0}^{\frac{1}{2}}y^{-q}dy\quad
$
$\int\limits_{0}^{\frac{1}{2}}y^{-q}dy=\frac{1}{1-q}2^{q-1}$ dla $q<1$ i $\infty$ dla $q\ge 1$\newline
Więc ostatecznie funkcja wyjściowa jest całkowalna dla $0<p<k,0<q<1$

\end{document}