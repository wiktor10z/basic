\documentclass{article}
\usepackage[utf8]{inputenc}
\usepackage{polski}
\usepackage{amsmath}
\usepackage{anysize}
\usepackage{amssymb}
\begin{document}

Wiktor Zuba 320501
\newline

Zadanie 6.1.
\newline
\newline
$K=-\frac{y"(s)}{y(s)}$\newline
Dla danego punktu w $\mathbb{R}^3$ i ustalonych wartości $n$ i $s$ dobieramy tak układ współrzednych aby wpółrzedna z była prostopadła do tego "wybieranego" kątu
oraz aby $y_k(s)\neq 0$ dla każdego $k\in\{0..n-1\}$, gdzie $y_k(s)=r(s)*\sin(\omega_k(s))=r(s)*\sin(\omega(s)+\frac{2k\pi}{n})$.\newline
Wtedy $y"(s)=r"(s)\sin(\omega_k(s))+2r'(s)\cos(\omega_k(s))\omega'(s)-r(s)\sin(\omega_k(s))(\omega'(s))^2+r(s)\cos(\omega_k(s))\omega"(s)$\newline
$K_k(s)=a(s)+b(s)\ctg(\omega_k(s))$.\newline
$\frac{1}{n}\sum\limits_{k=0}^{n-1}K_k(s)=a(s)+\frac{1}{n}b(s)*\sum\limits_{n=0}^{k-1}\ctg(\omega+\frac{2k\pi}{n})=a(s)+\frac{1}{n}b(s)\cdot 0=a(s)$
(Z GAL1 dla takich $\omega$, że $\ctg$ dobrze określony - zgodnie z założeniami to zachodzi).
\newline

Zadanie 6.3.
\newline
\newline
Prowadząc wszystkie obliczenia (pomijając wypiwanie części):\newline
$p_u=(\cos(v),\sin(v),0),p_v=(-u\sin(v),u\cos(v),1),q_u=(\cos(v),\sin(v),\frac{1}{u}),q_v=(-u\sin(v),u\cos(v),0)$
$\det(M(G_p))=\det(M(G_v))=u^2+1,\newline
N_p=(\sin(v),-\cos(v),u)\cdot(u^2+1)^{-\frac{1}{2}},N_q=(-\cos(v),\sin(v),u)\cdot(u^2+1)^{-\frac{1}{2}}$\newline
$\det(M(B_p))=\det(M(B_q))=(-1)\cdot(u^2+1)^{-1}\Rightarrow K_p(u,v)=K_q(u,v)=-\frac{1}{(u^2+1)^2}$\newline
Dwa punkty odpowiedające w hipotetycznych izometrycznych wewnętrznie podzbiorach muszą mieć tą samą krzywiznę, a więc tą samą współrzędną $u$.
Dla obszaru dwuwymiarowego jedna ze współrzędnych rozpinających jest ustalona ($u$),
więc długość drogi po drugiej musiałaby być taka sama dla obu obszarów (podzbiorów), jednak nie jest, gdyż dla stałego $u$ tylko $v$ moze się zmieniać,
a dla zmiany $v$ w rozmaitości wyznaczonej przez $q$ poruszamy się tylko po okregu, zaś dla tej z $p$ po spirali o tym samym promieniu, a więc obie drogi nie
są izomorficzne, co przeczy izomorficzności obszarów.
\newline

Zadanie 6.5.
\newline
\newline
Podobnie jak w przypadku stożka geodezyjna nie będąca południkiem najpierw zbliża się do "wierzchołka" paraboloidy,
a później się oddala przecinając swoją część ze zbliżania (kierunek obrotu wokół osi jest zachowany), jednak inaczej niż w przypadku stożka geodezyjna obiegnie
środek nieskończenie wiele razy w obu częściach wyznaczonych poprzez rozdzielenie w punkcie najblizszym "wierzchołka",
a więc i przecięcie nastąpi nieskończenie wiele razy.\newline
Nieskończenie wiele okrążeń osi obrotu paraboloidy przez geodezyjną można wywieść z tego, że podczas gdy dla stożka geodezyjna staje się w końcu równoległa do
pewnego południka, to w paraboloidzie odchylenie spada wraz z oddalaniem się od "wierzchołka", przez co "prędkość obrotowa"
nie ma możliwości spaść do 0 w skończonym czasie co więcej ponieważ jest jak $\frac{1}{x}$, które nie jest całkowalne na żadnym przedziele $(a,\infty)$
(z analizy matematycznej).

\end{document}