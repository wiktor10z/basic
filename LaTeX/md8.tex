\documentclass{article}
\usepackage[utf8]{inputenc}
\usepackage{polski}
\usepackage{amsmath}
\usepackage{anysize}
\usepackage{color}
\usepackage[usenames,dvipsnames]{xcolor}
\marginsize{2,5cm}{2,5cm}{1cm}{4cm}
\begin{document}

Wiktor Zuba 320501
\newline
Zadanie 8.
\newline
\newline
$2013=3\cdot 11\cdot 61$- rozkład 2013 na czynniki pierwsze, więc wszystkie dzielniki $2013^{2013}$ są postaci $3^{i}\cdot11^{j}\cdot61^{k},$
gdzie $i,j,k\in[0,2013]$\newline
$N=\prod\limits_{i,j,k\in[0,2013]}(3^{i}\cdot11^{j}\cdot61^{k})
=
\prod\limits_{i=0}^{2013}(\prod\limits_{j=0}^{2013}(\prod\limits_{k=0}^{2013}(3^{i}\cdot11^{j}\cdot61^{k})))
=
\prod\limits_{i=0}^{2013}(\prod\limits_{j=0}^{2013}(3^{2014i}\cdot11^{2014j}\cdot61^{\sum\limits_{k=0}^{2013}k}))
=
\prod\limits_{i=0}^{2013}(3^{2014^2}\cdot11^{2014\sum\limits_{j=0}^{2013}j}\cdot61^{2014\sum\limits_{k=0}^{2013}k})
=
3^{2014\sum\limits_{i=0}^{2013}i}\cdot11^{2014^2\sum\limits_{j=0}^{2013}j}\cdot61^{2014^2\sum\limits_{k=0}^{2013}k}
=
2013^{2014^2\sum\limits_{i=0}^{2013}i}=2013^{\frac{(2014)^32013}{2}}=2013^{(1007)(2013)(2014)^2}\newline
(a+b)^n=\sum\limits_{k=0}^{n}{n\choose k}a^kb^{n-k}
$dla a=13, b=2000 wyrazy mające w sobie b w potędze wyższej niż 0 są podzielne przez 100, więc\newline
$
N=2013^{(1007)(2013)(2014)^2}=13^{(1007)(2013)(2014)^2}\mod{(100)}\newline
$
przy mnożeniu i dodawaniu czynników na ostatnie dwa miejsca w zapisie dziesiętnym wpływ mają tylko ostatnie dwa miejsca w zapisie dziesiętnym czynników-
można je pomijać(suma pełnych setek i iloczyn stu z liczbą całkowitą są podzielne przez 100)\quad\quad
$
13^4=28561=61\mod{(100)}\newline
(10k+1)\cdot13^4=(10k+1)\cdot61\mod{(100)}=610k+61=(10(k+6)+1)\mod{(100)}\Rightarrow 13^{20}=1\cdot(13^4)^5=10(5\cdot6)+1\mod{(100)}=1\mod{(100)}
\Rightarrow$ mnożenie przez $13^{20}$ nie zmienia reszty z dzielenia przez 100\newline
$
N=2013^{(1007)(2013)(2014)^2}=13^{(1007)(2013)(2014)^2}\mod{(100)}=13^{(1000+7)(2000+13)(2000+14)^2}=13^{7\cdot13\cdot(14)^2}\mod{(100)}
=13^{91\cdot196}=13^{(80+11)(180+16)}
=13^{11\cdot16}\mod{(100)}=13^{176}=13^{16}\mod{(100)}=1\cdot(13^4)^4=10(4\cdot6)+1\mod{(100)}=241=\underline{41}\mod{(100)}\newline
$
Tak więc dwie ostatnie cyfry zapisu dziesiętnego liczby $N$ to 4 i 1
\end{document}