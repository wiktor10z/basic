\documentclass{article}
\usepackage[utf8]{inputenc}
\usepackage{polski}
\usepackage{amsmath}
\usepackage{anysize}
\usepackage{amssymb}
\begin{document}

Wiktor Zuba 320501\newline

Zadanie 1\newline

Algorytm sprawdzania czy dany język regularny akceptuje słowo posiadające $1$ na miejscach o indeksach będących kwadratami liczb naturalnych
i $0$ na pozostałych (czyli $1001(00)^2...1(00)^n1(00)^{n+1}...$).\newline

Zaczynając od krawędziowego deterministycznego automatu z warunkiem Mullera otrzymanego z wejściowego automatu z warunkeim Buchiego przeprowadzam szereg redukcji:\newline
Rozdzielenie automatu na kilka odpowiadających każdemu zbiorowi krawędzi akceptujących i sprawdzenie dla każdego takiego automatu osobno.\newline
(Jeżeli w trakcie redukcji automatu usuwamy krawędź akceptującą lub okazuje się, że ze stanu w którym się znajdujemy nie można odczytać właściwego znaku,
to automat nie akceptuje naszego słowa)\newline
Korzystając z deterministyczności automatu dochodzimy do miejsca $n^2$, gdzie $n$ to liczba stanów automatu. Występujące od tego miejsca ciągi $0$ są tak długie,
że pomiędzy kolejnymi $1$ musi wystąpić cykl na którym czytane są $0$ - z deterministyczności automatu czytając $0$ nie możemy wyjść z cyklu,
tak więc od tego momentu przechodzenie po stanach wygląda następująco: przemieszczamy się po ścieżce krawędziami $0$, dochodzimy do cyklu i po kilku obrotach
wychodzmy z niego ścieżką $1$ do nowego początku ścieżki (być może zerowej długości - czyli bezpośrednio w jakiś cykl).
Wszystkie wierzchołki z których nie wychodzi krawędź $0$ mozna usunąć, mozna też usunąć wszystkie krawędzie $1$ nie wychodzące z cyklu zerowego oraz
wszystkie wierzchołki nieosiągalne (aż do fix-pointu).\newline
Zaczynając z nowego stanu początkowego (po przejściu $n^2$ kroków w automacie wyjściowym) chodzimy po nowym automacie złożonym z krawędzi $1$,
których odwiedzenie implikuje odwiedzenie całej ścieżki $0$ z nich wychodzącej, oraz cyklu ją kończącego. Możemy więc zająć się jedynie tymi krawędziami
i funkcją przejścia między nimi (która jest modulo długość odpowiedniego cyklu, a więc jest modulo NWW długości wszystkich cykli
(których jest mniej niż krawędzi)). Jeśli po przeczytaniu $k*$NWW $1$ wrócimy do tego samego stanu, to oznacza to, że znajdujemy się na cyklu.
Ponieważ ilość krawędzi $1$ jest ograniczona przez $n$, to po $n$ przeczytaniach po NWW $1$ wychodząc z nowego stanu początkowego
któryś stan będący następnikiem $n+k*$NWW-szej $1$ musi się powtórzyć - mając cykl długości $k*$NWW przechodzimy go jeszcze raz sprawdzając czy przechodzimy przez wszystkie krawędzie akceptujące
(i tylko takie).\newline\newline

Uwaga: Redukcja poza początkowym rozdzieleniem nie jest konieczna.
Pozwala tylko usystematyzować strukturę automatu na którym lepiej widać zasadność kolejnych kroków.\newpage

Wiktor Zuba 320501\newline

Zadanie 2\newline

Determinizacja niedeterministycznego automatu z warunkiem Buchiego.\newline

Algorytm :\newline
1. Przekształcam wyjściowy automat w deterministyczny automat z warunkiem Mullera i akceptującą rodziną zbiorów stanów $\Gamma'$.\newline
2. Usuwam z $\Gamma'$ zbiory nieosiągalne lub  nie silnie spójne, otrzymując rodzinę $\Gamma$.\newline
3. Dla każdego zbioru stanów $F\in\Gamma,E\notin\Gamma,F\subseteq E$ jeśli $E$ jest silnie spójny,
to nie istnieje deterministyczny automat z warunkiem Buchiego, który akceptuje ten sam język, co automat wyjściowy.\newline
4. Jeżeli determinizacja nie została odrzucona na etapie 3, to deterministycznym automatem Buchiego akceptującym ten sam język
jest automat powstający z automatu Mullera poprzez rozmnożenie stanów, tak by pamiętały one jakie stany w oryginalnym automacie Mullera zostały odwiedzone.
Stanami akceptującymi są stany postaci $(p,X)$, takie że $F\subseteq X$ dla pewnego $F\in\Gamma$.
Tranzycje powstają poprzez przerobienie tranzycji $p\rightarrow q$ z automatu z warunkiem Mullera
na tranzycję $(p,X)\rightarrow (q,X\cup\{q\})$(o ile stan $(p,X)$ nie był akceptujący) lub $(p,X)\rightarrow (q,\{q\})$ (o ile był).\newline

Dowód poprawności:\newline
Uwaga : Deterministyczny automat z warunkiem Mullera i rodziną akceptującą $\{E\}$ ($E\subseteq Q$) akceptuje język niepusty wtedy i tylko wtedy,
gdy podgraf automatu generowany przez $E$ jest silnie spójny oraz osiągalny ze stanu początkowego.\newline
Krok 1 : Z wykładu.\newline
Krok 2 : Jeśli do rodziny akceptującej automatu dodamy lub zabierzemy zbiór który nie spełnia własności z uwagi,
to nie zmieni się język akceptowany przez automat (automaty z rodzinami $\Gamma'$ oraz $\Gamma$ są równoważne).\newline
Krok 3 : Załóżmy, że w automacie $\exists E\notin\Gamma,F\in\Gamma,F\subseteq E$, spełniające własność z uwagi
(istnieje słowo akceptowalne przez ten zbiór := takie które odwiedza dokładniete te stany $\infty$ wiele razy).\newline
Załóżmy, że istnieje deterministyczny automat z warunkiem Buchiego, który akceptuje słowa akceptowalne przez $F$ w automacie z warunkiem Mullera,
ale nie akceptuje tych akceptowalnych przez $E$.
Automat z warunkiem Buchiego ma skończenie wiele stanów ($k$), dlatego też jeżeli weźmiemy słowo ostatecznie okresowe,
to po conajwyzej $k$ pełnych odczytaniach okresu wylądujemy w tym samym stanie i będziemy się przemieszczeć po stałym cyklu stanów (z deterministyczności).
Definiuje słowo $w$ jako takie, którego początek powstaje przez dojście w automacie z warunkiem Mullera do zbioru stanów $F$,
a następnie wykonanie $k$ cykli przechodzących po wszystkich stanach należących do $F$. Gdyby powtórzyć cykl nie $k$ a $\infty$ wiele razy
słowo było by akceptowalne przez $F$, a więc w automacie z warunkiem Buchiego na cyklu po którym by krążyło musi istnieć co najmniej jeden stan akceptujących
(zostaje on odwiedzony w podanym prefiksie słowa $w$). Kolejny fragment słowa $w$ powstaje poprzez wykonanie jednego cyklu
w automacie z warunkiem Mullera po wszystkich stanach z $E$ i kolejnych $k$ po stanach z $F$.
Analogicznie do poprzedniego rozumowania w automacie z warunkiem Buchiego odwiedzony jest co najmniej jeden stan akceptujący.
Powtarzając konstrukcję $\infty$ wiele razy słowo $w$ odwiedza w automacie z warunkiem Mullera $\infty$ wiele razy stany $F\cup E=E$,
jednak jest akceptowane przez automat z warunkiem Buchiego, co daje nam sprzeczność.\newline
Krok 4 : Automat z warunkiem Buchiego jest deterministyczny ponieważ automat z warunkiem Mullera taki był.\newline
Jeśli automat z warunkiem Mullera akceptuje jakieś słowo zbiorem stanów $F$,
to w automacie z warunkiem Buchiego $\infty$ wiele razy pojawi się stan akceptujący (pojawi się stan pamiętający nadzbiór $F$).\newline
Jeśli automat z warunkiem Buchiego akceptuje jakieś słowo, to któryś stan jest odwiedzany $\infty$ wiele razy $\Rightarrow$ w automacie Mullera
pewien zbiór stanów $F\in\Gamma$ został odwiedzony $\infty$ wiele razy, więc słowo jest akceptowalne przez pewien nadzbiór $F$,
a więc jest akceptowane przez automat z warunkiem Mullera.


\end{document}