\documentclass{article}
\usepackage[utf8]{inputenc}
\usepackage{polski}
\usepackage{amsmath}
\usepackage{anysize}
\usepackage{amssymb}
\newcommand{\sgn}{\operatorname{sgn}}
\marginsize{1,5cm}{2cm}{1cm}{3cm}
\begin{document}

Wiktor Zuba 320501 grupa 4
\newline

Zadanie 10.1.
\newline
\newline
$
M=\{(x,y,z):x^2+y^2+z^2=\sqrt{z}>0\}
$-taka "bańka"- półsfery, ale tak zakrzywione(jak sinus), że się zamyka w środku układu współrzędnych(bez jednego punktu)
i dzieli przestrzeń na 2 składowe (po dodaniu tego punktu),
gdzie ta ograniczona($x^2+y^2+z^2<\sqrt{z}$) zawiera punkt$(0,0,\frac{1}{2})$,z twierdzenia o dywergencji\newline
$
\iint\limits_{M}\overrightarrow{F}=\iiint\limits_{x^2+y^2+z^2<\sqrt{z}}div\overrightarrow{F}dl_3
=\iiint\limits_{x^2+y^2+z^2<\sqrt{z}}\frac{1}{z}+\frac{1}{z}+\frac{1}{2\sqrt{z}}dl_3
$
podstawienie sferyczne
$
\iiint\limits_{r^{\frac{3}{2}}<\sqrt{\sin{\beta}}}(\frac{2}{r\sin{\beta}}+\frac{1}{2\sqrt{r\sin{\beta}}})r^2\cos{\beta}drd\alpha d\beta
$
Z twierdzenia Fubiniego(nieujemna)
$
\int\limits_{0}^{2\pi}d\alpha\int\limits_{0}^{\frac{\pi}{2}}d\beta\int\limits_{0}^{\sqrt[3]{\sin{\beta}}}dr 2r\ctg{\beta}+\frac{r^{\frac{3}{2}}\cos{\beta}}{2\sqrt{\sin{\beta}}}=
2\pi\cdot\int\limits_{0}^{\frac{\pi}{2}}(\frac{\cos{\beta}}{\sqrt[3]{\sin{\beta}}}+\left[\frac{r^{\frac{5}{2}}\cos{\beta}}{5\sqrt{\sin{\beta}}}\right]_{0}^{\sqrt[3]{\sin{\beta}}})d\beta=\newline
2\pi\cdot\int\limits_{0}^{\frac{\pi}{2}}(\frac{\cos{\beta}}{\sqrt[3]{\sin{\beta}}}+\frac{1}{5}\cos{\beta}\sqrt[3]{\sin{\beta}})d\beta=
2\pi\left[\frac{3}{2}\sin^{\frac{2}{3}}{\beta}+\frac{3}{20}\sin^{\frac{4}{3}}{\beta}\right]_{0}^{\frac{\pi}{2}}=2\pi(\frac{3}{2}+\frac{3}{20})=\underline{\frac{9}{5}\pi}
$
\newline
\newline

Zadanie 10.2.
\newline
\newline
$
\overrightarrow{F}=\left[\begin{array}{c}y^2-x^2\\z-x\\2xz-y\\\end{array}\right]
$
cykloida obrócona wokół osi symetrii- powstaje "czapka" zorientowana "do góry", gdyby dopełnić tę rozmaitość kołem $(x-\pi)^2+y^2<\pi^2,z=0$ zorientowanym "do dołu"(+miary zero),
to otrzymalibyśmy brzeg pewnej rozmaitości wymiaru 3, z twierdzenia o dywergencji\newline 
$
\iint\limits_{\text{obr. cykl.,+}}\omega+\iint\limits_{(x-\pi)^2+y^2<\pi^2,+}\omega=\iiint\limits_{\text{bryła}}div\overrightarrow{F}dl_3\quad\quad(z=0\Rightarrow d\text{coś}\wedge dz=0)\newline
\iint\limits_{\text{obr. cykl.,+}}\omega=\iiint\limits_{\text{bryła}}(-2x+2x)dl_3-\iint\limits_{(x-\pi)^2+y^2<\pi^2,+}-y dx\wedge dy \quad\quad\quad
\iint\limits_{\text{obr. cykl.,+}}\omega=\iint\limits_{(x-\pi)^2+y^2<\pi^2,+}y dx\wedge dy\newline
$ forma nieparzysta ze względu na y, a obszar symetryczny względem prostej y=0, więc całka po formie równa 0 (niezależnie od orientacji)
$
\iint\limits_{\text{obr.cykl.,+}}\omega=\underline{0}
$
\newline
\newline

Zadanie 10.3.
\newline
\newline
$
\omega=x_j dx_1\wedge...\wedge dx_{i-1}\wedge dx_{i+1}\wedge...\wedge dx_k\quad\quad
\overrightarrow{F}=[0...0,\pm x_j,0...0]^{\bot}
$, gdzie $x_j$ na i-tym miejscu
,z twierdzenia o dywergencji\newline
Całka=$\int\limits_{x\in\mathbb{R}^k:\lVert x\rVert<1}div\overrightarrow{F}dl_k=\int\limits_{x\in\mathbb{R}^k:\lVert x\rVert<1}\pm\frac{\delta x_j}{\delta x_i}dl_k$
co jest równe \underline{0} dla $j\neq i$,\underline{miara kuli} (czyli $\frac{\pi^r}{r!}$ dla $k=2r,\frac{2^r\pi^{r-1}}{k!!}$ dla $k=2r-1)$ dla i=j, nieparzystych,
oraz \underline{- miara kuli} dla i=j,parzystych.
\newline

Zadanie 10.4.
\newline
\newline
Z klasycznego twierdzenia Stokesa $\int\limits_{C,+}fdx+gdy+hdz=\iint\limits_{D}rot\overrightarrow{F}\cdot\overrightarrow{u}dS$, gdzie D- rozmaitość, a C jej brzeg,
w tym wypadku brzegiem są $\{x^2+y^2=1,z=5\}$ oraz $\{x^2+y^2=1,2x+y+z=2\}$, zorientowane tak, aby wyznacznik macierzy złożonej z wektorów:
prostopadłego do rozmaitości(w stronę dodatnią),stycznego do rozmaitości a prostopadłego do tego brzegu(na zewnątrz) i wektora stycznego do tego brzegu (dającego mu orientację)
był większy od 0. Weźmy na brzegach punkty x=1,y=0, oraz wektory (1,0,0),(0,0,$\pm$1), aby wyznacznik był dodatni orientacja górnego okręgu w stronę malejącego y(ujemna),
a w drugiej części odwrotnie (dodatnia).
$\iint\limits_{D}rot\overrightarrow{F}\cdot\overrightarrow{u}dS=\int\limits_{C,+}fdx+gdy+hdz=
\int\limits_{x^2+y^2=1,z=5,-}yzdx+x^3zdy+e^zdz+\int\limits_{x^2+y^2=1,z=2-2x-y,+}yzdx+x^3zdy+e^zdz
$
podstawienia okrężne
$
-5\int\limits_{0}^{2\pi}(-\sin^2{\alpha}+\cos^4{\alpha})d\alpha+
\int\limits_{0}^{2\pi}-\sin^2{\alpha}(2-2\cos{\alpha}-\sin{\alpha})+\cos^4{\alpha}(2-2\cos{\alpha}-\sin{\alpha})+e^{2-2\cos{\alpha}-\sin{\alpha}}(2\sin{\alpha}-\cos{\alpha}d\alpha)=
-5(-\pi+\frac{3}{4}\pi)+2(-\pi+\frac{3}{4}\pi)+\int\limits_{0}^{2\pi}2\sin^2{\alpha}\cos{\alpha}+\sin^3{\alpha}-2\cos^5{\alpha}-\sin{\alpha}\cos^4{\alpha}d\alpha-
\left[e^{2-2\cos{\alpha}-\sin{\alpha}}\right]_{0}^{2\pi}=
\frac{5\pi}{4}-\frac{\pi}{2}+0+0-0-0-0=\underline{\frac{3\pi}{4}}
$ (ponieważ funkcje podcałkowe $2\pi$-okresowe,część dodatnia równa części ujemnej)
\end{document}






