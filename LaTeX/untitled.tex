a\newline
ą\newline
b\newline
c\newline
ć\newline
d\newline
e\newline
ę\newline
f\newline
g\newline
h\newline
i\newline
j\newline
k\newline
l\newline
ł\newline
m\newline
n\newline
ń\newline
o\newline
ó\newline
p\newline
q\newline
r\newline
s\newline
ś\newline
t\newline
u\newline
v\newline
w\newline
x\newline
y\newline
z\newline
ź\newline
ż\newline
{\Large
\begin{tabular}{|c|}\hline

$\emptyset$(brak założeń)\\\hline($3\frac{1}{7},2\frac{5}{7}$),$c^{\top}x=14\frac{3}{7}$ \\\hline\hline
$x_1\le 3$\\\hline($3,2\frac{7}{9}$),$c^{\top}x=14\frac{1}{3}$ \\\hline\hline
$x_1\le 3,x_2\le 2$\\\hline($3,2$),$c^{\top}x=12$ \\\hline\hline
$x_1\le 3,x_2\ge 3$\\\hline($2\frac{1}{2},3$),$c^{\top}x=14$ \\\hline\hline
$x_1\le 2,x_2\ge 3$\\\hline($2,3\frac{2}{9}$),$c^{\top}x=13\frac{2}{3}$  \\\hline\hline
$x_1\le 2,3\le x_2\le 3$\\\hline($2,3$),$c^{\top}x=13$  \\\hline\hline
$x_1\le 2,x_2\ge 4$\\\hline($\frac{1}{4},4$),$c^{\top}x=12\frac{1}{2}$  \\\hline\hline
$3\le x_1\le 3,x_2\ge 3$\\\hline$\emptyset$(brak rozwiązań) \\\hline\hline
$x_1\ge 4$\\\hline($4,1$),$c^{\top}x=11$  \\\hline\hline
\end{tabular}
}
      \subsection{Travelling Salesman Problem}
W grafie z przypisanymi wagami problem komiwojażera jest uogólnieniem problemu cyklu Hamiltona, wyszukującym cykl Hamiltona o najmniejszej wadze.\newline\newline
%
Twierdzenie 5.2.3.1. Dla każdej funkcji $\alpha(n)$ obliczalnej w czasie wielomianowym od wielkości grafu nie ma $\alpha(n)$-aproksymacji TSP, o ile $P\neq NP$.\newline\newline
%
Szkic dowodu: jeśli istaniała by $\alpha(n)$ aproksymacja wystarczyłoby dla dowolnego grafu G stworzyć graf pełny H o wagach
$w((u,v))=1$ dla $(u,v)\in E(G)$, $w((u,v))=n\cdot\alpha(n)$ dla $(u,v)\notin E(G)$. Otrzymanie $\alpha$-aproksymacji dla problemu komiwojażera dla grafu H oznaczałoby rozwiązanie
NP-trudnego problemu grafu Hamiltona dla dowolnego grafu G.\newline\newline
%
Metryczny problem komiwojażera zakłada, że wagi krawędzi pomiędzy wierzchołkami spełniają warunek trójkąta
(założenie to jest spełniane przez większość rzeczywistych zastosowań problemu).\newline

%Programowanie liniowe jest klasą problemów zajmujących się maksymalizacją lub minimalizacją funkcji liniowej wielu zmiennych przy
%założeniu spełniania przez nie pewnej liczby również liniowych warunków.\newline%TODO