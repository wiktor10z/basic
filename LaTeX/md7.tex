\documentclass{article}
\usepackage[utf8]{inputenc}
\usepackage{polski}
\usepackage{amsmath}
\usepackage{anysize}
\usepackage{color}
\usepackage[usenames,dvipsnames]{xcolor}
\marginsize{2,5cm}{2,5cm}{1cm}{4cm}
\begin{document}

Wiktor Zuba 320501
\newline
Zadanie 7.
\newline
\newline
Rozważam kilka przypadków:\newline
1- ścieżka- ma tylko 2 liście więc połączenie w k-cykl nie jest możliwe- nie ma czegoś takiego jak 2-cykl, ale interpretując to jako tylko połączenie tych dwóch
wierzchołków początkowego i końcowego otrzymujemy z grafu cykl- który jest 3-kolorowalny\newline
2- gwiazda o parzystej liczbie liści - kolorujemy wierzchołek wewnętrzny na pierwszy kolor,
a pozostałe w utworzonym cyklu na przemian na drugi i trzeci kolor\newline
3- dla drzewa po usunięciu liści otrzymujemy ponownie drzewo(zredukowane) w tym przypadku co najmniej dwu wierzchołkowe- w przeciwnym przypadku jest to gwiazda,
a ten przypadek jest już rozważony.\newline\newline
Kilka faktów:\newline
Ponieważ w drzewie pomiędzy każdymi dwoma wierzchołkami istnieje dokładnie jedna ścieżka to jest on 2-kolorowalny(na dokładnie 2 sposoby),
tak więc w tym przypadku kolorujemy to drzewo zredukowane kolorami \textcolor{Green}{a} i \textcolor{blue}{b} i wybieramy jeden wierzchołek 
($\widetilde{w}$) który w tym zredukowanym drzewie jest liściem,
w drzewie pierwotnym odchodzi od niego liść (conajmniej jeden bo inaczej sam byłby liściem i nie znalazłby się w drzewie zredukowanym)
możemy założyć że jest on pokolorowany na \textcolor{Green}{a} (ponieważ zawsze możemy odwrócić te 2 kolory).\newline
W narysowanym drzewie by otrzymać ów k-cykl musimy połączyć liście "po kolei"- inaczej istnieje liść zamknięty pomiędzy dwoma niekolejnymi połączonymi ze sobą
liściami, i drugi odseparowany od niego po drugiej stronie, a jednocześnie te dwa wierchołki muszą się znaleźć na cyklu,
żeby graf był planarny prowadząc dwie ścieżki po liściach od jednego odseparowanego wierzchołka do drugiego musimy odwiedzić te dwa połączone niekolejne wierzchołki,
ale to spowoduje zamknięcie cyklu, niezawierającego tego drugiego odseparowanego wierzchołka.\newline
Ponieważ $\widetilde{w}$ sąsiaduje z tylko jednym nieliściem, to wszystkie liście z nim sąsiadujące(liście $\widetilde{w}$) na rysunku są "po kolei"
-gdyż ich krawędzie odchodzą "po kolei" z wierzchołka, a więc i z z drzewa zredukowanego. Z tych dwóch uwag wynika, że wszystkie liście
$\widetilde{w}$ są w tym k-cyklu połączone ścieżkami nie zawierającymi pozostałych liści.\newline\newline
Ponieważ $\widetilde{w}$ przed pokolorowaniem liści $\widetilde{w}$ sąsiaduje tylko z jednym pokolorowanym wierzchołkiem (na \textcolor{blue}{b}), to
w razie potrzeby możemy do przemalować na kolor \textcolor{red}{c} bez stwarzenia kolizji.\newline
Gdy w grafie G(T), z pokolorowanym już drzewem zredukowanym na 2 kolory zaczniemy malować niepokolorowane wierzchołki po kolei, to zawsze mamy conajmniej 1 kolor
dostępny $\rightarrow$ różny od koloru wierzchołka od którego dany liść odchodzi, i od koloru poprzedniego w kolejności liścia -
jedyny problem może nam sprawić zakańczanie cykluw którym to momencie kolorujemy wierchołek który sąsiaduje z 3 już pokolorowanymi
- malujemy tak by zdażyło się to dla liścia $\widetilde{w}$.\newline\newline
Rozważmy kilka przypadków:\newline
$\widetilde{w}$ ma parzyście wiele liści:\quad
Na k- cyklu malujemy na kolor \textcolor{red}{c} pierwszy liść po liściach $\widetilde{w}$,i tak malując przechodzimy cykl aż pomalujemy wszystkie oprócz
liści $\widetilde{w}$. Jeżeli ostatni pokolorowany wierzchołek ma kolor \textcolor{Green}{a} lub \textcolor{blue}{b}
to nasze liście kolorujemy na przemian na kolory \textcolor{red}{c} i \textcolor{blue}{b} 
(ten pokolorowany na \textcolor{Green}{a} lub \textcolor{blue}{b} sąsiaduje z \textcolor{red}{c},
a z parzystości nasz ostatni pokolorowany na \textcolor{blue}{b} sąsiaduje z pokolorowanym na \textcolor{red}{c}).
Jeżeli ostatni pokolorowany wierzchołek ma kolor \textcolor{red}{c}, to zmieniamy kolor $\widetilde{w}$ na \textcolor{red}{c}
i odchodzące od niego liście kolorujemy na przemian na \textcolor{Green}{a} i \textcolor{blue}{b}(jak wyżej)\newline
$\widetilde{w}$ ma nieparzyście wiele liści:\quad
Jak dla przypadku parzystego. Jeżeli ostatni pokolorowany wierchołek ma kolor \textcolor{Green}{a} lub \textcolor{red}{c},
to nasze liście kolorujemy na przemian na \textcolor{blue}{b} i \textcolor{red}{c} -zaczynając i kończąc na \textcolor{blue}{b} z nieparzystości(jak wyżej).
Jeżeli ostatni pokolorowany wierzchołek ma kolor \textcolor{blue}{b}, to zmieniamy kolor $\widetilde{w}$ na \textcolor{red}{c},
i odchodzące od niego liście kolorujemy na przemian na \textcolor{Green}{a} i \textcolor{blue}{b} (zaczynając i kończąc na \textcolor{Green}{a})(jak wyżej)\newline
\end{document}