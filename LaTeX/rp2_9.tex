\documentclass{article}
\usepackage[utf8]{inputenc}
\usepackage{polski}
\usepackage{amsmath}
\usepackage{anysize}
\usepackage{amssymb}
\usepackage{bbm}
\begin{document}

Wiktor Zuba 320501 grupa 3
\newline

Zadanie 9.
\newline
\newline
Gęstość $X$ wyraż się wzorem $g_X(x)=e^{-x}\mathbbm{1}_{\{x\ge0\}}$.\newline
Pod warunkiem $X=x$ gęstość $Y$ wyraża się wzorem $g_{Y|X}(y|x)=xe^{-xy}\mathbbm{1}_{\{y\ge0\}}$.\newline
Rozkład łączny $X,Y$ ma gęstość $g_{X,Y}(x,y)=g_X(x)\cdot g_{Y|X}(y|x)=xe^{-(y+1)x}\mathbbm{1}_{\{x,y\ge0\}}$.\newline
Rozkład $Y$ ma gęstość $\int\limits_{\mathbb{R}}g_{X,Y}(x,y)dx=\mathbbm{1}_{\{y\ge0\}}\int\limits_{0}^{\infty}xe^{-(y+1)x}dx=
\frac{1}{y+1}\mathbbm{1}_{\{y\ge0\}}\mathbb{E}Z$ (gdzie $Z\sim Exp(y+1))$ =$\frac{1}{(y+1)^2}\mathbbm{1}_{\{y\ge0\}}$\newline
Z twierdzenia z wykładu (V.11.) $\mathbb{P}(X\in A|Y)=\int\limits_{A}g_{X|Y}(x|Y)dx$ p.n.\newline
W przypadku z zadania $\mathbb{P}(X>a|Y)=\int\limits_{a}^{\infty}g_{X|Y}(x|Y)dx=\int\limits_{a}^{\infty}\frac{g_{X,Y}(x,Y)}{g_{Y}(Y)}dx=
\int\limits_{a}^{\infty}(Y+1)^2xe^{-(Y+1)x}\mathbbm{1}_{\{x>0\}}dx=\left[-((Y+1)x+1)e^{-(Y+1)x}\mathbbm{1}_{\{x>0\}}\right]_{a}^{\infty}$\newline
$=\left\{
\begin{array}{ccc}
((Y+1)a+1)e^{-(Y+1)a}&\text{dla}&a>0\\
1&\text{dla}&a\le0
\end{array}\right.$ p.n.

\end{document}