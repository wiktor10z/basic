\documentclass{article}
\usepackage[utf8]{inputenc}
\usepackage{polski}
\usepackage{amsmath}
\begin{document}

Wiktor Zuba 320501 grupa 4
\newline
Zadanie 3.
\newline



$
\iint\limits_{y>0,|x+y|\le 1}xe^{-y^{2}}
=
\iint\limits_{1\ge y>0,|x+y|\le 1}xe^{-y^{2}}
+
\iint\limits_{y>1,|x+y|\le 1}xe^{-y^{2}}
$
pierwsza funkcja jest całkowalna (ciągła i ograniczona, na ograniczonym przedziale), druga jest nieujemna,
każda spełnia założenia twierdzenia Fubiniego, możemy więc je rozdzielić na całkowanie w kolejności
$
\iint\limits_{1\ge y>0,|x+y|\le 1}xe^{-y^{2}}
+\iint\limits_{y>1,|x+y|\le 1}xe^{-y^{2}}
=
\int\limits_{0}^{1}
\bigl(
\int\limits_{-y-1}^{-y+1}
xe^{-y^{2}} dx
\bigr) dy
+
\int\limits_{1}^{\infty}
\bigl(
\int\limits_{-y-1}^{-y+1}
xe^{-y^{2}} dx
\bigr) dy
=
\int\limits_{0}^{\infty}
\bigl(
\int\limits_{-y-1}^{-y+1}
xe^{-y^{2}} dx
\bigr) dy
=
\int\limits_{0}^{\infty}
\bigl[
\frac{x^2 e^{-y^{2}}}{2}
\bigr]_{-y-1}^{-y+1} dy
=
\int\limits_{0}^{\infty} -2ye^{-y^{2}} dy
=
\bigl[
e^{-y^2}
\bigr]_{0}^{\infty}
=
-1
$
\newline
\newline




Zadanie 4.
\newline




$
\iiint\limits_{x>0,y>1,z>1} \frac{x dx dy dz}{1+(xyz)^4}
$
funkcja jest nieujemna więc spełnia założenia twierdzenia Fubiniego, można więc rozdzielić całkowanie w kolejności
\newline
$
\iint\limits_{y>1,z>1} \bigl(
\int\limits_{0}^{\infty} \frac{x dx}{1+(xyz)^4}
\bigr) dy dz
$
podstawiamy
$
t=x^2 \quad dt=2xdx
$
mamy więc
$
\iint\limits_{y>1,z>1} \bigl(
\frac{1}{2}\int\limits_{0}^{\infty} \frac{dt}{1+t^2 (yz)^4}
\bigr) dy dz
=
\frac{1}{2}\iint\limits_{y>1,z>1}
\frac{\Pi}{2 (yz)^2} dy dz
$
ponowne rozbicie
$
\frac{\Pi}{4}
\int\limits_{1}^{\infty} \bigl(
\int\limits_{1}^{\infty}
\frac{dy}{(yz)^2}
\bigr) dz
=
\frac{\Pi}{4}
\int\limits_{1}^{\infty}
\frac{dz}{z^2}
=
\frac{\Pi}{4}
$
\newline
\newline




Zadanie 5.
\newline



$
\iiint\limits_{0<x<1,0<x+y<1,0<z(x+y+1)<1} \frac{dx dy dz}{(x+y+1)^2}
$
funkcja jest nieujemna więc spełnia założenia twierdzenia Fubiniego, można więc rozdzielić całkowanie w kolejności
$
\int\limits_{0}^{1} \bigl(
\int\limits_{-x}^{1-x} \bigl(
\int\limits_{0}^{\frac{1}{x+y+1}}
\frac{dz}{(x+y+1)^2}
\bigr) dy
\bigr) dx
=
\int\limits_{0}^{1} \bigl(
\int\limits_{-x}^{1-x} \bigl(
\frac{dy}{(x+y+1)^3}
\bigr) dx
=
\int\limits_{0}^{1} \bigl[
- \frac{1}{2(x+y+1)^2}
\bigr]_{y=-x}^{y=1-x} dx
=
\int\limits_{0}^{1} (\frac{1}{2}-\frac{1}{8})
=
\frac{3}{8}
$


\end{document}
