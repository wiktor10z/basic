\documentclass{article}
\usepackage[utf8]{inputenc}
\usepackage{polski}
\usepackage{amsmath}
\usepackage{anysize}
\usepackage{amssymb}
\usepackage{eurosym}
\begin{document}

Wiktor Zuba 320501\newline

Sortowanie w $LogSpace$\newline

Dowód należenia do klasy poprzez pokazanie algorytmu opisującego
maszynę Turinga o taśmie roboczej długości logarytmicznej względem taśmy wejściowej.\newline

Na taśmie roboczej przechowuje 3 logarytmicznej wielkości wskaźniki na pola na taśmie wejściowej
(wystarczy wskazywanie na odpowiednie znaki \officialeuro).\newline
Pierwszy wskaźnik wskazuje na znak \officialeuro$ $ przed liczbą ostatnio wypisaną na taśmę wyjściową.\newline
Drugi wskażnik wskazuje na znak \officialeuro$ $ przed aktualnie znaną najmniejszą liczbą jeszcze nie wypisaną.\newline
Trzeci wskaźnik jest roboczy -- wskazuje liczbę, którą aktualnie porównujemy z innymi.\newline
W stanie będę pamiętać czy wskaźnik pierwszy na coś wskazuje czy jeszcze nie.\newline

Pseudokod wysokopoziomowy (niżej przetłumaczenie rzeczy prostych na trywialne):\newline
Ustawiam pierwszy wskaźnik na nieważny, drugi i trzeci wskażnik wskazują na pierwszą liczbę.\newline
W pętli:\newline
Porównuję liczby wskazywane przez znaczniki i jeżeli trzecia jest większa od pierwszej (lub równa i wskaźnik trzeci jest dalej niż wskaźnik pierwszy)
i mniejsza od drugiej, to drugi wskaźnik staje się kopią trzeciego.\newline
Przestawiam trzeci wskażnik tak, żeby wskazywał kolejną liczbę.
Kiedy trzeci wskaźnik dojdzie do końca słowa wypisuję na taśmę wyjściową liczbę wskazywaną przez drugi wskaźnik,
a następnie pierwszy wskaźnik staje się kopią drugiego.
Drugi wskaźnik wędruje na początek słowa i odczytuje kolejne liczby aż trafi na większą od wskazywanej przez pierwszy wskaźnik,
lub równą, ale dalej położoną w słowie. Wskaźnik trzeci staje się kopią drugiego i wracam do początku pętli.
Pętla kończy się gdy drugi wskaźnik wyjdzie poza słowo (może się to stać tylko w trakcie wyszukiwania słowa większego niż dopiero co wypisane).\newline

Zapis pseudokodu w formie programistycznej
($l[i]$ - liczba wskazywana przez wskaźnik $i$ $|$ $i1,i2,i3$ -- wskaźniki,
$i++$ -- przeniesienie wskaźnika tak aby wskazywał na następny znak różny od $0,1$):\newline\newline
$
i1=null,i2=\vdash,i3=\vdash;\newline
while(i2!=\dashv)\{\newline\hspace{1cm}
\hspace*{16pt} if((l[i3]<l[i2])\&\&((l[i3]>l[ii])||((l[i3]==l[i1])\&\&(i3>i1))))\{\newline
\hspace*{32pt} i2=i3;\newline
\hspace*{16pt}\}\newline
\hspace*{16pt} i3++;\newline
\hspace*{16pt} if(i3==\dashv)\{\newline
\hspace*{16pt} wypisz(l[i2]);\newline
\hspace*{16pt} i1=i2;\newline
\hspace*{16pt} i2=\vdash;\newline
\hspace*{16pt} while((l[i2]<l[i1])||((l[i2]==l[i1])\&\&(i2<=i1)))\{\newline
\hspace*{32pt} i2++;\newline
\hspace*{16pt}\}\newline
\hspace*{16pt} i3=i2;\newline
\hspace*{16pt} \}\newline
\}\newline
$
\newline

Uwagi:\newline
Jeżeli wskaźnik jest nulem, to liczba przez niego wskazywana jest mniejsza od wszystkich innych
(jednyny przypadek, to porównanie liczby ostatnio wypisanej jeżeli jeszcze nic nie wypisano).\newline
W stanie oprócz ważności pierwszego wskaźnika pamiętamy aktualnie wykonywaną instrukcję alogrytmu (pseudokodu),
plus to co jest potrzebne do zapamiętania w stanie podczas wykonywania podprocedur.\newline
Na taśmie roboczej pamiętamy 3 wskaźniki logarytmicznej długości plus dla ułatwienia wskaźnik głowicy taśmy wejściowej
plus to co jest potrzebne w aktualnie wykonywanej podprocedurze, łącznie pamięć logarytmiczna
(po skończeniu wykonywania podprocedury tylko te 4 wskaźniki zostają).\newline

Poprawność algorytmu:\newline
Liczby zostaną wypisane w kolejności niemalejącej --
wypisywane są tylko liczby na które wskazuje drugi wskaźnik.
Na początku pętli wskaźnik ten wskazuje na liczbę niemniejszą niż ostatnio wypisana, w trakcie przesuwania wskaźnika trzeciego,
liczba wskazywana może zmaleć, ale tylko na taką która również jest niemniejsza niż ostatnio wypisana.
Po wypisaniu liczby wskaźnik jest przestawiany, aż niezmiennik staje się prawdziwy.
Dodatkowo jeżeli istnieją dwie takie same liczby, to zostaną wypisane w taj samej kolejności w jakiej znajdują się na taśmie wejściowej
(z konstrukcji porównań).\newline
Każda liczba z taśmy wejściowej zostanie wypisana dokładnie raz --
wskaźnik drugi nie może wskazywać tego samego co pierwszy poza fragmentem na końcu pętli,
co w połączeniu ze stabilnością daje niemożliwość wypisania dwa razy tej samej liczby (przy rozróżnieniu liczb o tej samej wartości).
Każdy element musi zostać wypisany, gdyż gdyby się tak nie stało,
to w odpowiednim kroku algorytmu musiałby zostać wybrany element większy od niego,
co przeczy poprawności procedury przejścia trzeciego wskaźnika po słowie z porównaniami.\newline

Szczegóły implementacyjne podprocedur:\newline
Porównanie wskaźników
(wskaźniki są binarnym zapisem ilości liter przed nimi w słowie z wiodącymi zerami długości dokładnie $\lceil\log(n)\rceil$)
-- idąc po taśmie roboczej porównuje bit po bicie oba zapisy wskaźników (korzystając z dodatkowej pamięci $loglog(n)$), wynik w stanie.\newline
Skopiowanie jednego wskaźnika na drugi -- przepisanie bit po bicie (korzystacjąc z dodatkowej pamięci $loglog(n)$).\newline
Przesunięcie wskaźnika $i$ aby wskazywał na następną liczbę -- przeniesienie głowicy aż jej wskaźnik będzie równy wskaźnikowi $i$,
przesunięcie głowicy w prawo do napotkania znaku \officialeuro, skopiowanie wskaźnika głowicy na wskaźnik $i$.\newline
Porównanie wskazywanych liczb -- najpierw porównanie ich długości, a następnie w przypadku równej długości porównanie bit po bicie
(dodatkowa pamięć $log(n)$), wynik w stanie.\newline
Wypisanie wskazywanej liczby -- przesunięcie głowicy na początek tej liczby i przepisanie bit po bicie na taśmę wyjściową.\newline
Sprawdzanie i ustawianie wskaźników na końce słowa -- przy pomocy głowicy.
\end{document}