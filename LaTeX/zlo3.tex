\documentclass{article}
\usepackage[utf8]{inputenc}
\usepackage{polski}
\usepackage{amsmath}
\usepackage{anysize}
\usepackage{amssymb}
\usepackage{eurosym}
\usepackage[pdftex]{graphicx}
\begin{document}

Wiktor Zuba 320501\newline

$PP$ zupełność MAJ-EVAL\newline

MAJ-EVAL $\in PP$\newline

Zakładam, że w obwodzie występują tylko bramki logiczne OR, AND i NOT.\newline
Słowem wejściowym do maszyny jest kod obwodu w postaci ciągu krotek opisujących fragmenty obwodu
(indeks bramki,rodzaj,lista indeksów bramek wejściowych) ułożonego warstwami od dołu (wszystko odpowiednio pooddzielane, indeksy zapisane binarnie).\newline
Taśma robocza zawiera ciąg wartości kolejno ponumerowanych bramek w obwodzie (również od dołu) -- inicjalnie pusta.\newline

Maszyna w trakcie działania najpierw przepisuje otrzymane bity losowe na pierwsze pozycje na taśmie roboczej (pozycje bramek wejściowych obwodu),
a następnie czytając wejście zapamiętuje w stanie rodzaj operacji i przesuwa głowicę po pozycjach na liście indeksów aż ustali wartość
(nie trzeba pamiętać wartości bramek wejściowych, a jedynie czytać albo aż napotka się 1 w przypadku OR, 0 w przypadku AND albo końca listy),
i zapisuje wartość pod wskazanym adresem.

Gdy maszyna skończy czytać słowo odczytuje wartość z ostatniej pozycji
(zawierającej wartość bramki wyjściowej obwodu) i przechodzi do stanu akceptującego lub odrzucającego dla wartości 1 lub 0 odpowiednio.\newline

Redukcja jest wielomianowa -- długość słowa wejściowego jest logarytmicznie większa od wielkości obwodu,
maszyna działa w czasie wielomianowym od wejścia -- każda krotka jest obsługiwana raz --
czas obsługi jednej krotki jest liniowy ze względu na jej krotność razy czas odszukania bramki na taśmie roboczej (liniowy ze względu na ilość bramek).\newline

(Maszyna otrzymana w redukcji jest ogólna dla problemu zaś konkretny obwód podawany jest jako wejście. 
Redukcję można przeprowadzić nieco inaczej -- kodując obwód w stanach i funkcji przejścia,
jednak ta redukcja byłaby o wiele brzydsza, a nawet trudna do spójnego opisania.)\newline\newline\newline

MAJ-EVAL -- $PP$-trudne\newline

Mając daną maszynę Turinga (działającą w czasie ograniczonym przez wielomian $P(n)$) i słowo na wejściu buduję obwód logiczny.\newline
W pierwszej warstwie znajduje się wejście do obwodu (bity losowe maszyny)
+ bramki zawierające binarne reprezentacje kolejnych znaków wejścia (wartości stałe łatwo otrzymać przy pomocy bramek wejściowych).\newline

W kolejnych warstwach zapamiętywany jest stan taśmy roboczej w kolejnych momentach wykonania.
Pojedyncze pole taśmy roboczej zapisane jest przy pomocy kilku bramek --
(bramki na litery -- logiczny zapis czy ta litera znajduje się w tym miejscu + bramka zapamiętująca czy głowica znajduje się w tym miejscu),
dodatkowo osobno w warstwie znajdują się bramki reprezentujące stan (jedna bramka dla każdego możliwego stanu)
+ miejsce głowicy na taśmie wejściowej (jedna bramka dla każdej pozycji na taśmie wejściowej).
Dodatkowo aby poruszać się tylko w obrębie dwóch sąsiednich warstw można dodać bramki na taśmę wejściową -- zawsze kopiować poprzednią warstwę.\newline\newline
Pomiędzy poziomami znajdują się dodatkowe bramki służące wykonaniu funkcji przejścia (bardziej skomplikowanych niż AND,OR,NOT).\newline
Wyliczana tam jest litera znajdująca się aktualnie pod głowicą
(jedna dla taśmy wejściowej, jedna dla roboczej) wyliczana jest ona przy pomocy funkcji --
dla każdej bramki oznaczającej binarnie czy litera jest pod głowicą i której wynikiem jest OR po pozycjach taśmy funkcji
(AND po bramce zapamiętującej czy głowica znajduje się w tej pozycji i po bramce pamiętającej czy ta litera znajduje się w tej pozycji
-- jak w zadaniu domowym 2).\newline
Mając stan i litery pod głowicami wyliczane są wartości kolejnych bramek -- bramki oznaczające przesunięcie pozycji konkretnych głowic +
bramki oznaczające literę zapisywaną pod głowicą na taśmie roboczej w kolejnym stanie maszyny + nowy stan
(podobwód to wyliczający jest zależny od kodu maszyny -- reprezentuje on funkcję przejścia w maszynie i jest wielkości wielomianowej względem kodu maszyny).\newline

Kolejna warstwa wyliczana jest na podstawie poprzedniej i wyników funkcji pośrednich:\newline
Pozycja głowicy jest OR-em po 3 AND-ach binarnych (zmiana w prawo/lewo/zostaje -- wyznaczone przez podobwód funkcji przejścia
+ czy głowica znajdowała się na odpowiedniej pozycji w warstwie poprzedniej). Na taśmie roboczej litera jest równa starej jeśli bramka głowicy miała wartość 0,
i równa literze wyznaczonej przez podobwód funkcji przejścia gdy miała wartość 1.\newline
Nowy stan jest wyliczany w podobwodzie funkcji przejścia.\newline

Obwód ma $P(n)$ warstw -- tak żeby wystarczyło na obliczenia maszyny w każdym biegu na słowie wejściowym.

Pozostaje wygenerować wartość bramki wyjściowej -- jest to OR po bramkach reprezentujących stan akceptujący we wszystkich warstwach
(maszyna się nie pętli i nie może wyjść ze stanu akceptującego czy odrzucającego,
dlatego nie ma możliwości aby weszła do stanu odrzucającego, a potem akceptującego).\newline

Jako, że obwód dokładnie symuluje działanie maszyny poprawność redukcji wynika prosto z jej konstrukcji.\newline
Wielomianowość redukcji (w tym przypadku wyznaczana przez wielkość obwodu) wynika z tego, że występuje w nim
$P(n)$ warstw głównych -- złożonych z bramek o ilości ograniczonej przez
(rozmiar alfabetów+1)*(długość taśmy wejściowej+roboczej)+(ilość możliwych stanów).
Warstwy pośrednie mają rozmiar (stała)*(rozmiary alfabetów)+(stała)*(ilość możliwych stanów)*(rozmiar alfabetów+1) --
wykorzystywane na poznanie litery pod głowicą + wyznaczenie konkretnego argumentu funkcji przejścia
(mając znany argument bezpośrednio wyznaczana jest nowa litera, stan i przesunięcia,
zaś prawdziwy wynik funkcji jest OR-em po wyjściach dla wszystkich możliwych argumentów).

























\end{document}