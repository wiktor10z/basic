\documentclass{article}
\usepackage[utf8]{inputenc}
\usepackage{polski}
\usepackage{amsmath}
\usepackage{anysize}
\marginsize{2,5cm}{2,5cm}{1cm}{4cm}
\begin{document}

Wiktor Zuba 320501
\newline
Zadanie 5.
\newline
\newline
$
\pi_i$ -długość i-tego wiersza diagramu, $\pi'_k$ -wysokość k tej kolumny.\newline
Każdy element w diagramie ma jednoznaczne współrzędne $\langle i,k\rangle$\newline
$
\sum\limits_{i}(i-1)\pi_i
=
\sum\limits_{i=1}^{n}(i-1)\pi_i
=
\sum\limits_{i=1}^{n}\left(\sum\limits_{k=1}^{\pi_i}(i-1)\right)
=
\sum\limits_{k=1}^{n}\left(\sum\limits_{i=1}^{\pi'_k}(i-1)\right)
=
\sum\limits_{k=1}^{n}\left(\frac{(\pi'_{k}-1)(\pi'_{k})}{2}\right)
=
\sum\limits_{k=1}^{n}{\pi'_{k}\choose 2}
=
\sum\limits_{i}{\pi'_{i}\choose 2}
$
Zmiana kolejności sumowania -ponieważ elementów jest skończenie wiele 
(z jednoznaczności współrzędnych każdy element w obu sumowaniach liczony dokładnie raz), oraz wyrażona jest ta sama funkcja
(przypisująca elementowi na diagramie wartość równą numerowi wiersza w którym występuje pomniejszonemu o 1),
pozostałe przejścia oczywiste.
\end{document}
