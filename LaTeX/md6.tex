\documentclass{article}
\usepackage[utf8]{inputenc}
\usepackage{polski}
\usepackage{amsmath}
\usepackage{anysize}
\marginsize{2,5cm}{2,5cm}{1cm}{4cm}
\begin{document}

Wiktor Zuba 320501
\newline
Zadanie 6.
\newline
\newline
Dla dowolnego k$>$1 znaleźć rozpinający k-dzielny podgraf H grafu G t.że $\forall_{v\in V[G]}\deg_H(v)\ge\frac{k-1}{k}\deg_G(v)$\newline\newline
Dla k $\ge|V[G]|$ H=G.\newline\newline
Dla k $\in(1,|V[G]|)$:\newline
Dzielimy wszystkie wierchołki grafu G na k podzbiorów (dowolnie ale niepustych), w każdej takiej otrzymanej grupce wierchołków
(grupki rozpatruje jako podzbiory, a nie grupy algebraiczne)
niektóre są ze sobą połączone krawędziami z grafu G, są też one połączone z niektórymi wierchołkami z poza grupki (liczby te mogą oczywiście wynosić 0).\newline
Niech m oznacza łączną liczbę krawędzi pomiędzy wierzhołkami należącymi do tej samej grupki.\newline\newline
Następnie indukcyjnie:\newline
Jeżeli dla pewnego wierzchołka istnieje taka grupka, w której znajduje się mniej wierchołków z nim sąsiadujących niż w jego własnej, to przenosimy go do tej grupki.\newline
Po takiej operacji liczba krawędzi "wewnątrz grupkowych" zmniejszy się o conajmniej 1 (w nowej grupce przenoszonego wierzchołka przybędzie mniej krawędzi wewnętrznych
niż ubędzie w starej, gdyż z założenia przeniesiony wierzchołek ma w niej mniej sąsiadów), po takiej akcji liczba grupek pozostanie taka sama,
ponieważ jeżeli stara grupa była złożona wyłącznie z jednego wierzchołka, to miał on w niej 0 sąsiadów, więc nie może w żadnej grupce mieć ich mniej.\newline
Ponieważ po takim kroku liczba m zmniejsza się, a jednocześnie musi pozostać nieujemna tak więc takich akcji wykonamy conajwyżej 
m$_0$(czyli m dla podziału startowego) tak więc indukcja kiedyś się skończy.\newline\newline
Po wykonaniu tych przenoszeń otrzymany podział grafu na k części ma taką właściwość, że dla dowolnego wierzhołka $v$ liczba jego sąsiadów ze swojej grupy
jest niewiększa niż liczba sąsiadów w dowolnej z pozostałych (k-1) grup, czyli:\newline
(k-1)$\cdot$(liczba sąsiadów $v$ w grupie)$\le$(liczba sąsiadów $v$ z poza grupy)\newline
k$\cdot$(liczba sąsiadów $v$ w grupie)$\le$(liczba wszystkich sąsiadów $v$)=$\deg_G(v)$\newline
(liczba sąsiadów $v$ w grupie)$\le\frac{1}{k}\deg_G(v)$\newline
Oznacza to, że jeżeli usuniemy wszystkie krawędzie pomiędzy wierzchołkami z tej samej grupy, to otrzymamy
graf k-dzielny, zawierający wszystkie wierzchołki grafu G i jednocześnie dla dowolnego wierchołka $v$,
usuniemy $\le\frac{1}{k}\deg_G(v)$ jego krawędzi, czyli w otrzymanym grafie zostanie mu $\ge\frac{k-1}{k}\deg_G(v)$ krawędzi.\newline
Zatem ten graf k-dzielny spełnia wszystkie założenia zadania.
\end{document}
