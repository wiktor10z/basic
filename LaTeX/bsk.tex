\documentclass{article}
\usepackage[utf8]{inputenc}
\usepackage{polski}
\usepackage{amsmath}
\usepackage{anysize}
\usepackage{amssymb}
\usepackage{enumitem}
\marginsize{1,5cm}{2cm}{1cm}{3cm}
\begin{document}

Wiktor Zuba 320501 grupa 4
\newline\newline
(OR) Spowodowanie błędu w startowaniu systemu operacyjnego za pomocą protokołu TFTP
\begin{enumerate}[label*=\arabic*.]
  \item (OR) Przesłanie dodatkowego datagramu służącego do przerwania poprawnej komunikacji
  \begin{enumerate}[label*=\arabic*.]
    \item Sprawienie by któryś z komputerów uruchamiający lub uruchamiany wysłał dodatkową wiadomość
    \item (AND) Podszycie się pod któryś z komputerów
    \begin{enumerate}[label*=\arabic*.]
      \item (OR) Zdobycie numeru TID jednego z komputerów
      \begin{enumerate}[label*=\arabic*.]
	\item Otrzymanie takiego samego numeru TID przy pomocy analizy programu losującego te numery
	\item Otrzymanie numeru bezpośrednio z jednego z komputerów
	\item (AND) Otrzymanie numeru z datagramu wychodzącego
	\end{enumerate}
	  1.2.1.3.1 Podsłuchanie datagramu TFTP\newline
	  1.2.1.3.2 Wyekstraktowanie numeru TID z podsłuchanego datagramu
      \item (OR)Wysłanie wiadomości do drugiego z komputerów
      \begin{enumerate}[label*=\arabic*.]
	\item Wysłanie potwierdzenia odbioru paczki podczas gdy nie została odczytana przez właściwy protokół
	\item Wysłanie komunikatu o błędzie
      \end{enumerate}
    \end{enumerate}
  \end{enumerate}
  \item (OR) Zablokowanie przepływu datagramów między stronami
    \begin{enumerate}[label*=\arabic*.]
      \item Zablokowanie dostępu do sieci jednego z komputerów
      \item Spowodowanie braku wychodzenia datagramów z jednego z komputerów
      \item Spowodowanie braku odbioru datagramów w jednym z komputerów
      \item Zablokowanie przepływu w "wąskim gardle"  - na przykład na ruterze jednego z komputerów
    \end{enumerate}
  \item (OR) Wywołanie błędu protokołu TFTP (error)
  \begin{enumerate}[label*=\arabic*.]
    \item (OR) Wywołanie sytuacji w której nie można wykonać żądania TFTP
    \begin{enumerate}[label*=\arabic*.]
      \item Usunięcie lub zmiana nazwy któregoś pliku konfiguracyjnego
      \item Usnunięcie lub zmiana nazwy użytkownika
    \end{enumerate}
    \item (OR) Przesłanie dodatkowej złej (składniowo) paczki z informacjami
    \begin{enumerate}[label*=\arabic*.]
      \item Wysłanie paczki z innego komputera
      \item Spowodowanie w komputerze uruchamiajacym wysłania złej paczki
    \end{enumerate}
    \item (OR) Spowodowanie braku dostępnych surowców
    \begin{enumerate}[label*=\arabic*.]
      \item (OR) Spowodowanie przepełnienia dysku
      \begin{enumerate}[label*=\arabic*.]
	\item Wysłanie do komputera uruchamiającego lub uruchamianego dużej liczby danych zapychających dysk
	\item Zmniejszenie wielkości całkowitej dysku poprzez wywołanie błędu w odczycie jego wielkości lub zabranie części pamięci z komputera
      \end{enumerate}
      \item (OR) Spowodowanie odmowy dostępu w trakcie uruchamiania
      \begin{enumerate}[label*=\arabic*.]
	\item Wysłanie sztucznego sygnału odmowy dostępu z innego komputera
	\item Zmiana uprawnień w którymś z potrzebnych plików
      \end{enumerate}    
    \end{enumerate}
  \end{enumerate}
  \item (AND) Spowodowanie wystartowania systemu operacyjnego z pewnymi zmianami (aby stworzyć luki bezpieczeństwa lub utrudnić korzystanie z systemu)
  \begin{enumerate}[label*=\arabic*.]
    \item Znalezienie miejsc w plikach bootujących które odpowiadają za ważne pod względem ochrony informacje (jak ustawienie haseł)
    \item (OR) Wprowadzenie zmian zgodnie z wygenerowanymi "poprawkami"
    \begin{enumerate}[label*=\arabic*.]
      \item Nadpisanie plików lub protokołu startowania systemu na serwerze (komputerze uruchamiającym)
      \item (AND) Przejęcie komunikacji i przesłanie zmienionych danych (rozgałęzienie jak w 1.2.)
    \end{enumerate}
  \end{enumerate}
\end{enumerate}




\end{document}