\documentclass{article}
\usepackage[utf8]{inputenc}
\usepackage{polski}
\usepackage{amsmath}
\usepackage{anysize}
\usepackage{amssymb}
\begin{document}

Wiktor Zuba 320501 grupa 3
\newline

Zadanie 5.1.(Niepełne)
\newline
\newline
$\varphi_{XY}(t)=\mathbb{E}e^{itXY}=\int\limits_{\mathbb{R}^2}\frac{1}{2\pi}e^{itxy}e^{-\frac{x^2}{2}}e^{-\frac{y^2}{2}}d(x,y)$ (z twierdzenie Fubiniego)
$\int\limits_{-\infty}^{\infty}\frac{1}{\sqrt{2\pi}}\left(\int\limits_{-\infty}^{\infty}\frac{1}{\sqrt{2\pi}}e^{i(xt)y}e^{-\frac{y^2}{2}}dy\right)e^{-\frac{x^2}{2}}dx
=\int\limits_{-\infty}^{\infty}\frac{1}{\sqrt{2\pi}}\varphi_{Y}(xt)e^{-\frac{x^2}{2}}dx
=\int\limits_{-\infty}^{\infty}\frac{1}{\sqrt{2\pi}}e^{-\frac{x^2t^2}{2}-\frac{x^2}{2}}dx
=\int\limits_{-\infty}^{\infty}\frac{1}{\sqrt{2\pi}}e^{-\frac{x^2}{2}(1+t^2)}dx$ (zamiana zmiennych)
$=\int\limits_{-\infty}^{\infty}\frac{1}{\sqrt{1+t^2}}\frac{1}{\sqrt{2\pi}}e^{-\frac{z^2}{2}}dz
=\frac{1}{\sqrt{1+t^2}}$\newline
Gęstość rozkładu może być wyrażona przez całkę $C\cdot\int\limits_{-\infty}^{\infty}e^{-itx}\frac{1}{\sqrt{1+x^2}}dx$ (gdzie $C$ to stała normalizująca).\newline
$\varphi_{X^2}(t)=\mathbb{E}e^{itX^2}
=\int\limits_{-\infty}^{\infty}\frac{1}{\sqrt{2\pi}}e^{itx^2}e^{-\frac{x^2}{2}}dx
=\int\limits_{-\infty}^{\infty}\frac{1}{\sqrt{2\pi}}e^{-\frac{x^2}{2}(1-2it)}dx$ (zamiana zmiennych)
$\frac{1}{\sqrt{1-2t}}\int\limits_{-\infty}^{\infty}\frac{1}{\sqrt{2\pi}}e^{-\frac{z^2}{2}}dz=\frac{1}{\sqrt{1-2t}}$\newline
Dla rozkładu $\chi$ kwadrat $Z\sim \chi_1^2$ mamy:\newline
$\varphi_{Z}(t)=\int\limits_{-\infty}^{\infty}e^{itx}\frac{1}{\sqrt{2\pi}}x^{-\frac{1}{2}}e^{-\frac{x}{2}}dx
=\int\limits_{-\infty}^{\infty}\frac{1}{\sqrt{2\pi}}x^{-\frac{1}{2}}e^{-\frac{x}{2}(1-2it)}dx=\frac{1}{\sqrt{1-2it}}$\newline
Z jednoznaczności funkcji charakterystycznej względem rozkładu $X^2\sim \chi_1^2$\newline
$\varphi_{X/Y}(t)=\mathbb{E}e^{itX/Y}=
\int\limits_{\mathbb{R}^2}\frac{1}{2\pi}e^{it\frac{x}{y}}e^{-\frac{x^2}{2}}e^{-\frac{y^2}{2}}d(x,y)$ (z twierdzenie Fubiniego)
$\int\limits_{-\infty}^{\infty}\frac{1}{\sqrt{2\pi}}\left(\int\limits_{-\infty}^{\infty}\frac{1}{\sqrt{2\pi}}e^{i\frac{t}{y}x}e^{-\frac{x^2}{2}}dx\right)e^{-\frac{y^2}{2}}dy
=\int\limits_{-\infty}^{\infty}\frac{1}{\sqrt{2\pi}}\varphi_{X}(\frac{t}{y})e^{-\frac{x^2}{2}}dx
=\int\limits_{-\infty}^{\infty}\frac{1}{\sqrt{2\pi}}e^{-\frac{t^2+x^4}{2x^2}}dx=e^{-|t|}$\newline
Dla rozkładu Cauchy'ego $C\sim C(0,1)$ mamy:\newline
$\varphi_{C}(t)=\int\limits_{-\infty}^{\infty}e^{itx}\frac{1}{\pi(1+x^2)}dx=e^{-|t|}$\newline
Z jednoznaczności funkcji charakterystycznej względem rozkładu $X/Y\sim C(0,1)$\newline
$\varphi_{X^2-Y^2}(t)=\mathbb{E}e^{itX^2-Y^2}$ (z niezależności $X,Y$)
$\varphi_{X^2}(t)\cdot\overline{\varphi_{Y^2}(t)}=\frac{1}{\sqrt{1-2it}}\cdot\overline{\frac{1}{\sqrt{1-2it}}}=\left|\frac{1}{1-2it}\right|dx$\newline
Gęstość rozkładu $X^2-Y^2$ może być wyrażona przez całkę $C\cdot\int\limits_{-\infty}^{\infty}e^{-itx}\left|\frac{1}{1-2ix}\right|dx$ (gdzie $C$ to stała normalizująca),
zaś podzielenie przez 2 powoduje "zwężenie" rozkładu względem 0.\newline
$\varphi_{XY+UV}(t)=\mathbb{E}e^{it(XY+UV)}$ (z niezależności $XY,UV$)
$(\varphi_{XY}(t))^2=\frac{1}{1+t^2}$ wyliczając całkę $C\cdot\int\limits_{-\infty}^{\infty}e^{-itx}\frac{1}{1+x^2}dx$ (odwrócenie całki z rozkładu Cauchy'ego)
otrzymujemy $Ce^{-|t|}\Rightarrow XY+UV$ ma rozkład o gęstości $\frac{e^{-|x|}}{2}$
\end{document} 