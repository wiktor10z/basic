\documentclass{article}
\usepackage[utf8]{inputenc}
\usepackage{polski}
\usepackage{amsmath}
\usepackage{anysize}
\usepackage{amssymb}
\usepackage{bbm}
\newcommand{\sgn}{\operatorname{sgn}}
\marginsize{1,5cm}{2cm}{1cm}{3cm}
\begin{document}

Wiktor Zuba 320501
\newline

Zadanie 7.7.
\newline
\newline
$\mathbb{Z}_{K}=\mathbb{Z}[\sqrt{p}]=\{a+b\sqrt{p}:a,b\in\mathbb{Z}\}$ dla $p=4k+3$ lub $p=2$ oraz $\mathbb{Z}[\frac{1+\sqrt{p}}{2}]$ dla $p=4k+1$\newline
Dowód:$(x-a-b\sqrt{p})(x-a+b\sqrt{p})=x^2-2ax+a^2+b^2p$ jest wielomianem monicznym nad $\mathbb{Z}$ o pierwiastku $a+b\sqrt{p}\Rightarrow\mathbb{Z}[\sqrt{p}]\subset\mathbb{Z}_{K}$\newline
W drugą stronę $z=\frac{a}{m}+\frac{b}{n}\sqrt{p}\in\mathbb{Q}[\sqrt{p}]\backslash\mathbb{Z}[\sqrt{p}]$ (ułamki nieskracalne, $m>1$ lub $n>1$)
jeśli $z=\frac{a}{m}+\frac{b}{n}\sqrt{p}\in\mathbb{Z}_{K}$, to albo $m\neq n$ ($NWD=1$) (można załozyć, że $m<n$), wtedy $a+\frac{m}{n}\sqrt{p}\in\mathbb{Z}_{K}\Rightarrow\frac{m^2p}{n^2}\in\mathbb{Z}_{K}$,
a to należy do $\mathbb{Q}\backslash\mathbb{Z}$, albo $z=\frac{a+b\sqrt{p}}{n}$, który jest pierwiastkiem wielomianu $nx-a-b\sqrt{p}$, który nie jest moniczny, ale jest minimalny,
więc każdy wielomian który ma ten element za pierwiastek musi go zawierać jako czynnik (a więc nie może być moniczny), co daje sprzeczność.
\newline

Zadanie 7.8,9.
\newline
\newline
Dla $K=\mathbb{Q}[\sqrt{-p}]$ mamy $\mathbb{Z}_{K}=\mathbb{Z}[\sqrt{-p}]$ -- dowód identyczny jak w zadaniu 7.7.\newline
chcemy znaleźć takie $a,b,c,d\in\mathbb{Z}$, że $(a+b\sqrt{\pm p})(c+d\sqrt{\pm p})=1$, daje nam to równania:\newline
$ac\pm3bd=1,ad+bc=0\Rightarrow ac\pm3bd=1, b=-\frac{-ad}{c}\Rightarrow ac\mp 3\frac{ad^2}{c}=1\Leftrightarrow c^2\mp3d^2=e$ (gdzie $^e|c$)\newline
Dla $\sqrt{3}$ mamy $x^2-3y^2=1$ posiada nieskończenie wiele rozwiązań:\newline
($x=\pm\frac{1}{2}(a^n+b^n), y=\pm\frac{1}{2\sqrt{3}}(a^n-b^n)$, gdzie $a=2+\sqrt{3},b=2-\sqrt{3}$)\newline
Dla $\sqrt{-3}$ mamy $x^2+3y^2=e$, gdzie $|e|\le |x|\Rightarrow x^2\le |x|,3y^2\le |x|\Rightarrow x=\pm1,y=0$ lub $x=0,y=0$, a więc istenieją tylko 2 rozwiązania (1,-1).
\end{document}