\documentclass{article}
\usepackage[utf8]{inputenc}
\usepackage{polski}
\usepackage{amsmath}
\usepackage{anysize}
\usepackage{amssymb}
\newcommand{\sgn}{\operatorname{sgn}}
\marginsize{1,5cm}{2cm}{1cm}{3cm}
\begin{document}

Wiktor Zuba 320501
\newline

Zadanie 1.
\newline
\newline
Mając równanie : $x^2+y^2=4z+3$
rozważmy równanie modulo 4: $x^2+y^2=3$.\newline
Jako, że $0^2=0,1^2=1,2^2=0(\text{mod }4),3^2=1(\text{mod }4)$ $x^2+y^2\in\{0,1,2\}(\text{mod }4)$, a więc $x^2+y^2\neq3 (\text{mod }4)$ dla kazdych $x,y$,
a więc wyjściowe równanie diofantyczne nie posiada rozwiązań (gdyż inaczej rozwiązanie równania było by również rozwiązaniem równania modulo 4).
\newline

Zadanie 2. Na odwrocie.
\newline

Zadanie 3.
\newline
\newline
a) Rozpatrując równanie jednorodne $x_1+x_2+x_3=0$ wybieramy dwa niezależne od siebie rozwiązania $(1,-1,0)$ oraz $(1,0,-1)$ oraz rozwiązanie $(5,0,0)$ rówania wyjściowego.
Otrzymujemy formułę $(1,-1,0)n_1+(1,0,-1)n_2+(5,0,0)$, lub inaczej $\{(n_1+n_2+5,-n_1,-n_2):n_1,n_2\in\mathbb{Z}\}$ jako rozwiązanie ogólne równania $x_1+x_2+x_3=5$.\newline\newline
b) Postępując jak w poprzednim podpunkcie $(2,-1,0)$ oraz $(3,0,-1)$ są rozwiazaniami równania jednorodnego $x_1+2x_2+3x_3=0$, zaś $(6,0,0)$ jest rozwiązaniem równania wyjściowego.
Rozwiązanie ogólne równania $x_1+x_2+x_3=6$ przyjmuje postać $(2n_1+3n_2+6,-n_1,-n_2)$, gdzi $n_1,n_2\in\mathbb{Z}$.
\newline

Zadanie 4.
\newline
\newline
Mając dane równianie $x^2+y^2=z^2$ załóżmy, że NWD($x,y,z$)=1 (z czego wynika, że conajmniej 1 a więc dokładnie 2 z liczb są nieparzyste). Rozpatrzmy równanie modulo 4.
Gdyby $^2\nmid x,y$, oraz $^2\mid z$, to $2=x^2+y^2=z^2=0(\text{mod }4)$, co oznacza sprzeczność. Dla ustalenia uwagi $^2\mid x$, $^2\nmid y,z$.
$x^2=z^2-y^2=4\cdot\frac{z+y}{2}\frac{z-y}{2}$ (oba ułamki są całkowite ze względu na parzystość sumy i różnicy).
Wybierzmy taką liczbę całkowitą $c$, że $\frac{z+y}{2c}$ oraz $\frac{z-y}{2c}$ są kwadratami (na przykład $c$=NWD($\frac{z+y}{2},\frac{z-y}{2}$)),
oraz liczby $a=\sqrt{\frac{z+y}{2c}},b=\sqrt{\frac{z-y}{2c}}$. Otrzymujemy $x=2abc,y=(a^2-b^2)c,z=(a^2+b^2)c$ dla $a,b,c\in\mathbb{Z}$, jako rozwiązanie ogólne (gdyż $c$ może być przemnożone przez NWD($x,y,z$)),
nie jest ono jednak jednoznaczne (trzeba by dodatkowo założyć, że $a,b\ge0$ oraz są względnie pierwsze).
\newline

Zadanie 5.
\newline
\newline
$3$ jest względnie pierwsze z $2^n$, co oznacza, że istnieją takie $a,b\in\mathbb{Z}$, że $3a+2^nb=\text{NWD}(3,2^n)=1$, co oznacza, że dla $x=a$ otrzymujemy $(3x-1)=0(\text{mod }2^n)$.
Podobnie $2$ jest względnie pierwsze z $p^n$ dla pierwszego $p>2$, co oznacza, że istnieją takie $c,d\in\mathbb{Z}$, że $2c+p^nd=\text{NWD}(2,p^n)=1$,
co oznacza, że dla $x=c$ otrzymujemy $(2x-1)=0(\text{mod }p^n)$, więc $(2x-1)(3x-1)=0(\text{mod }p^n)$ dla $p$ pierwszego.
Jednocześnie jedynymi rozwiązaniami $(2x-1)(3x-1)=0$ w $\mathbb{R}$ są $\frac{1}{2}$ i $\frac{1}{3}$, jednak skoro żadne z nich nie jest całkowite, to równanie diofantyczne nie posiada rozwiązań.
\newline

Zadanie 6.
\newline
\newline
Istnienie:\newline
Jeżeli wyjściowa funkcja $f$ spełnia warunek b), to $\overline{f}=f$, w przeciwnym przypdku korzystamy z równości $x^p=x(\text{mod }p)$
i w ten sposób obniżamy stopnie składników wielomianu $f$, tak by spełnione było załośenie b).\newline
Jednoznaczność:\newline
Załóżmy, że istnieją dwa wielomiany $\overline{f},\widetilde{f}$ spełniające warunki. Udowodnie, że wtedy $\overline{f}-\widetilde{f}=0$.
Indukcyjnie po liczbie zmiennych: dla wielomianu jednej zmiennej istnieje tylko jeden wielomian stopnia $<p$, który zeruje się w $p$ miejscach.
Krok indukcyjny: mając wielomian $g$ $n$ zmiennych $p-1$ razy stosujemy "sztuczkę"- wybieramy $x_n=0$ - otrzymując wielomian $n-1$ zmiennych z kroku indukcyjnego równy $0$.
Mamy teraz $g=g'x_n$ zakładając, że $x_n\neq 0$ (tracąc w ten sposób $p^{n-1}$ miejsc zerowych z pośród $p^n$ co nie jest istotne gdyż będziemy ten krok wykonywać tylko $p-1$ razy)
możemy usunąć to $x_n$ otrzymując $g'$ zerujące się w $p^n-p^{n-1}$ miejscach. Podstawiamy $x'_n=x_n-1$ (pozbywając się warunku $x_n\neq0$ - znów możemy to zrobić tylko $p-1$ razy)
wracamy do punktu wyjścia z wielomianem $g'$ o jeden stopień niższym- wykonując ten krok $p-1$ razy otrzymujemy $g=0$.
\end{document}