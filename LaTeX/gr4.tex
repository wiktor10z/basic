\documentclass{article}
\usepackage[utf8]{inputenc}
\usepackage{polski}
\usepackage{amsmath}
\usepackage{anysize}
\usepackage{amssymb}
\begin{document}

Wiktor Zuba 320501
\newline

Zadanie 4.2.
\newline
\newline
$[X,Y]_f=X(Yf)-Y(Xf)=X(\sin(x)\frac{\partial f}{\partial x}+xy\frac{\partial f}{\partial y})-Y(2x\frac{\partial f}{\partial x}+y^2\frac{\partial f}{\partial y})=
2x\cos(x)\frac{\partial f}{\partial x}+2xy\frac{\partial f}{\partial y}+2x^2y\frac{\partial^2 f}{\partial x\partial y}+\sin(x)y^2\frac{\partial^2 f}{\partial x\partial y}+
xy^2\frac{\partial f}{\partial y}-2\sin(x)\frac{\partial f}{\partial x}-\sin(x)y^2\frac{\partial^2 f}{\partial x\partial y}-2x^2y\frac{\partial^2 f}{\partial x\partial y}-
2xy^2\frac{\partial f}{\partial y}=\newline
(2x\cos(x)-2\sin(x))\frac{\partial f}{\partial x}+(2xy-xy^2)\frac{\partial f}{\partial y}\newline
\omega([X,Y])=-x(2x\cos(x)-2\sin(x))dx(\frac{\partial}{\partial x})+y((2xy-xy^2))dy(\frac{\partial}{\partial y})=
2x^2\cos(x)-2x\sin(x)+2xy^2-xy^3\newline
i_X\eta(Z)=\eta(X,Z)=(-xy^3dx+2x^2ydy)(Z)\newline
\omega\wedge i_X\eta=(-2x^3y+xy^4)dx\wedge dy\newline
d\omega=(\frac{\partial -x}{\partial y}-\frac{\partial -y}{\partial x})dx\wedge dy=0$
\newline

Zadanie 4.3.
\newline
\newline
Dla gładkich $\omega$ i $X$ $i_X\omega$ jest gładka z definicji gładkości. Przekształcenie jest różnowartościowe, ponieważ dla $i_X\omega=i_Y\omega$ $i_{X-Y}\omega=0$
z $n$ liniowości formy $\omega$, a więc dla $X\neq Y$ przeczyłoby to założeniu o niezerowości.
Przekształcenie jest na ponieważ przeprowadza przestrzeń $n$ wymiarową na $n-1$ wymiarową, oraz ma zerowe jądro.
\newline

Zadanie 4.4.
\newline
\newline
Wiązka orientowalna jest trywialna, ponieważ posiada 1 przekrój (liniowo niezależny = niezerowy), jest więc izomorficzna z produktową, gdyż rzutowanie jest
wyznaczone na $\mathbb{R}$ jednoznacznie, oraz dyfemorfizm na wiązkę produktową spełnia założenia.\newline
Wiązka trywialna jest orientowalna ponieważ wiązka produktowa rangi 1 jest orientowalna, zaś izomorfizm zachowuje tą orientację.
\end{document}