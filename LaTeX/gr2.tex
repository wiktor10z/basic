\documentclass{article}
\usepackage[utf8]{inputenc}
\usepackage{polski}
\usepackage{amsmath}
\usepackage{anysize}
\usepackage{amssymb}
\begin{document}

Wiktor Zuba 320501
\newline

Zadanie 2.2.
\newline
\newline
$A=F^{-1}(0,1)$ jest wyznaczone przez równiania $y=-x^2,x^4+s^2+t^2=1$ (taka zdeformowana sfera).
Jeśli przeskalujemy współrzędną $x$ pierwiastkowo (punktowi $x_0$ przyporządkowujemy punkt $sgn(x_0)\cdot\sqrt{|x_0|}$)
otrzymamy w ten sposób $A={y=-|x|,x^2+y^2+z^2=1}$ (mniej zdeformowana sfera) - na tym można już łatwo zdefiniować przejście $f(x,y,s,t)=(x,s,t)$
Złożnie dwóch funkcji klasy $C^{r}$ jest też taka funkcją, ich odwrotnosci również są tej klasy - funkcja przejścia miedzy dwoma rozmaitościami
jest różnowartościowa i na (zdefiniowana przez $g(x,y,s,t)=(x\cdot|x|,s,t)$), a więc są one dyfeomorficzne
(skoro przekształcenie miedzy nimi jest klasy $C^1$, to złożenia z mapami też są mapami conajmniej $C^1$).
\newline

Zadanie 2.3.
\newline
\newline
Mnożenie macierzy wyrażane jest wzorem zawierającym dla kazdej komórki wyłącznie sumy iloczynów (wielomiany), a więc jest przekształceniem gładkim.\newline
Macierzą odwrotną do macierzy ortogonalnej jest jeje transpozycja, która jest gładkim przekształceniem macierzy.\newline
Ponieważ zbiór tych macierzy jest ograniczony równaniami $\forall_i \sum\limits_{j=1}^{n}a_{ij}^2=1,\forall_{i\neq k}\sum\limits_{j=1}^{n}a_{ij}a_{ik}=0$,
to jeżeli jest on rozmiatością, to spodzeiwamy sie, że o strukturze $\mathbb{R}^{\frac{n(n-1)}{2}}$.
Narzucającym się wyborem mapy jest przypisanie kolejnym zmiennym w $\mathbb{R}^{\frac{n(n-1)}{2}}$ wartości komórek w górnym trójkącie macierzy (bez przekątnej),
jest ono przekształceniem gładkim. Wyznaczenie macierzy odwrotnej odbywałoby się "rekurencyjnie" -> mamy $a_{11}^2=1-\sum\limits_{j=2}^{n}a_{1j}^2=c_1$,
następnie mając $a_11$ otrzymujemy równania $a_{11}*a_{21}+a_{12}a_{22}=c_2,a_{21}^2+a_{22}^2=c_3$ za każdym razem otrzymując przecięcie sfery z prostą
(jako przecięciem przestrzeni wymiaru n niezależnych liniowo), co za każdym razem daje nam conajwyżej 2 możliwości
(i conajmniej 1 o ile wystartowaliśmy z poprawnej macierzy) - tutaj wprowadzamy binarne rozróżnienie map -> pierwsza grupa to ta,
dla której wybieramy punkt leksykograficznie mniejszy, zaś druga leksykograficznie większy - kontynuując podział wynikający z kolejnych kroków algorytmu.\newline
przekształcenie jest gładkie, i atlas pokrywa całą rozmaitość, pozostaje sprawdzić, czy mpay się zgadzają na swoim przecięciu
(dwie mapy się przecinają tam, gdzie w $k$-tym kroku algorytmu mamy wybór jednostkowy (chodzi o ostatni taki wybór)
- to są te dwie mapy które powstały w $k$-tym kroku podziału atlasu)
przekształcenie pomiędzy tymi mapami jes gładkie, ponieważ jest identycznością (mapy różnią się tylko tym, że ich odwroności zawierają funkcje (min/max)).
\newline

Zadanie 2.?.
\newline
\newline


\end{document}