\documentclass{article}
\usepackage[utf8]{inputenc}
\usepackage{polski}
\usepackage{amsmath}
\usepackage{anysize}
\usepackage{amssymb}
\begin{document}

Wiktor Zuba 320501\newline

Zadanie 6 (o distance automaton)\newline

Dla zadanego $n$ - liczba liczników i $m\in\mathbb{N}$ - limit.\newline

Wartością ścieżki jest tablica długości n o wartościach od 0 do m, której wartość powstaje poprzez
zaaplikowanie funkcji przejścia po krawędziach (nic , inkrementacja , reset), jeżeli zwiększamy element równy m,
to pozostaje on równy m(m oznacza $\ge m$ -
prawdziwa wartość nas nie interesuje gdyż jedynym sposobem na zmniejszenie wartości jest jej wyzerowanie, zaś m jest już przekroczeniem limitu).\newline

Konstrukcja gry:\newline
Definiuję grę odległości częściowo analogicznie do gry zdefiniowanej na wykładzie:\newline

Jak w normalnej grze w danym wierzchołku gracz Input wybiera literę, natomiast gracz Automat wybiera podzbiór krawędzi wychodzących
ze stanów automatu pamiętanych w wierzchołku tworząc nowy stan.\newline

W przeciwieństwie do klasycznej gry po przeczytaniu $k$ liter nie
ma jednej najlepszej ścieżki do wierzchołka (łatwy przykład), dlatego wybieram kilka ścieżek dojścia do stanu,
a żeby nie musieć pamietać wszystkich tych scieżek wystarczy że w wierzchołku gry zapamiętuję wartości tych ścieżek.
Ponieważ dowolny licznik może zostać zresetowany samo to, że w wierzchołku gry dla któregoś stanu występuje wartościowanie w
którym pewien licznik został przekroczony nie oznacza przegranej gracza Automat (ścieżka o tej wartości może być prefiksem ścieżki poprawnej).
Prawdziwnym warunkiem wygranej gracza Input jest to żeby dla każdego stanu akceptującego we wszystkich pamiętanych wartościowaniach
przekroczony został któryś licznik (lub stan był nieosiągalny = brak wartościowań).
Ponieważ w tej grze wybór kilku ścieżek dojścia do stanu pomaga nie graczowi Input (jak w normalnej) a graczowi Automat
(chociaż jedna ścieżka jest dobra = wierzchołek nie jest przegrywający), to wybór wszystkich krawędzi jest dla niego zawsze najlepszy
(pozwala to uprościć grę).\newline

Zredukowana gra:\newline

Wierzchołkiem grafu gry należącym do gracza Input jest [funkcja ze stanów automatu w podzbiory wartościowań].
Z wierzchołków wychodzą krawędzie odpowiadające literom alfabetu do wierzchołków o funkcjach zdefiniowanych następująco:\newline
mając funkcję $f$ dla poprzedniego wierzchołka i literę $a$ zaczynamy od funkcji $g$ stałej w zbiór pusty, a następnie dla każdej krawędzi
$p\xrightarrow{a}q$ należącej do automatu do wyniku $g(q)$ dodajemy teoriomnogościowo każdy element zbioru
$f(p)$ po zaaplikowaniu do niego funkcji przejścia po tej krawędzi (nic , inkrementacja , reset).
Wierzchołkiem startowym jest funkcja ze stanu początkowego automatu w zbiór jednoelementowy (wartościowanie zerowe) i pusty dla innych stanów.
Gracz Input wygrywa grę jeśli doprowadzi do wierzchołka o funkcji $f$, takiej że dla każdego stanu akceptującego $p$ $f(p)$
nie zawiera żadnego wartościowania w którym każda współrzędna jest $<m$, zaś gracz Automat wygrywa jeśli gra się nie kończy.
Gra jest zdeterminowana pozycyjnie - gracz Input wygrywa w atraktorze wierzchołków wygrywających
i rozwiązywalna w skończonym czasie (wierzchołków jest skończenie wiele).\newline

Jeśli gracz Automat wygrywa grę (ciężko mówić o posiadaniu strategii skoro nie należą do niego żadne wierzchołki),
to dla wybranego słowa po wybraniu przez gracza Input kolejnych jego liter w grze dochodzimy do wierzchołka w którym
zapamietane jest wartościowanie pewnej ścieżki akceptującej to słowo, która nie przekracza żadnego limitu, a więc taka ścieżka istnieje.\newline

Załóżmy, że wygrywa gracz Input. Ponieważ on w pełni decyduje o przebiegu rozgrywki, to istnieje dobrze zdefiniowana sekwencja liter
(tworząca słowo), której wybranie doprowadza do wierzchołka w którym nie jest zapamiętana żadna
wartość ścieżki akceptującej nieprzekraczająca limitu, a więc (ponieważ gracz Automat nie odrzuca żadnej ścieżki)
nie istnieje taka ścieżka akceptująca to słowo w automacie.












\end{document}