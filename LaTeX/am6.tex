\documentclass{article}
\usepackage[utf8]{inputenc}
\usepackage{polski}
\usepackage{amsmath}
\usepackage{anysize}
\marginsize{2,5cm}{2,5cm}{1cm}{4cm}
\begin{document}

Wiktor Zuba 320501 grupa 4
\newline

Zadanie 4.1.
\newline
\newline
$
\varphi(y)=\int\limits_{y}^{y+1}f(x)dx
=
\int\limits_{R}f(x)\cdot\chi_{[y,y+1]}(x)dx
$
$
x\in [y,y+1]\Leftrightarrow y\in [x-1,x]
$\newline
Całkowalność:
$
\int\limits_{R}\bigl|\int\limits_{R}f(x)\cdot\chi_{[y,y+1]}(x)dx\bigr|dy
\le
\int\limits_{R}\int\limits_{R}\bigl|f(x)\cdot\chi_{[y,y+1]}(x)\bigr|dxdy
$\newline
funkcja nieujemna więc z Fubiniego:
$
\iint\limits_{R^2}\bigl|f(x)\bigr|\cdot\chi_{[y,y+1]}(x)dxdy
=
\iint\limits_{R^2}\bigl|f(x)\bigr|\cdot\chi_{[x-1,x]}(y)dxdy
=
\int\limits_{R}\bigl|f(x)\bigr|\bigl(\int\limits_{x-1}^{x}dy\bigr)dx
=
\int\limits_{R}\bigl|f(x)\bigr|\cdot 1dx
$ ponieważ f całkowalna to i $\varphi$\newline
Równość(to samo tylko bez modułu):
$
\int\limits_{R}\varphi(y)dy
=
\int\limits_{R}\int\limits_{R}f(x)\cdot\chi_{[y,y+1]}(x)dxdy
=
\iint\limits_{R^2}f(x)\cdot\chi_{[x-1,x]}(y)dxdy
=
\int\limits_{R}f(x)\bigl(\int\limits_{x-1}^{x}dy\bigr)dx
=
\int\limits_{R}f(x)dx
$
\newline
\newline

Zadanie 4.2.
\newline
\newline
$
f(t)$-miara części zbioru A zawartej między płaszczyznami $z=\alpha t$ i $z=\beta t(0<\alpha<\beta),(t,z>0)
$\newline
czyli
$
f(t)=\iiint\limits_{(x,y,z)\in A,z\in(\alpha t,\beta t)}dxdydz
=
\iiint\limits_{A}\chi_{(R^2\times[\alpha t,\beta t])}(x,y,z)dxdydz
=
\iiint\limits_{R^3}\chi_{A}(x,y,z)\cdot\chi_{(R^2\times[\alpha t,\beta t])}(x,y,z)dxdydz
$
gdy funkcja charakterystyczna względem którejś zmiennej jest dla całego R, to tak naprawdę od tej zmiennej nie zależy(można pominąć przy pisaniu)
\newline
$
\int\limits_{0}^{\infty}f(t)dt
=
\int\limits_{0}^{\infty}\bigl(\iiint\limits_{R^3}\chi_{A}(x,y,z)\cdot\chi_{([\alpha t,\beta t])}(z)dxdydz\bigr)dt
$\newline
Funkcja nieujemna, więc z Fubiniego
$
\iiiint\limits_{R^3\times[0,\infty]}\chi_{A}(x,y,z)\cdot\chi_{([\alpha t,\beta t])}(z)dxdydzdt
$\newline
$
z\in[\alpha t,\beta t] \Leftrightarrow t\in[\frac{z}{\beta},\frac{z}{\alpha}]
$
$
\iiiint\limits_{R^3\times[0,\infty]}\chi_{A}(x,y,z)\cdot\chi_{([\frac{z}{\beta},\frac{z}{\alpha}])}(t)dxdydzdt
$
Ponownie Fubini
$
\iiint\limits_{R^3}\chi_{A}(x,y,z)\cdot\bigl(\int\limits_{0}^{\infty}\cdot\chi_{([\frac{z}{\beta},\frac{z}{\alpha}])}(t))dt\bigr)dxdydz
=
\iiint\limits_{R^3}\chi_{A}\bigl(\int\limits_{0}^{\infty}\chi_{[\frac{z}{\beta},\frac{z}{\alpha}]}dt\bigr)dxdydz
=
\iiint\limits_{R^3}\chi_{A}\bigl(\frac{z}{\alpha}-\frac{z}{\beta}\bigr)dxdydz
=
\bigl(\frac{1}{\alpha}-\frac{1}{\beta}\bigr)\iiint\limits_{A}zdxdydz
$
Wiemy, że wspólrzędna z środka ciężkości zbioru A jest równa c, $l_{3}(A)=m$\quad
$
\frac{1}{m}\cdot\iiint\limits_{A}zdxdydz=c
$
$
\bigl(\frac{1}{\alpha}-\frac{1}{\beta}\bigr)\iiint\limits_{A}zdxdydz
=
\bigl(\frac{1}{\alpha}-\frac{1}{\beta}\bigr)cm
$
\newline
\newline

Zadanie 4.3.
\newline
\newline
Pole powierzchni
$
M=\{(x,y,z):e^x+e^{-x}=z-\sqrt{3}y,0<y<x<1\}\quad
z=e^x+e^{-x}+\sqrt{3}y
$
 więc M jest wykresem funkcji (różniczkowalnej) o argumentach w $R^2$ i wartościach w R-jest więc rozmaitością wymiaru 2
$
\iint\limits_{z=e^x+e^{-x}+\sqrt{3}y,0<y<x<1}dS
$
 Stosujemy podstawienie 
$
x=ln{a},y=\frac{b}{\sqrt{3}},z=a+\frac{1}{a}+b\newline
G=det\bigl(
\left[\begin{array}{ccc}
\frac{1}{a}&0&1-\frac{1}{a^2}\\
0&\frac{\sqrt{3}}{3}&1
\end{array}\right]
\cdot
\left[\begin{array}{cc}
\frac{1}{a}&0\\
0&\frac{\sqrt{3}}{3}\\
1-\frac{1}{a^2}&1
\end{array}\right]
\bigr)
=
det\bigl(
\left[\begin{array}{cc}
1-\frac{1}{a^2}+\frac{1}{a^4}&1-\frac{1}{a^2}\\
1-\frac{1}{a^2}&\frac{4}{3}
\end{array}\right]
\bigr)
=
\frac{1}{3}\bigl(1+\frac{1}{a^2}\bigr)^2
$\newline
$
\frac{\sqrt{3}}{3}\iint\limits_{0<\frac{b}{\sqrt{3}}<ln{a}<1}(1+\frac{1}{a^2})dadb
$
funkcja nieujemna (po kilkunastu zadaniach wiadomo co)\newline
$
\frac{\sqrt{3}}{3}\int\limits_{1}^{e}da\int\limits_{0}^{\sqrt{3}ln{a}}db(1+\frac{1}{a^2})
=
\frac{\sqrt{3}}{3}\int\limits_{1}^{e}\sqrt{3}ln{a}(1+\frac{1}{a^2})da
=
\int\limits_{1}^{e}ln{a}(1+\frac{1}{a^2})da
$
teraz niejako powrót $a=e^c$
$
\int\limits_{0}^{1}c(1+e^{-2c})e^cdc
=
\int\limits_{0}^{1}c(e^c+e^{-c})dc
=
\bigl[c(e^c-e^{-c})\bigr]_{0}^{1}-\int\limits_{0}^{1}e^c-e^{-c}dc=e-e^{-1}-e-e^{-1}+2=2-\frac{2}{e}
$
\newpage

Wiktor Zuba 320501 grupa 4
\newline

Zadanie 4.4.
\newline
\newline
$
M=\{(x,y,z):2z=x^2+y^2<\sqrt{x}\}
$
M jest wykresem funkcji(różniczkowalnej) (od x i y) jest więc rozmaitością
$
\iint\limits_{M}\sqrt{\frac{z}{1+2z}}dS
$
podstawienie
$
x=rcos\alpha,y=rsin\alpha,z=\frac{r^2}{2}
$\newline
$
G=det\bigl(
\left[\begin{array}{ccc}
cos\alpha&sin\alpha&r\\
-rsin\alpha&rcos\alpha&0\\
\end{array}\right]
\cdot
\left[\begin{array}{cc}
cos\alpha&-rsin\alpha\\
sin\alpha&rcos\alpha\\
r&0\\
\end{array}\right]
\bigr)
=
det\bigl(
\left[\begin{array}{cc}
r^2+1&0\\
0&r^2\\
\end{array}\right]
\bigr)
=r^4+r^2
$\newline
$
\iint\limits_{r^2<\sqrt{rcos\alpha},\alpha\in(-\frac{\Pi}{2},\frac{\Pi}{2})}\sqrt{\frac{\frac{r^2}{2}}{1+r^2}}\sqrt{r^2+r^4}drd\alpha
=
\iint\limits_{0<r^\frac{3}{2}<\sqrt{cos\alpha},\alpha\in(-\frac{\Pi}{2},\frac{\Pi}{2})}\sqrt{\frac{r^2}{2}\frac{r^2+r^4}{1+r^2}}drd\alpha
=
\frac{\sqrt{2}}{2}\iint\limits_{0<r<\sqrt[3]{cos\alpha},\alpha\in(-\frac{\Pi}{2},\frac{\Pi}{2})}r^2drd\alpha
$
funkcja nieujemna...
$
\frac{\sqrt{2}}{2}\int\limits_{-\frac{\Pi}{2}}^{\frac{\Pi}{2}}d\alpha\int\limits_{0}^{\sqrt[3]{cos\alpha}}r^2dr
=
\frac{\sqrt{2}}{2}\int\limits_{-\frac{\Pi}{2}}^{\frac{\Pi}{2}}\bigl[\frac{r^3}{3}\bigr]_{0}^{\sqrt[3]{cos\alpha}}d\alpha
=
\frac{\sqrt{2}}{6}\int\limits_{-\frac{\Pi}{2}}^{\frac{\Pi}{2}}cos\alpha d\alpha
=
\frac{\sqrt{2}}{6}
$
\newline
\newline

Zadanie 4.5.
\newline
\newline
M-wykres funkcji $f:A=\{(x,y):0<y<x<1\}\rightarrow R\quad f(x,y)=\frac{y}{x}$\newline
Jest to funkcja różniczkowalna$(x>0)$, więc M jest rozmaitością różniczkową\newline
$
\iint\limits_{z=\frac{y}{x},0<y<x<1}ydS
=
\iint\limits_{y=xz,0<z<1,0<x<1}ydS
$
podstawienie $x=x, y=xz, z=z$\newline
$
G=det\bigl(
\left[\begin{array}{ccc}
1&z&0\\
0&x&1\\
\end{array}\right]
\cdot
\left[\begin{array}{cc}
1&0\\
z&x\\
0&1\\
\end{array}\right]
\bigr)
=
det\bigl(
\left[\begin{array}{cc}
1+z^2&xz\\
xz&1+x^2\\
\end{array}\right]
\bigr)
=
1+x^2+z^2
$\newline
$
\iint\limits_{0<x<1,0<z<1}xz\sqrt{1+x^2+z^2}dxdz
$
nieujemna-Fubini:
$
\int\limits_{0}^{1}z(\int\limits_{0}^{1}x\sqrt{1+z^2+x^2}dx)dz
=
\int\limits_{0}^{1}z\bigl[\frac{1}{3}(1+z^2+x^2)^\frac{3}{2}\bigr]_{0}^{1}dz
=
\frac{1}{3}\int\limits_{0}^{1}z((2+z^2)^\frac{3}{2}-(1+z^2)^\frac{3}{2})dz
=
\frac{1}{3}(\bigl[\frac{1}{5}(2+z^2)^\frac{5}{2}\bigr]_{0}^{1}-\bigl[\frac{1}{5}(1+z^2)^\frac{5}{2}\bigr]_{0}^{1})
=
\frac{1}{15}(3^\frac{5}{2}-2^\frac{5}{2}-2^\frac{5}{2}+1^\frac{5}{2})
=
\frac{1}{15}(9\sqrt{3}-8\sqrt{2}+1)
$

\end{document}