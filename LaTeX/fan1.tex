\documentclass{article}
\usepackage[utf8]{inputenc}
\usepackage{polski}
\usepackage{amsmath}
\usepackage{anysize}
\usepackage{amssymb}
\begin{document}
 
Wiktor Zuba 320501 grupa 3
\newline

Zadanie 7 b)
\newline
\newline
$
z=x+iy\newline
|\ctg{(\pi z)}|=
\frac{|\frac{1}{2}(e^{i\pi z}+e^{-i\pi z})|}{|\frac{-i}{2}(e^{i\pi z}-e^{-i\pi z})|}=
\frac{|e^{i\pi z}+e^{-i\pi z}|}{|e^{i\pi z}-e^{-i\pi z}|}=
\frac{|e^{i\pi z}+e^{-i\pi z}|}{|e^{i\pi z}-e^{-i\pi z}|}\cdot\frac{|e^{i\pi z}|}{|e^{i\pi z}|}=
\frac{|e^{2\pi iz}+1|}{|e^{2\pi iz}-1|}=
\frac{|e^{-2\pi y}e^{2\pi ix}+1|}{|e^{-2\pi y}e^{2\pi ix}-1|}\newline\newline
$Dla $ x=\pm(N+\frac{1}{2})\newline\newline
\frac{|e^{-2\pi y}e^{\pm(2N+1)\pi i}+1|}{|e^{-2\pi y}e^{\pm(2N+1)\pi i}-1|}=$[z $2\pi$ okresowości $e^z$]$=
\frac{|e^{-2\pi y}e^{\pm\pi i}+1|}{|e^{-2\pi y}e^{\pm\pi i}-1|}=
\frac{|-e^{-2\pi y}+1|}{|-e^{-2\pi y}-1|}=
\frac{|e^{-2\pi y}-1|}{|e^{-2\pi y}+1|}\le \underline{1}\newline
$[z własności $e^x\ge0$ dla $x\in \mathbb{R}$]$\newline\newline
$Dla $ y=\pm(N+\frac{1}{2})\newline\newline
\frac{|e^{\mp(2N+1)\pi}e^{2\pi ix}+1|}{|e^{\mp(2N+1)\pi}e^{2\pi ix}-1|}\le
$[z warunku trójkąta(2 razy)]$\le
\frac{|e^{\mp(2N+1)\pi}e^{2\pi ix}|+1}{||e^{\mp(2N+1)\pi}e^{2\pi ix}|-1|}=
\frac{e^{\mp(2N+1)\pi}+1}{|e^{\mp(2N+1)\pi}-1|}\newline
$ Dla $+(2N+1)\pi $ mamy $
\frac{e^{(2N+1)\pi}+1}{e^{(2N+1)\pi}-1}=
1+\frac{2}{e^{(2N+1)\pi}-1}\le
1+\frac{2}{e^{\pi}-1}\le
1+\frac{2}{23-1}=
\underline{\frac{12}{11}}\newline
$ Dla $-(2N+1)\pi $ mamy $
\frac{e^{-(2N+1)\pi}+1}{|e^{-(2N+1)\pi}-1|}=
\frac{e^{-(2N+1)\pi}+1}{|e^{-(2N+1)\pi}-1|}\cdot\frac{e^{(2N+1)\pi}}{e^{(2N+1)\pi}}
=
\frac{e^{(2N+1)\pi}+1}{e^{(2N+1)\pi}-1}\le\underline{\frac{12}{11}}\newline
$
Tak więc całość jest ograniczona przez stałą $\frac{12}{11}$, a właściwie przez $(1+\frac{2}{e^{\pi}-1})$ która to wartość jest przyjmowana np. dla $z=\frac{i}{2}$
\end{document}