\documentclass{article}
\usepackage[utf8]{inputenc}
\usepackage{polski}

\begin{document}

Wiktor Zuba 320501 grupa 4
\newline
Zadanie 1.
\newline





$
\lim\limits_{n \to \infty}
\int\limits_{0}^{n}
\frac{dx}{n+n^2 \sin{(n^{-2}x)}}
$
\newline
dla każdego n stosujemy podstawienie $y=\frac{x}{n}\quad dx=n\cdot dy$
\newline
z twierdzenia o całkowaniu przez podstawianie wynika,że
\newline
$
\forall_{n\in N}
\int\limits_{0}^{n}
\frac{dx}{n+n^2 \sin{(n^{-2}x)}}
=
\int\limits_{0}^{1}
\frac{n\cdot dy}{n+n^2 \sin{(n^{-1}y)}}
$
co przenosi się na granice
\newline
$
\lim\limits_{n \to \infty}
\int\limits_{0}^{n}
\frac{dx}{n+n^2 \sin{(n^{-2}x)}}
=
\lim\limits_{n \to \infty}
\int\limits_{0}^{1}
\frac{n\cdot dy}{n+n^2 \sin{(n^{-1}y)}}
=
\lim\limits_{n \to \infty}
\int\limits_{0}^{1}
\frac{dy}{1+n \sin{(n^{-1}y)}}
$
\newline
$\lim\limits_{n \to \infty} \frac{sin(n^{-1}y)}{n^{-1}y}=1 \Rightarrow \lim\limits_{n \to \infty} n\sin{(n^{-1}y)}=y$
\newline
granica punktowa tych funkcji podcałkowych$\quad \lim_{n \to \infty} \frac{dy}{1+n \sin{(n^{-1}y)}} =\frac{dy}{1+y}$


dla $n>1 \quad n \sin{(n^{-1}y)} > 0$ dla każdego n funkcja jest ograniczona przez funkcje stale równą 1 która jest całkowalna na [0,1]
spełnione są więc założenia twierdzenia Lebesgue'a o zbieżności zmajoryzowanej, więc zarówno funkcje jak i ich granica są całkowalne oraz 
\newline
$\lim\limits_{n \to \infty}
\int\limits_{0}^{1}
\frac{dy}{1+n \sin{(n^{-1}y)}}
=
\int\limits_{0}^{1}
\lim\limits_{n \to \infty}
\frac{dy}{1+n \sin{(n^{-1}y)}}
=
\int\limits_{0}^{1}
\frac{dy}{1+y}
=
\biggl[
\ln{(1+y)}
\biggr]_{0}^{1}
=
\ln{2}
$
\end{document}
