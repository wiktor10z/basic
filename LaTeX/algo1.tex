\documentclass{article}
\usepackage[utf8]{inputenc}
\usepackage{polski}
\usepackage{amsmath}
\usepackage{anysize}
\usepackage{amssymb}
\begin{document}

Wiktor Zuba 320501 group 1.
\newline\newline
a) Graph $G=({s,v,t},{(s,v),(v,t)}), c(s,v)=c(v,t)>0$ then both edges are decreasing, but none increasing.\newline\newline
b)
Consider residual network, after sending the flow found by the given algorithm. In the network there is no path from $s$ to $t$,
otherwise we could have sent flow through it and acquired flow of bigger value. If we now build subgraph by sending a BFS from s in residual graph,
and backward BFS from t (through edges backward), then if the edge leads from the first subgraph to the second then it is "increasing",
as if we turn its value to $>0$ then there would exist s-t path in the graph. Otherwise increasing capacity of the edge wouldn't give us any s-t path, so the edge is not "increasing".\newline
Algorithm:\newline
1.Find max-flow in O(mn).\newline
2.For every edges capacity subtract the value of the flow sent through the edge in the found flow (and increase of the dual edge) in O(m).\newline
3.a)send BFS from s over the positive value edges (we set vertexes flag to 1).\newline
3.b)send BFS backwards from t over positive value edges (we set vertexes flag to 2) sum in O(m+n).\newline
4.For every edge uv of capacity 0 we check if u's flag is 1 and v's is 2 then the path is increasing otherwise it is not. In O(m).\newline
Entire algorithm in time O(mn)\newline\newline
c)
Now in our residual network (after sending the found flow) the edge is decreasing only if its capacity is 0. Then edge $uv$ is decreasing if and only if there exists no u-v path.\newline
Proof: If there exist an u-v path then if we decrease the initial capacity of the edge by less then the minimum capacity of the path then we can send this
amount through that path not decreasing the value of the flow and thus acquiring maximum flow with edge uv of non 0 value in its residual network.\newline
Now assume there is no such path. In residual network there must be a t-s path which is going through vu (as there must be a part of the flow going from s to t through the uv).
If uv is not decreasing, then when we send some small(smaller then half of the smallest non 0 edge capacity) flow through this t-s path
then there would have to exist a s-t path as we would have to send that amount of the flow from s to t (omitting uv) to hold the maximal value of flow.\newline
Then having those two paths we can start from u on the way back to s till we cross the s-t path then proceed through it until we cross the t-s path again... till we get to v.
If we have chosen the s-t path shortest possible then there would be no "coming back" in terms of the (reversed) t-s path namely no ... is needed. Consequently we have found an u-v path,
and we assumed there was no such thus getting contradiction.\newline
Algorithm:\newline
1.Find max-flow in O(mn).\newline
2.For every edges capacity subtract the value of the flow sent through the edge in the found flow (and increase of the dual edge) in O(m).\newline
3.For every edge uv of capacity 0 we send DFS from u over positive value edges and if we don't get to v in the process then the edge is decreasing in O($m\cdot(m+n)$).\newline
Assuming n is O(m) we obtain O($m^2$) as a time of entire algorithm.







\end{document}
