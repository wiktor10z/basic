\documentclass{article}
\usepackage[utf8]{inputenc}
\usepackage{polski}
\usepackage{amsmath}
\usepackage{anysize}
\usepackage{amssymb}
\usepackage[pdftex]{graphicx}
\begin{document}

Wiktor Zuba 320501
\newline
Zadanie 2.2. (NP-zupełność "niezwykłej" gry parzystości)
\newline
\newline
NP:\newline
Jeżeli gracz $\exists$ posiada strategię wygrywającą, to posiada pozycyjną strategię wygrywającą (w przeciwieństwie do gracza $\forall$) -
dowód jak w zwykłej grze parzystości - wierzchołki odwiedzane $\infty$ wiele razy tworzą silnie spójną składową = kilka połączonych cykli.
Jeżeli gracz wygrywa na tym zbiorze cykli, to zbiór ten zawiera wierzchołek należący do zbioru $E_i$ i nie zawiera ze zbioru $F_i$, dla pewnego $i$,
więc jest to również prawda dla pewnego pojedyńczego cyklu z tego zbioru cykli (reszta dowodu jest już analogiczna do standardowego).\newline 
Wielomianowym świadkiem będzie strategia gracza $\exists$.\newline
Dla uproszczonego grafu (z usuniętymi krawędziami wychodzącymi z wierzchołków gracza $\exists$ nienależącymi do strategii)
w czasie wielomianowym sprawdzam czy gracz $\forall$ może wygrać startując z wierzchołka $v$:\newline
Dzielę podgraf osiągalny z $v$ na maksymalne silnie spójne składowe i sprawdzam każdą z nich osobno (wygrana jako OR po wygranych w składowych).\newline
Sprawdzam czy jeżeli inf jest równe wszystkim wierzchołkom w tej składowej, to gracz $\forall$ wygrywa, jeśli nie, to istnieje takie $i$, że
inf$\cap E_i\neq\emptyset$ i inf$\cap F_i=\emptyset$ - usuwam wierzchołki neleżące do $E_i$, znów dzielę (tę składową, bez $E_i$)
na maksymalne silnie spójne składowe itd. Algorytm ten jest $O(n^2)$, dla n - wielkości grafu (w każdym liniowym kroku usuwam conajmniej 1 wierzchołek).\newline
Poprawność algorytmu: jeżeli w algorytmie wyjdzie, że gracz $\forall$ może wygrać, to znajduje on silnie spójną składową (może nie maksymalną),
osiągalną z $v$ chodząc po której gracz wygrywa (oczywiście nie rozważam składowych wielkości 1, chyba, że jest to cykl do siebie samego).
Jeżeli algorytm wskazuje, że gracz $\forall$ nie może wygrać, to znaczy, że nie istnieje taka silnie spójna składowa
(niezależnie od wyboru (kolejności) odpowiedniego zbioru $E_i$ do usunięcia w całej składowej nie ma wierzchołków z $F_i$,
więc i tak nie będą one mogły być odwiedzane - zbiór $E_i$ będzie musiał być usunięty).\newline\newline
NP-trudność:\newline
Redukcja problemu SAT - dla każdej zmiennej binarnej $X_i$ występującej w formule CNF tworzymy wierzchołki $x_i$ i $\neg x_i$ oraz
dla każdej klauzuli tworzymy wierzchołek $Y_j$ jak na obrazku:\newline
\includegraphics[scale=0.9]{aatg_rys2.jpg}\newline
Gdzie wierzchołki okrągłe należą do gracza $\exists$, zaś kwadratowy do gracza $\forall$.
Dla każdego literału $a_k$ w klauzuli $Y_j$ dobieramy zbiory $E_{jk}=\{Y_j\},F_{jk}=\{\neg a_k\}$, gdzie $a_k=x_l$ lub $a_k=\neg x_l$.\newline
Gracz $\exists$ wygrywa w tej grze wtedy i tylko wtedy gdy istnieje wartościowanie spełniające formułę.\newline
Intuicyjnie gracz $\exists$ wybiera wartościowanie, zaś gracza $\forall$ wskazuje klauzulę która nie jest spełniona przy tym wartościowaniu.\newline
Jeśli istnieje wartościowanie spełniające formułę, to dla każdej zmiennej $x_i=1$ gracz $\exists$ przy wyborze wierzchołka $x_i$ lub $\neg x_i$ wybiera $x_i$
i odwrotnie w przeciwnym przypadku - wtedy niezależnie jaki zbiór klauzul nieskończenie często wybiera gracz $\forall$ któryś element każdej alternatywy jest prawdziwy
- istnieje zbiór $F_l$ o pustym przecięciu ze zbiorem inf, dla pewnego $E_l=\{Y_i\}$ odwiedzanego $\infty$ wiele razy
(czyli dla $Y_i$ wybieranego).
Jeżeli takie wartościowanie nie istnieje, to przy każdym ruchu gracz $\forall$ wybiera klauzulę niespełnianą,
przez ostatnio wybrane przez gracza $\exists$ wartościowanie.
Przy tej strategii gracza $\forall$ jeżeli klauzula $Y_i$ zostanie odwiedzona $\infty$ wiele razy, to również $\infty$ wiele razy zostanie wybrane wartościowanie
jej nie spełniające, czyli $\infty$ wiele razy zostaną wybrane wierzchołki przeciwne literałom należącym do tej klauzuli, a więc i wszystkie zbiory 
$F_l$ odpowiadające zbiorom $E_l=\{Y_i\}$ odwiedzanym. Ponieważ jest tak dla każdej klauzuli, to dla każdego odwiedzanego $\infty$ wiele razy zbioru $E_l$ rownież zbiór $F_l$
jest odwiedzany, czyli gracz $\forall$ wygrywa.\newline
Alternatywną strategią dla gracza $\forall$ jest wybieranie wszystkich klauzul w pętli (nie zmienia to analizy poprawności). 
\newpage
Zadanie 2.3. (liniowe rozwiązywanie nudnych gier parzystości)
\newline
\newline
Preprocessing:\newline
1. Znajduję maksymalne silnie spójne składowe grafu gry (bez zwracania uwagi na właściciela wierzchołka) w czasie liniowym.\newline
2. Dla każdej silnie spójnej składowej znajduję maksymalny priorytet
lub maksymalny priorytet na dowolnym cyklu w niej występującym i jego parzystość zapamiętuję dla tej składowej
(ponieważ gra jest nudna oba sposoby dadzą tą samą parzystość).\newline
3. Dla każdej silnie spójnej składowej zapamiętuję ilość krawędzi z niej wychodzących,
te silnie spójne składowe z których nie da się wyjść dodatkowo zapamiętuję (na przykład na kolejce).\newline
4. Dla każdego wierzchołka zapamiętuję krawędzi do niego wchodzące.\newline
5. Dla każdego wierzchołka zapamiętuję dodatkowo krawędzi z niego wychodzące.\newline
\newline
Właściwy algorytm:\newline
Wybieramy jedną (nieopróżnioną wcześniej) spójną składową która nie ma krawędzi wychodzących, dla każdego wierzchołka w niej zawartego,
bez przypisanej strategii wybieramy dowolny ruch pozostający w tej samej spójnej składowej
(do wierzchołka do niej należącego i nie wyrzuconego we wcześniejszym przebiegu).
Przy każdym przypisaniu strategii dla wierzchołka przeglądamy jego poprzedników odejmując u nich jedną krawędź
wychodzącą odpowiedniego typu. Jeżeli właściciel takiego poprzednika należącego do innej silnie spójnej składowej
preferuję parzystość taką jak parzystość tej składowej,
to wybranie rozważanego wcześniej wierzchołka jest dobrym wyborem strategii, a więc przypisujemy strategię temu poprzednikowi.
W przeciwnym przypadku (inna parzystość) sprawdzamy, czy w tym wierzchołku została niezerowa ilość krawędzi wychodzących
dowolnego typu, jeśli nie to również dobieramy ten ruch jako wybór strategii (jedyny pozostały).
Jeżeli na jeden z tych sposobów któremuś wierzchołkowi z innej spójnej składowej została przypisana strategia,
to niejako przypisujemy go do tej spójnej składowej (nadajemy mu jej parzystość i usuwamy go z jego składowej nie rozważając go w przyszłych krokach)
i podobnie jak dla wierzchołków należących do niej oryginalnie przeglądamy ich poprzedników (czyli po prostu znajdujemy atraktor).\newline
Jednocześnie ze zmniejszaniem liczby krawędzi wychodzących z wierzchołków, zmniejszamy liczbę krawędzi wychodzących
ze spójnej składowej do której należą (tylko w przypadku krawędzi miedzy składowymi).
Jeżeli dla spójnej składowej nie ma już krawędzi wychodzących poza nią, to jak w preprocessingu zapamiętujemy ją.\newline
Algorytm powtarzamy w pętli, aż nie będzie żadnej silnie spójnej składowej bez krawędzi wychodzących (a więc żadnych).\newline
\newline
Analiza złożoności:\newline
Krok 1 - Liniowy ze względu na rozmiar grafu.\newline
Krok 2 - Co najwyżej przeglądnięcie każdego wierzchołka raz - czas liniowy ze względu na ilość wierzchołków.\newline
Krok 3 - Przeglądnięcie każdej krawędzi raz (+ po 2 sprawdzenia w których składowych znajdują się końce) -
czas liniowy ze względu na ilość krawędzi.\newline
Krok 4 - Czas liniowy ze względu na ilość krawędzi.\newline
Krok 5 - Czas liniowy ze względu na rozmiar grafu.\newline
Pętla - Wybranie silnie spójnej składowej bez krawędzi wychodzących i zapamiętanie nowej w czasie stałym
- silnie spójnych składowych jest niewięcej niż wierzchołków, więc liniowe ze względu na ilość wierzchołków.
Skreslanie krawędzi (zapominanie) i zmniejszanie liczników krawędzi wychodzących w czasie liniowym ze względu na ilość krawędzi. 
Ustalanie strategii dla wierzchołków, sprawdzanie parzystości silnie spójnej składowej, zmienianie parzystości itp.
w czasie liniowym ze względu na rozmiar grafu.\newline
Łącznie czas $O(|V|+|E|)$\newline
\newline\newline
Analiza poprawności:\newline
W czasie działania algorytmu każdemu wierzchołkowi zostaje przypisana strategia i parzystość
(parzystość silnie spójnej składowej, lub nadpisana w trakcie działania algorytmu).
Parzystość wierzchołka oznacza który gracz wygrywa startując z tego wierzchołka.
Strategia wierzchołka zawsze kieruje gracza do wierzchołka o tej samej parzystości
(w konstrukcji albo zostajemy na zawsze w silniej spójnej składowej - pozostałej
po usunięciu wierzchołków w trakcie wykonania pętli dla innych składowych,
albo wierzchołkowi zostaje nadpisana parzystość na taką jaką ma jego następnik w strategii).
Pozostaje udowodnić, że gracz wygrywa w wierzchołku dla oryginalnej gry parzystości wtedy i
tylko wtedy gdy parzystość wierzchołka jest na jego korzyść.
Określona parzystość dla wierzchołka oznacza, że jeśli obaj gracze będą grali z wyznaczoną strategią,
to gra dojdzie do pewnej silnie spójnej składowej o tej parzystości i już z niej nie wyjdzie, a więc wygra gracz preferujący tą parzystość.
Aby gracz zmieniając strategię (przy zachowanej strategii przeciwnika) mógł wygrać
w wierzchołku ustalonym dla niego jako niekorzystny musiałby dojść przy nowej strategii w skończonym czasie do wierzchołka o korzystnej parzystości,
a więc po którymś kroku parzystość musiałaby się zmienić na korzyść gracza.
Jeżeli z wierzchołka gracza z inną parzystością niż ustalona dało by się przejść do wierzchołka z parzystością korzystną dla gracza,
to w konstrukcji byłaby mu przypisana inna parzystość.\newline
\newline
Część 2 - Czas działania algorytmu rekurencyjnego dla nudnych gier parzystości.\newline
\newline
Algorytm rekurencyjny na nudnych grach również osiąga pesymistyczną złożoność wykładniczą - przykład:\newline
\includegraphics[scale=0.9]{aatg_przyklad.jpg}\newline
Graf w którym gra tylko gracz parzysty (i wygrywa w każdym punkcie) składający się z jednej ścieżki złożonej z wierzchołków o malejących nieparzystych wagach,
zakończonej cyklem zerowym oraz z dołączonymi cyklami jedynkowymi (graf ten spełnia założenia nudności).\newline
Algorytm rekurencyjny wywoła się rekurencyjnie na grafie gry bez jednego (górnego lewego) wierzchołka o maksymalnym priorytecie
(sam jest swoim całym atraktorem), a następnie nastąpi drugie rekurencyjne wywołanie na grafie bez dolnego lewego wierzchołka
(czyli zbioru wygrywającego dla gracza nieparzystego w grze bez wierzchołka górnego lewego, będącego swoim całym atraktorem w oryginalnej grze).
Daje nam to po 2 wywołania na grafach bedących ścieżką o 1 krótszą (plus kilka dodatkowych na innych grafach), a więc w tym przypadku złożoność $\Omega(2^n)$, gdzie $2n$,
to ilość wierzchołków w grafie.








\end{document} 
