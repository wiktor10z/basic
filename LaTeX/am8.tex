\documentclass{article}
\usepackage[utf8]{inputenc}
\usepackage{polski}
\usepackage{amsmath}
\usepackage{anysize}
\usepackage{amssymb}
\marginsize{1,5cm}{2cm}{1cm}{3cm}
\begin{document}

Wiktor Zuba 320501 grupa 4
\newline

Zadanie 6.1.
\newline
\newline
$
M=\left\{t\left(
\begin{array}{c}
\alpha\cos{\alpha}\\\alpha\sin{\alpha}\\\alpha
\end{array}
\right)+(1-t)\left(
\begin{array}{c}
0\\0\\\alpha
\end{array}
\right)
:t\in(0,1),\alpha\in(0,2\pi)
\right\}
=\newline
\left\{(x,y,z):x=t\alpha\cos{\alpha},y=t\alpha\sin{\alpha},z=\alpha:t\in(0,1),\alpha\in(0,2\pi)\right\}
$
Bycie rozmiatością: istnieje homeomorfizm $\varphi:M\cap U\rightarrow $\~{O}$\subseteq R^2\quad\varphi(x,y,z)=\left(\frac{\sqrt{x^2+y^2}}{z},z\right)$\quad
$\varphi$ jest przekształceniem otwartym -można dobrać $z$ lekko większe i lekko mniejsze, aby uzyskać otwartość na drugiej współrzędnej, i dla każdego $z\in(0,2\pi)$
można dobrać $x$ i $y$ trochę bliżej lub dalej osi $Oz$ -na tym odcinku z treści zadania, różnowartościowa, bo dla $z_1\neq z_2$ różne wartości drugiej współrzednej
dla stałego $z$ i $(x_1,y_1)\neq(x_2,y_2)\sqrt{x_1^2+y_1^2}=\sqrt{x_2^2+y_2^2}$ tylko wtedy gdy "na płasko" leżą na tym samym okręgu,
ale to oznacza różne kąty, czyli różne $z$, $\varphi^{-1}(t,\alpha)=(t\alpha\cos{\alpha},t\alpha\sin{\alpha},\alpha)$ jest oczywiście $C^1$
reszta własności wynikaz tych lub oczywista.\newline
$
\iint\limits_{M}\frac{1}{z}\sqrt{x^2+y^2}dS
$ Zamiana zmiennych $\varphi^-1$ (poprawność już udowodniona)\newline
$
G=det\left(
\left[\begin{array}{ccc}
\alpha\cos{\alpha}&\alpha\sin{\alpha}&0\\
t(\cos{\alpha}-\alpha\sin{\alpha})&t(\sin{\alpha}+\alpha\cos{\alpha})&1\\
\end{array}\right]
\cdot
\left[\begin{array}{cc}
\alpha\cos{\alpha}&t(\cos{\alpha}-\alpha\sin{\alpha})\\
\alpha\sin{\alpha}&t(\sin{\alpha}+\alpha\cos{\alpha})\\
0&1\\
\end{array}\right]
\right)
=
det\left(
\left[\begin{array}{cc}
\alpha^2&t\alpha\\
t\alpha&t^2(1+\alpha^2)+1\\
\end{array}\right]
\right)
=
\alpha^2(1+t^2\alpha^2)
$\newline
$
\iint\limits_{t\in(0,1),\alpha\in(0,2\pi)}t\alpha\sqrt{1+t^2\alpha^2}dtd\alpha
$
Funkcja nieujemna-Fubini:\newline
$
\int\limits_{0}^{2\pi}\alpha\left(\int\limits_{0}^{1}t\sqrt{1+t^2\alpha^2}dt\right)d\alpha
=
\int\limits_{0}^{2\pi}\frac{1}{3\alpha}((1+\alpha^2)^{\frac{3}{2}}-1)d\alpha
$
Całka niezbyt ładna-uproszczenie poprzez kilka zamian zmiennych (można traktować jak całkę Riemmana),(oczywiste spełnianie założeń)
$
\alpha=\sqrt{x}\quad
\int\limits_{0}^{4\pi^2}\frac{1}{6x}((1+x)^{\frac{3}{2}}-1)dx\quad
x=y-1\quad
\frac{1}{6}\int\limits_{1}^{4\pi^2+1}\frac{1}{y-1}(y^{\frac{3}{2}}-1)dy\quad
y=z^2\quad
\frac{1}{3}\int\limits_{1}^{\sqrt{4\pi^2+1}}\frac{z^4-z}{z^2-1}dz\quad
=
\frac{1}{3}\int\limits_{1}^{\sqrt{4\pi^2+1}}z^2+1+\frac{1-z}{z^2-1}dz\quad
=
\frac{1}{3}\int\limits_{1}^{\sqrt{4\pi^2+1}}z^2+1-\frac{1}{z+1}dz\quad
=
\frac{1}{3}\left[\frac{z^3}{3}+z-\ln(1+z)\right]_{1}^{\sqrt{4\pi^2+1}}
=
\underline{\frac{1}{9}(4\pi^2+4)\sqrt{4\pi^2+1}-\frac{4}{9}+\frac{\ln{2}}{3}-\frac{1}{3}\ln(1+\sqrt{4\pi^2+1})}
$
\newline
\newline

Zadanie 6.2.
\newline
\newline
Łuk regularny nie musi być niestety krzywą klasy $C^2$ - wtedy wystarczyło by wziąć tubularne otoczenie i wersję twierdzenia dla wymiaru 2,
jest na szczęście rozmiatością ponieważ właściwie z definicji jest krzywą prostowalną
(właściwie nie wiem, czy mieliśmy gdzieś na ćwiczeniach, lub wykładzie (wcześniej byłem na innym potoku) podaną definicję więc korzystam z tej)
(w każdym punkcie posiada jednoznaczną pochodną).
W każdym bądź razie analogicznie do dowodu twierdzenia dla rozmiatości wymiaru 2:
$\varphi$ będzie parametryzacją tego łuku K z odcinka $(0,d)$, 
$\varphi(t)=(x(t),0,z(t))$
bryle obrotowej P automatycznie przyporządkowujemy funkcję
$\psi:(0,d)\times(0,2\pi)\rightarrow\mathbb{R}^3$ $\psi(t,\alpha)=(x(t)\cos{\alpha},x(t)\sin{\alpha},z(t))$ która jest parametryzacją P (ciągła i otwarta-oczywiste,
różnowartościowa-dla różnych t inne z(t) lub x(t), dla równych z(t)- niezależnie od $\alpha$ $x^2+y^2$ różne,
dla różnych $\alpha$ i tego samego x(t) różne wartości sin,cos, pochodna później przy okazji liczenia)
skoro P jest rozmaitością to możemy policzyć jej miarę (właściwie to $P\backslash K$, ale K ma miarę $l_2$ równą 0):
$\iint\limits_{P}dl_2(t,\alpha)$ podstawienie $\psi\newline
D\psi=\left[\begin{array}{cc}
x'(t)\cos{\alpha}&-x(t)\sin{\alpha}\\
x'(t)\sin{\alpha}&x(t)\cos{\alpha}\\
z'(t)&0\\
\end{array}\right]
G=((x'(t))^2+(z'(t))^2)\cdot(x(t))^2>0$ $\varphi$ bo $x(t)>0$, różnowartościowe więc któraś z pochodnych różna od 0.
Z Fubiniego będzie można skorzystać ponieważ miara jest nieujemna.
$
\iint\limits_{t\in(0,d),\alpha\in(0,2\pi)}\sqrt{((x'(t))^2+(z'(t))^2)\cdot(x(t))^2}dtd\alpha
=
2\pi\int\limits_{0}^{d}x(t)\sqrt{(x'(t))^2+(z'(t))^2}dt
=
2\pi\int\limits_{K}xdl_1(t)=2\pi d\cdot\frac{1}{d}\int\limits_{K}xdl_1(t)=\underline{d\cdot2\pi x_0}
$
\newpage

Wiktor Zuba 320501 grupa 4
\newline

Zadanie 6.3.
\newline
\newline
Nieruchomy okrąg o promieniu 3 i środku w (0,0), oraz okrąg o promieniu 1 styczny do pierwszego wewnątrznie, toczący się w nim bez poślizgu.
Podczas toczenia środek mniejszego okręgu przebiega drogę wokół środka większego będącą okręgiem o promieniu 2.
Podczas obrócenia się środka mniejszego okręgu względem środka większego o kąt $\alpha$ 
punkt styczności przesunie się względem większego okręcu o łuk $3\cdot\alpha$ względem wyjściowego to samo stanie się w mniejszym okręgu
tak więc każdy punkt mniejszego okręgu obróci się względem jego środka o kąt $3\alpha-\alpha=2\alpha$
(przesunięcie kątowe od punktu styczności-przesunięcie kątowe punktu styczności)w przeciwną stronę.
(Można też to było wywnioskować z treści zadania ("dotknie dwukrotnie")).\newline
Pozycja punktu wyjściowego(pozycja względem srodka małego okręgu+pozycja tego środka)w zależności od kąta $\alpha$:
$(2\cos{\alpha}+\cos{(-2\alpha)},2\sin{\alpha}+\sin{(-2\alpha)})=(2\cos{\alpha}+\cos{(2\alpha)},2\sin{\alpha}-\sin{(2\alpha)})$
Droga przebyta przez punkt:$s=\left\{(x,y):x=2\cos{\alpha}+\cos{(2\alpha)},y=2\sin{\alpha}-\sin{(2\alpha):\alpha\in(0,2\pi)}\right\}$\quad
Długość:
$
\int\limits_{s}dl_1(s)\newline
$
zamiana zmiennych na $\alpha$ jak wyżej(dyfeomorfizm $C^1$ na przedziałach (rozdzielamy całkę a następnie łączymy))\newline
$
\int\limits_{0}^{2\pi}\sqrt{((2\cos{\alpha}+\cos{(2\alpha)})')^2+((2\sin{\alpha}-\sin{(2\alpha)})')^2}d\alpha
=
\int\limits_{0}^{2\pi}\sqrt{(-2\sin{\alpha}-2\sin{(2\alpha)})^2+(2\cos{\alpha}-2\cos{(2\alpha)})^2}d\alpha
=
\int\limits_{0}^{2\pi}2\sqrt{\sin^2{\alpha}+\cos^2{\alpha}+\sin^2{(2\alpha)}+\cos^2{(2\alpha)}+2\sin{\alpha}\sin{(2\alpha)-2\cos{\alpha}\cos{(2\alpha)}}}d\alpha
=
2\int\limits_{0}^{2\pi}\sqrt{2-2\cos{(3\alpha)}}d\alpha
$
Podstawienie $\alpha=\frac{2\beta}{3}$ (oczywiście dobre)
$
\frac{4}{3}\int\limits_{0}^{3\pi}\sqrt{2-2\cos{(2\beta)}}d\beta
=
\frac{4}{3}\int\limits_{0}^{3\pi}\sqrt{4\sin^2{\beta}}d\beta
=
\frac{4}{3}\int\limits_{0}^{3\pi}|2\sin{\beta}|d\beta
=
8\int\limits_{0}^{\pi}\sin{\beta}d\beta
=
\underline{16}
$\newline
Co uogólnia się na $8(a-1)$ dla promienia większego okręgu a i obrotu środka mniejszego o $2\pi$ (rachunki analogiczne, zamiast 3 - a, zamiast 2 - a-1,
oprócz niektórych użyć 2 przy $2\pi$)
\newline
\newline

Zadanie 6.4.
\newline
\newline
$
\left\{(x,y,z):y+z=5x^2,yz=4x^4,y\le z\right\}\quad y(5x^2-y)=4x^4\quad (y=x^2,z=4x^2\vee y=4x^2,z=x^2),y\le z\Rightarrow y=x^2,z=4x^2\quad$
(dla każdego x tylko jeden taki punkt)\quad
$
C=\left\{(x,x^2,4x^2):x\in(-2,1)\right\}\newline
$
$
\int\limits_{C}(x+\sqrt{z})dy+(x+\sqrt{y})dz=\int\limits_{C}(x+\sqrt{z})dy+\int\limits_{C}(x+\sqrt{y})dz\quad
$
Parametryzacja dyfeo $C^1$,różniczka rzędu 1 (wyrazy macierzy: 2x,8x,1)
$
dy=2xdx,dz=8xdx\quad
\int\limits_{-2}^{1}(x+\sqrt{4x^2})2xdx+\int\limits_{-2}^{1}(x+\sqrt{x^2})8xdx
=
\int\limits_{-2}^{1}(x+2|x|)2x+(x+|x|)8xdx
=
\int\limits_{-2}^{1}10x^2+12x|x|dx
=
\int\limits_{-2}^{0}-2x^2dx+\int\limits_{0}^{1}22x^2dx
=
-\frac{16}{3}+\frac{22}{3}
=
\underline{2}
$
\end{document}